%&pdflatex 
\documentclass[12pt]{article} 
\usepackage[margin=1in]{geometry} 
\usepackage{amsmath,amsthm,amssymb,amsfonts,tikz-cd} 
\usepackage{enumerate}

\newenvironment{problem}[2][Problem]{\begin{trivlist}
\item[\hskip \labelsep {\bfseries #1}\hskip \labelsep {\bfseries #2.}]}{\end{trivlist}}

\newenvironment{proposition}[1][Proposition]{\begin{trivlist}
\item[\hskip \labelsep {\bfseries #1.}]}{\end{trivlist}}

\newenvironment{definition}[1][Definition]{\begin{trivlist}
\item[\hskip \labelsep {\bfseries #1.}]}{\end{trivlist}}

\newcommand{\til}{\char`\~}
\newcommand{\bs}{\textbackslash} 

\newcommand{\catname}[1]{\normalfont\textbf{#1}}
\newcommand{\catsup}[2]{\normalfont\textbf{#1}^{#2}}
\newcommand{\catsub}[2]{\normalfont\textbf{#1}_{#2}}

\newcommand{\Hom}{\text{Hom}}
\newcommand{\Homc}[2]{\Hom_{\catname{#1}}(#2)}

\newcommand{\Obj}{\text{Obj}}
\newcommand{\Objc}[1]{\text{Obj}(\catname{#1})}
\newcommand{\Aut}{\text{Aut}}
\newcommand{\End}{\text{End}}

\newcommand{\Int}[1]{\text{Int}\,#1}
\newcommand{\Cl}[2]{\text{Cl}_{#1}\,#2}

\newcommand{\id}{\text{id}}

\newcommand{\lcm}[1]{\text{lcm}(#1)}

\newenvironment{solution}
  {\renewcommand\qedsymbol{$\blacksquare$}\begin{proof}[Solution]}
{\end{proof}}

\newenvironment{sproof}{
  \renewcommand\qedsymbol{$\square$}
  \begin{proof}
  }{
  \end{proof}
}

\begin{document}

\title{Topology and Groupoids Exercises\\ \large Chapter 2, Section 4}
\author{David Melendez}
\maketitle

\begin{problem}{1}
  Prove that the relation '$X$ is a subspace of $Y$' is a partial
  order relation for topological spaces.
\end{problem}
\begin{solution}
  Denote by $\leq$ the relation 'is a subspace of'.
  Of course '$X$ is a subset of $Y$' is a partial order relation for
  sets, so we simply need to verify that the topologies line up as
  they should. We will be working with open set topologies.\\
  \indent First, consider $(X,\tau_1)$ as a subspace of $(X,\tau)$,
  where $\tau_1$ is the relative topology on $X$.
  If $U\in\tau_1$, then there exists a set $V\in\tau$ such that
  $U=V\cap X$.
  Since $X$ and $V$ are $\tau$-open, $U=V\cap X$ is also $\tau$-open,
  so $U\in\tau$.
  Hence $\tau_1\subseteq\tau$.
  Conversely, if $U\in\tau$, then of course $U=U\cap X\in\tau_1$,
  and so $\tau_1=\tau$.
  Therefore $X\leq X$, and so $\leq$ is reflexive.\\
  \indent Next, suppose $X\leq Y$ and $Y\leq X$.
  Then, in particular, $X\subseteq Y$ and $Y\subseteq X$, so $X=Y$,
  and so $\leq$ is antisymmetric.\\
  \indent Finally, suppose $(X,\tau_X)\leq(Y,\tau_Y)$ and
  $(Y,\tau_Y)\leq(Z,\tau_Z)$.
  Then $U\in\tau_X$ implies there exists some $V\in\tau_Y$ such that
  $U = V\cap X$.
  Further, since $V\in\tau_Y$, and $Y$ is a subspace of $Z$, there
  exists a $\tau_Z$-open $W$ such that $V=W\cup Y$.
  We then have $U=W\cup Y\cup X = W\cup X$ (since $X\subseteq Y$
  implies $Y\cup X=X$, and so $U$ is open in the relative
  topology on $X$ with respect to $Z$. \\
  \indent Conversely, suppose $U$ is open in $Z$.
  Then note that $U\cup X=U\cup(X\cup Y)=(U\cup Y)\cup X$ is open in $X$,
  since $Y\leq Z$ implies $U\cup Y$ is $\tau_Y$-open, and 
  $X\leq Y$ implies $(U\cup Y)\cup X$ is $\tau_X-open$.
  Thus $\leq$ is transitive, completing the proof that $\leq$ is a
  partial order of topological spaces.
\end{solution}

\begin{problem}{2}
  Prove that the set $I=\{x\in\mathbb{Q} : -\sqrt2\leq x\leq \sqrt2\}$
  is both open and closed in $\mathbb{Q}$.
\end{problem}
\begin{solution}
  Note that $I=\left(-\sqrt2,\sqrt2\right)\cap\mathbb{Q}
  =\left[-\sqrt2,\sqrt2\right]\cap\mathbb{Q}$, where the first interval is
  open in $\mathbb{R}$ and the second is closed in $\mathbb{R}$.
  The result then follows from the conditions on open and closed sets
  in a relative topology.
\end{solution}

\begin{problem}{3}
  Let $A$ be the subspace of $\mathbb{R}$ of point $1/n$ for 
  $n\in\mathbb{Z}\setminus\{0\}$.
  Prove that $A$ is discrete, but that the subspace $A\cup\{0\}$ of
  $\mathbb{R}$ is not discrete.
\end{problem}
\begin{solution}
  Note that if $1/n\in A$, then we have 
  $\displaystyle{\left\{\frac{1}{n}\right\}=\left(\frac{1}{n+1}, \frac{1}{n-1}\right)\cap A}$, where the interval
  is an open interval in $\mathbb{R}$.
  Hence all singletons in $A$ are open, implying that all subsets of $A$
  are open since the union of any family of open sets of $A$ is also open.
  This means $A$ is discrete.\\
  \indent Now, denote by $Y$ the subspace $A\cup\{0\}$ of $X$.
  Take $A$ as a subset of $Y$, and note that 
  $\Cl{Y}{A}=\overline{A}\cap Y = Y\neq A$ (note that every 
  $\mathbb{R}$-neighborhood of $0$ meets $A$, and so $0\in\overline{A}$);
  hence $A$ is not closed in $Y$, and its complement $\{0\}$ is not
  open in $Y$.
  Therefore $Y$ does not have the discrete topology.
\end{solution}

\begin{problem}{4}
  Prove that a subspace of a discrete space is discrete, and a subspace of an indiscrete space is indiscrete.
\end{problem}
\begin{solution}
  Let $Z$ be a discrete topological space, and let $A\subseteq Z$ 
  be a subspace.
  Then $x\in A$ implies $\{x\}=\{x\}\cap A$, where the singleton on the right
  side is open in $Z$; hence $\{x\}$ is open in $A$, implying every set in
  $A$ is open, and so $A$ is discrete.\\
  \indent On the other hand, let $X$ be an indiscrete topological space,
  and let $A\subseteq X$ be a subspace. 
  If $U\subseteq A$ is open, then there exists an open set $V$ in $X$
  such that $U=V\cap A$.
  Since $X$ is indiscrete, $V$ is either $\varnothing$ or $X$, and so
  $U$ is either $\varnothing\cap A=\varnothing$ or $X\cap A=A$;
  hence $A$ is indiscrete.
\end{solution}
  
\begin{problem}{5}
  Let $A$ be the subspace $[0,2]\setminus\{1\}$ of $\mathbb{R}$.
  Prove that $I=[0,1)$ is both open and closed in $A$.
\end{problem}
\begin{solution}
  This follows from the fact that $I=(-1, 1)\cap A=[-1,1]\cap A$. 
\end{solution}

\begin{problem}{6}
  Let $x\in X$ and let $A$ be a neighborhood (in $X$) of $x$.
  Prove that the neighborhoods in $A$ of $x$ are exactly the
  neighborhoods in $X$ of $x$ which are contained in $A$.
\end{problem}
\begin{solution}
  First, suppose $N$ is a neighborhood in $X$ of $x$ contained in $A$. 
  Then there exists an open set $U$ in $X$ such that $x\in U\subseteq N$.
  Note, then, that $U\cap A\subseteq N\cap A = N$ is an open set in $A$
  that contains $x$, and so $N$ is a neighborhood of $x$ in $A$.\\
  \indent Conversely, suppose $N$ is a neighborhood of $x$ in $A$, and so
  there exists an open set $U$ in $A$ such that $x\in U\cap A$.
  Since $U$ is open in $A$, there exists a set $V$ open in $X$ such
  that $U=V\cap A$.
  Hence, we have $V\cap A\subseteq N$ is a neighborhood of $x$ in $X$,
  and so $N$ is a neighborhood of $x$ in $X$ contained in $A$.
\end{solution}

\begin{problem}{7}
  Let $\leq$ be an order relation on the set $X$.
  If $A\subseteq X$ then the restriction of $\leq$ is an order relation on $A$.
  Show that it is not necessarily true that if $A,X$ have the order topologies, then $A$
  is a subspace of $X$.
  What is the order topology on $\mathbb{Q}$?
\end{problem}

\end{document}
