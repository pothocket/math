%&pdflatex 
\documentclass[12pt]{article} 
\usepackage[margin=1in]{geometry} 
\usepackage{amsmath,amsthm,amssymb,amsfonts,tikz-cd, mathrsfs, enumitem} 
%\usepackage{enumerate}

\newenvironment{problem}[2][Problem]{\begin{trivlist}
\item[\hskip \labelsep {\bfseries #1}\hskip \labelsep {\bfseries #2.}]}{\end{trivlist}}

\newenvironment{proposition}[1][Proposition]{\begin{trivlist}
\item[\hskip \labelsep {\bfseries #1.}]}{\end{trivlist}}

\newenvironment{definition}[1][Definition]{\begin{trivlist}
\item[\hskip \labelsep {\bfseries #1.}]}{\end{trivlist}}

\newcommand{\til}{\char`\~}
\newcommand{\bs}{\textbackslash} 

\newcommand{\catname}[1]{\normalfont\textbf{#1}}
\newcommand{\catsup}[2]{\normalfont\textbf{#1}^{#2}}
\newcommand{\catsub}[2]{\normalfont\textbf{#1}_{#2}}

\newcommand{\Hom}{\text{Hom}}
\newcommand{\Homc}[2]{\Hom_{\catname{#1}}(#2)}

\newcommand{\Obj}{\text{Obj}}
\newcommand{\Objc}[1]{\text{Obj}(\catname{#1})}
\newcommand{\Aut}{\text{Aut}}
\newcommand{\End}{\text{End}}

\newcommand{\Int}[1]{\text{Int}\,#1}
\newcommand{\Ints}[2]{\text{Int}_{#1}\,#2}
\newcommand{\Cl}[2]{\text{Cl}_{#1}\,#2}

\newcommand{\id}{\text{id}}

\newcommand{\lcm}[1]{\text{lcm}(#1)}

\newcommand{\N}{\mathbb{N}}
\newcommand{\Z}{\mathbb{Z}}
\newcommand{\Q}{\mathbb{Q}}
\newcommand{\R}{\mathbb{R}}
\newcommand{\C}{\mathbb{C}}

\newenvironment{solution}
  {\renewcommand\qedsymbol{$\blacksquare$}\begin{proof}[Solution]}
{\end{proof}}

\newenvironment{sproof}{
  \renewcommand\qedsymbol{$\square$}
  \begin{proof}
  }{
  \end{proof}
}

\begin{document}

\title{Topology and Groupoids Exercises\\ \large Chapter 2, Section 8}
\author{David Melendez}
\maketitle

\begin{problem}{8.3}
  Let $X,Y$ be metric spaces (their metrics will both be referred to as $d$)
  Show that the following functions define metrics on $X\times Y$
  whose metric topology is the product topology.
  \begin{enumerate}[label=(\alph*)]
    \item $D\left( (x,y),(x',y') \right)=d(x,x')+d(y,y')$ 
    \item $D\left( (x,y),(x',y') \right)=\sqrt{d(x,x')^2+d(y,y')^2}$
  \end{enumerate}
\end{problem}

\begin{proof}
  For (a), first let $N$ be a $D$-neighbourhood of $(a,b)\in X\times Y$.
  Then there exists an $\varepsilon>0$ such that 
  $B_D\left( (a,b), \varepsilon\right)\subseteq N$; that is, if
  $(x,y)\in B_D\left( (a,b),\varepsilon \right)$, then 
  \begin{align*}
    D\left( (x,y),(a,b) \right) &= d(x,a) + d(y,b) \\
    &< \varepsilon.
  \end{align*}
  Let $$M=\displaystyle B_d\left(a,\frac{\varepsilon}{2}\right)
  \times B_d\left(b,\frac{\varepsilon}{2}\right).$$
  We then have that if $(x,y)\in M$, then $d(x,a)<\varepsilon/2$ and 
  $d(y,b)<\varepsilon/2$, and so 
  \begin{align*}
    D\left( (x,y),(a,b) \right) &= d(x,a) + d(y,b) \\
    &< \frac{\varepsilon}{2} + \frac{\varepsilon}{2} \\
      &= \varepsilon,
  \end{align*}
  and so we have that $M\subseteq B_D\left( (a,b),\varepsilon \right)
  \subseteq N$, and so $N$ is a product neighbourhood of $(a,b)$.
  Thus every $D$-neighbourhood of $(a,b)$ is a product neighbourhood of 
  $(a,b)$. \\
  \indent Conversely, suppose $N$ is a product neighbourhood of 
  $(a,b)\in X\times Y$.
  Then there exist $\varepsilon_1,\varepsilon_2>0$ such that
  $B_d(a,\varepsilon_1)\times B_d(b,\varepsilon_2)\subseteq N$.
  If we let $\varepsilon=\min\{\varepsilon_1,\varepsilon_2\}$, then we have
  that if $(x,y)\in B_D\left( (a,b),\varepsilon \right)$, then
  \begin{align*}
    D\left( (a,b),(x,y) \right) &= d(a,x) + d(b,y) \\
    &< \varepsilon;
  \end{align*}
  and so certainly $d(a,x)<\varepsilon$ and $d(b,y)<\varepsilon$.
  Consequently, we have that $x\in B_d(a,\varepsilon)$ and
  $y\in B_d(b,\varepsilon)$, and so 
  $(x,y)\in N$.
  Therefore, we have that $B_D\left( (a,b),\varepsilon \right)\subseteq N$,
  and so $N$ is a $D$-neighbourhood of $(a,b)$, as desired.\\
  \indent
  For (b), let $(a,b)\in X\times Y$ and assume $N$ is a $D$-ball about
  $(a,b)$; that is, there exists an $\varepsilon>0$ such that
  \begin{equation*}
    N=B_D\left( (a,b),\varepsilon \right).
  \end{equation*}
  Then, let $\delta = \varepsilon/\sqrt2$, and let $M$ be the set
  \begin{equation*}
    M=B_d(a,\delta)\times B_d(b,\delta)\subseteq X\times Y
  \end{equation*}
  \indent We then have that if $(x,y)\in M$, then $d(x,a)<\delta$
  and $d(y,b)<\delta$, and so
  \begin{align*}
    d(x,a)^2 + d(y,b)^2 &< 2\delta^2 \\
    &= \varepsilon^2.
  \end{align*}
  Therefore, we have that $D\left( (a,b),(x,y) \right)<\varepsilon$,
  and so $(x,y)\in N$.
  Consequently, we have that $M\subseteq N$, and so $N$ is a product
  neighbourhood of $(a,b)$.
  Thus, every $D$-neighbourhood is a product neighbourhood.\\
  \indent Suppose conversely that $N$ is a basic product neighbourhood of \
  $(a,b)$, and so \\$N=B_d(a,\varepsilon_1)\times B_d(b,\varepsilon_2)$
  for some $\varepsilon_1,\varepsilon_2 > 0$.
  Then, let $\delta=\min\{\varepsilon_1,\varepsilon_2,1\}$.
  Note that $\delta<1$, and let $M=B_D\left( (a,b),\delta \right).$
  We then have that if $(x,y)\in M$, then
  \begin{alignat*}{2}
    & B_d\left( (a,b),(x,y) \right) &&< \delta \\
    \implies & \sqrt{d(a,x)^2+d(b,y)^2} &&< \delta \\
    \implies & d(a,x)^2 + d(b,y)^2 &&< \delta^2 \\
    \implies & d(a,x)^2 &&< \delta^2 - d(b,y)^2 \\
    \implies & d(a,x) &&< \sqrt{\delta^2-d(b,y)^2} \\
    &&&< \sqrt{\varepsilon_1^2-d(b,y)^2}\\
    &&&< \sqrt{\varepsilon_1^2}\\
    &&&= \varepsilon_1.
  \end{alignat*}
  Similarly, we have that $d(b,y)<\varepsilon_2$.
  Thus, $(x,y)\in N$, and so we have that $M\subseteq N$, showing that
  $N$ is a $D$-neighbourhood of $(a,b)$.
  Consequently, every product neighbourhood is a $D$-neighbourhood.\\ 
  \indent Since product neighbourhoods and $D$-neighbourhoods coincide,
  it then follows that the metric topology induced by $D$ is the
  product topology, as desired.
\end{proof}

\end{document}
