%&pdflatex 
\documentclass[12pt]{article} 
\usepackage[margin=1in]{geometry} 
\usepackage{amsmath,amsthm,amssymb,amsfonts,tikz-cd} 
\usepackage{enumerate}

\newenvironment{problem}[2][Problem]{\begin{trivlist}
\item[\hskip \labelsep {\bfseries #1}\hskip \labelsep {\bfseries #2.}]}{\end{trivlist}}

\newenvironment{proposition}[1][Proposition]{\begin{trivlist}
\item[\hskip \labelsep {\bfseries #1.}]}{\end{trivlist}}

\newenvironment{definition}[1][Definition]{\begin{trivlist}
\item[\hskip \labelsep {\bfseries #1.}]}{\end{trivlist}}

\newcommand{\til}{\char`\~}
\newcommand{\bs}{\textbackslash} 

\newcommand{\catname}[1]{\normalfont\textbf{#1}}
\newcommand{\catsup}[2]{\normalfont\textbf{#1}^{#2}}
\newcommand{\catsub}[2]{\normalfont\textbf{#1}_{#2}}

\newcommand{\Hom}{\text{Hom}}
\newcommand{\Homc}[2]{\Hom_{\catname{#1}}(#2)}

\newcommand{\Obj}{\text{Obj}}
\newcommand{\Objc}[1]{\text{Obj}(\catname{#1})}
\newcommand{\Aut}{\text{Aut}}
\newcommand{\End}{\text{End}}

\newcommand{\Int}[1]{\text{Int}\,#1}

\newcommand{\id}{\text{id}}

\newcommand{\lcm}[1]{\text{lcm}(#1)}

\newenvironment{solution}
  {\renewcommand\qedsymbol{$\blacksquare$}\begin{proof}[Solution]}
{\end{proof}}

\newenvironment{sproof}{
  \renewcommand\qedsymbol{$\square$}
  \begin{proof}
  }{
  \end{proof}
}

\begin{document}

\title{Topology and Groupoids Exercises\\ \large Chapter 2, Section 2}
\author{David Melendez}
\maketitle

\begin{problem}{1}
  What are the open sets of $X$ hen $X$ is discrete, that is, has the discrete topology?,
  is indiscrete, that is, has the indescrete topology?
  What is the closure of \\ $\{x\}, x\in X,$ in these cases?
\end{problem}
\begin{solution}
  Recall that under the discrete topology, a set $N\subseteq X$ is a neighborhood of a point $x\in X$
  if and only if $x\in N$; that is, $\Int{N} = N$.
  Hence, under the discrete topology, every set is open. \\\indent
  On the other hand, under the indescrete toplogy, a set $N\subseteq X$ is a neighborhood of a point $x\in X$
  if and only if $N=X$ and $x\in N$.
  Here, the only open sets are $\null$ and $X$ itself.
\end{solution}

\begin{problem}{2}
  Let $X$ be a topological space and let $A\subseteq X$. 
  Prove that $\Int{A}$ is the union of all open sets $U$ such that 
  $U\subseteq A$ and $\overline{A}$ is the intersection of all closed sets $C$ such that $A\subseteq C$.
\end{problem}
\begin{proof}
  First, note that $x\in\Int{A}$ if and only if there exists some 
  $U\subseteq A$ such that $x\in U$, if and only if
  $x\in \mathcal{U},$
  where $\mathcal{U}$ is the family of all open sets containing $A$.
  
  \indent For closed sets, first suppose $x\in \overline{A}$, 
  and so every neighborhood of $x$ meets $A$.
  If $C$ is a closed subset of $X$ containing $A$, then every neighborhood of $x$ must also meet $C$,
  implying $x\in\overline{C}=C$ since $C$ is closed. 
  Hence $x$ is in the intersection of all closed sets containing $A$,
  completing the inclusion in one direction.\\
  \indent Conversely, if $x\notin\overline{A}$, then $\overline{A}$ itself is a closed set containing $A$
  that does not contain $x$, so $x$ is certainly not in the intersection of all closed sets containing $A$.
  Thus, $\overline{A}$ is the intersection of all closed sets in $X$ that contain $A$ as a subset.
\end{proof}
\indent The previous result essentially means that the interior of $A$ is the largest open set within $A$,
and the closure of $A$ is the smallest closed set containing $A$.

\begin{problem}{3}
  Let $X$ be a topological space, and let $A\subseteq X$.
  A point $x\in X$ is called a \textit{limit point} of $A$ if each neighborhood of $x$ contains
  points of $A$ other than $x$.
  The set of limit points of $A$ is written $\widehat{A}$.
  Prove that $\overline{A} = A\cup\widehat{A}$, and that $A$ is closed iff $\widehat{A}\subseteq A$.
  Give examples of non-empty subsets $A$ of $\mathbb{R}$ such that:
  \begin{enumerate}[(i)]
    \item $\widehat{A} = \varnothing$
    \item $\widehat{A}\neq\varnothing$ and $\widehat{A}\subseteq A$
    \item $A$ is a proper subset of $\widehat{A}$
    \item $\widehat{A}\neq\varnothing$ but $A\cap\widehat{A}=\varnothing$
  \end{enumerate}
\end{problem}
\begin{solution}
  First we will prove that $\overline{A}=A\cup\widehat{A}$.
  \begin{sproof}
    First suppose that $x\in\overline{A}$.
    Then, by definition, every neighborhood of $x$ meets $A$.
    If $x$ is not in $A$, then that every neighborhood of $x$ meets $A$ means that every neighborhood of
    $x$ contains points of $A$ that aren't $x$, meaning $x\in\widehat{A}$.
    Hence $\overline{A}\subseteq A\cup\widehat{A}$.\\\indent
    Conversely, suppose $x\in A\cup\widehat{A}$.
    If $x\in A$, then every neighborhood of $x$ contains $x\in A$.
    If $x\in \widehat{A}$, then every neighborhood of $x$ contains a point in $A$.
    Hence $A\cup\widehat{A}\subseteq \overline{A}$, and so $\overline{A} = A\cup\widehat{A}$.
  \end{sproof}
  Next, we will prove that $A$ is closed iff $\widehat{A}\subseteq A$.
  \begin{sproof}
    $A$ is closed iff $A=\overline{A}$, meaning $A=A\cup\widehat{A}$, whence $\widehat{A}\subseteq{A}$.
  \end{sproof}
  Now, we produce each of the examples requested:
  \begin{enumerate}[(i)]
    \item Let $A$ be the singleton set $\{0\}$.
      Obviously every neighborhood of 0 contains 0, so 0 is not a limit point of $A$.
      If $x\neq0$, then the open interval $(x-|x|, x+|x|)$ does not contain $0$, 
      so $x$ is not a limit point of $A$.

    \item Let $A = [0, 1] \cup \{2\}$.
      Then $\widehat{A} = [0,1]\subseteq A$.

    \item Let $A=(0,1)$. 
      Then $\widehat{A}=[0,1]\supset A$.

    \item Let $A = \left\{ 1/n\  |\  n\in\mathbb{N} \right\}$.
      Then for any $1/n\in A$, the interval $\left(\frac{1}{n}-\delta,\frac{1}{n}+\delta\right)$ 
      with $\delta=\frac{1}{n} - \frac{1}{n+1}$ contains only $1/n\in A$, 
      and so $A\cap\widehat{A}=\varnothing.$
      However, by the Archimedean property of the real numbers, there exists for every 
      $\varepsilon>0$ an $m\in\mathbb{N}$ such that $\frac{1}{m} < \varepsilon$.
      Hence every open interval containing $0$ also contains a point in $A$, and so $0\in\widehat{A}$. 
  \end{enumerate}
\end{solution}

\begin{problem}{4}
  Let $X$ be a topological space and let $A\subseteq B\subseteq X$.
  We say that $A$ is \textit{dense} in $B$ if $B\subseteq\overline{A}$, 
  and $A$ is \textit{dense} if $\overline{A}=X$.
  Prove that if $A$ is dense in $X$ and $U$ is open then 
  $$U\subseteq \overline{A\cap U}.$$
\end{problem}
\begin{proof}
  Recall from the previous exercise that $\overline{A}=A\cup\widehat{A}$.
  Since $A$ is dense in $X$, we then know from this that $X=A\cup\widehat{A}$.
  With this in mind, we proceed to consider the two cases for any $x\in U\subseteq X$:\\
  \indent If $x\in A$, then of course $x\in U\cap A\subseteq\overline{U\cap A}$.\\
  \indent Otherwise, suppose $x\in \widehat{A}$, and let $N$ be any neighborhood of $x$.
  Since $U$ is open, we know that $N\cap U$ is also a neighborhood of $x$, 
  and because $x\in\widehat{A}$, we know that $N\cap U$ meets $A$; 
  that is, there exists an $a\in N\cap U\cap A$ with $a\neq x$.
  Clearly, then, this $a$ is also an element of $U\cap A$, allowing us to conclude that
  every neighborhood of $x$ meets $U\cap A$, and so $x\in \overline{U\cap A}$, as desired.
\end{proof}

\begin{problem}{5}
  Let $\mathbb{I} = [0,1]$.
  Define an order relation $\leq$ on $\mathbb{I}^2 = \mathbb{I}\times\mathbb{I}$ by
  $$(x,y)\leq(x',y') \Leftrightarrow y < y' \text{ or } (y=y' \text{ and } x\leq x').$$
  \indent The \textit{television topology} on $\mathbb{I}^2$ is the order topology 
  with respect to $\leq$.
  Let A be the set of points $(1/2, 1-n^{-1})$ for positive integral $n$.
  Prove that in the television topology on $\mathbb{I}^2$,
  $$\overline{A} = A\cup\{(0,1)\}.$$
\end{problem}
\begin{solution}
  (Sketch)
  Note that an interval of a point $p=(x,y)\in\mathbb{I}^2$ 
  (with respect to the television order) which
  also contains the point $(x',y')\in\mathbb{I}^2$ for $y'<y$ will contain
  the "vertical interval" $\mathbb{I}\times ]y',y[$, where $]y',y[$
  is an open interval with respect to the usual order topology on $\mathbb{R}$.\\
  \indent Since $p=(x,y) \leq (0,1)$ if and only if $y < 1$, we then know that any neighborhood
  of $p$ contains the open $\mathbb{I}^2$-interval $\mathbb{I}^2\times]r, 1[$ for some real number
  $0\leq r < 1$.
  By the Archimedean propoerty of the reals, there exists some positive integral $n$ such that
  $n^{-1} < 1-r$, whence $1-n^{-1} > r$, and so the point $(1/2, 1-n^{-1})$ is contained in the 
    neighborhood in question.
    Hence the point $(0,1)$ is in $\overline{A}$.\\
    \indent Points $(x,y)$ with $y<1$ are not contained in $\overline{A}$ 
    since $A$ will contain some points above $\mathbb{I}^2$-neighborhoods about $(x,y)$ 
    that don't stretch to the top of $\mathbb{I}^2$.
    Neither will points with $x>0$, since there exist $\mathbb{I}^2$-neighborhoods about this point
    which only contain points of one $y$-value.
\end{solution}

\begin{problem}{7}
  Prove that if $A$ is the closure of an open set, then $A = \overline{\Int{A}}$.
\end{problem}
\begin{proof}
  First, suppose that $x\in A = \overline{U}$, where $U\subset X$ is open.
  By definition, this means that every neighborhood of $x$ meets $U = \Int{U}$.
  Since the interior operator preserves inclusions, we know that $U\subseteq\overline{U}$
  implies $\Int{U}\subseteq\Int\overline{U}$, and so, continuing our previous line of reasoning,
  every neighborhood of $x$ meets $\Int\overline{U}$, meaning 
  $x\in\overline{\Int\overline{U}}=\overline{\Int{A}}$, 
  by the definition of closure.
  Therefore, $A\subseteq\overline{\Int{A}}$.\\
  \indent Conversely, suppose $x\in\overline{\Int{A}}$.
  Then, by the definition of closure, every neighborhood of $x$ meets $\Int{A}$, and so
  every neighborhood of $x$ meets $A$, whence $x\in\overline{A}$, which equals $A$
  since $A$ is closed.
  Thus $\overline{\Int\overline{A}}\subseteq A$, and so $\overline{\Int\overline{A}}=A$,
  completing the proof.
\end{proof}

\begin{problem}{9}
  A topological space $H$ (the \textit{half-open topology}) is defined as follows.
  The underlying set of $H$ is $\mathbb{R}$, and for each $x\in H$ and $N\subseteq H$, 
  $N$ is a neighborhood of $x$ iff there are real numbers $a$ and $b$ such that
  \begin{equation*}
    x\in[a,b[\ \subseteq N.
  \end{equation*}
  \indent Prove hat $H$ is a topological space and that:
  \begin{enumerate}
    \item Each interval $[a,b[$ is both open and closed
    \item $H$ is separable
    \item If $A\subseteq H$, then $A\setminus\widehat{A}$ is countable
  \end{enumerate}
\end{problem}
\begin{solution}
  To begin, we verify that $H$ is a topological space using the neighborhood topology axioms.

  \indent First, if $N$ is a neighborhood of $x$, then there exist $a,b$ in $\mathbb{R}$
  such that $x\in[a,b[\ \subseteq N$, meaning $x\in N$. \\
  \indent Next, if $N\subseteq\mathbb{R}$ contains a neighborhood $M$ of $x$, then
  there exists an interval $[a,b[$ with $x\in[a,b[\ \subseteq M\subseteq N$, whence
  $M$ is a neighborhood of $x$. \\
  \indent Next, suppose $N_1$ and $N_2$ are neighborhoods of $x$ so that 
  $x\in[a_1,b_1[\ \subseteq N_1$, and $x\in[a_2,b_2[\ \subseteq N_2$.
  We then have $x\in[\max(a_1,a_2),\min(b_1,b_2)[\ \subseteq{N_1\cap N_2}$, and so
  $N_1\cap N_2$ is a neighborhood of $x$. \\
  \indent Finally, let $N$ be a neighborhood of $x$, and let $I=[a,b[\ \subseteq N$ be
      an interval containing $x$.
      Then $I$ itself is a subset of $N$ containing $x$ such that $N$ is a 
      neighborhood of every point of $I$.
  Now we prove each of the additional statements:
  \begin{enumerate}
    \item Each interval $[a,b[$ is both open and closed.
          \begin{sproof}
            Note that if $x\in [a,b[$, then $[a,b[$ is itself a half-open interval 
            within $[a,b[$ containing $x$, and so $\Int[a,b[\ =[a,b[$.
            Hence $[a,b[$ is open.
            Additionally, $H\setminus[a,b[\ = ]-\infty, a[\ \cup\ [b, \infty[$, which is
            the union of two open sets, and hence is open.
            Therefore $[a,b[$ is also closed.
          \end{sproof}
    \item $H$ is separable.
    \begin{sproof}
        Consider the rationals $\mathbb{Q}\subseteq H$.
        Since half open intervals $[a,b[$ cannot contain a single element, every neighborhood of
        an irrational number contains a rational number, and so $\overline{\mathbb{Q}} =  H$.
        Hence $H$ is separable.
    \end{sproof}
  \end{enumerate}


\end{solution}

\end{document}
