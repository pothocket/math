%&pdflatex 
\documentclass[12pt]{article} 
\usepackage[margin=1in]{geometry} 
\usepackage{amsmath,amsthm,amssymb,amsfonts,tikz-cd} 
\usepackage{enumerate}

\newenvironment{problem}[2][Problem]{\begin{trivlist}
\item[\hskip \labelsep {\bfseries #1}\hskip \labelsep {\bfseries #2.}]}{\end{trivlist}}

\newenvironment{proposition}[1][Proposition]{\begin{trivlist}
\item[\hskip \labelsep {\bfseries #1.}]}{\end{trivlist}}

\newenvironment{definition}[1][Definition]{\begin{trivlist}
\item[\hskip \labelsep {\bfseries #1.}]}{\end{trivlist}}

\newcommand{\til}{\char`\~}
\newcommand{\bs}{\textbackslash} 

\newcommand{\catname}[1]{\normalfont\textbf{#1}}
\newcommand{\catsup}[2]{\normalfont\textbf{#1}^{#2}}
\newcommand{\catsub}[2]{\normalfont\textbf{#1}_{#2}}

\newcommand{\Hom}{\text{Hom}}
\newcommand{\Homc}[2]{\Hom_{\catname{#1}}(#2)}

\newcommand{\Obj}{\text{Obj}}
\newcommand{\Objc}[1]{\text{Obj}(\catname{#1})}
\newcommand{\Aut}{\text{Aut}}
\newcommand{\End}{\text{End}}

\newcommand{\Int}{\text{Int}}

\newcommand{\id}{\text{id}}

\newcommand{\lcm}[1]{\text{lcm}(#1)}

\newenvironment{solution}
  {\renewcommand\qedsymbol{$\blacksquare$}\begin{proof}[Solution]}
{\end{proof}}

\newenvironment{sproof}{
  \renewcommand\qedsymbol{$\square$}
  \begin{proof}
  }{
  \end{proof}
}

\begin{document}

\title{Topology and Groupoids Exercises\\ \large Chapter 2, Section 2}
\author{David Melendez}
\maketitle

\begin{problem}{1}
  What are the open sets of $X$ hen $X$ is discrete, that is, has the discrete topology?,
  is indiscrete, that is, has the indescrete topology?
  What is the closure of \\ $\{x\}, x\in X,$ in these cases?
\end{problem}
\begin{solution}
  Recall that under the discrete topology, a set $N\subseteq X$ is a neighborhood of a point $x\in X$
  if and only if $x\in N$; that is, $\Int{N} = N$.
  Hence, under the discrete topology, every set is open. \\\indent
  On the other hand, under the indescrete toplogy, a set $N\subseteq X$ is a neighborhood of a point $x\in X$
  if and only if $N=X$ and $x\in N$.
  Here, the only open sets are $\null$ and $X$ itself.
\end{solution}

\begin{problem}{2}
  Let $X$ be a topological space and let $A\subseteq X$. 
  Prove that $\Int{A}$ is the union of all open sets $U$ such that 
  $U\subseteq A$ and $\overline{A}$ is the intersection of all closed sets $C$ such that $A\subseteq C$.
\end{problem}
\begin{proof}
  First, note that $x\in\Int{A}$ if and only if there exists some 
  $U\subseteq A$ such that $x\in U$, if and only if
  $x\in \mathcal{U},$
  where $\mathcal{U}$ is the family of all open sets containing $A$.
  
  \indent For closed sets,
\end{proof}<++>

\begin{problem}{3}
  Let $X$ be a topological space, and let $A\subseteq X$.
  A point $x\in X$ is called a \textit{limit point} of $A$ if each neighborhood of $x$ contains
  points of $A$ other than $x$.
  The set of limit points of $A$ is written $\widehat{A}$.
  Prove that $\overline{A} = A\cup\widehat{A}$, and that $A$ is closed iff $\widehat{A}\subseteq A$.
  Give examples of non-empty subsets $A$ of $\mathbb{R}$ such that:
  \begin{enumerate}[(i)]
    \item $\widehat{A} = \varnothing$
    \item $\widehat{A}\neq\varnothing$ and $\widehat{A}\subseteq A$
    \item $A$ is a proper subset of $\widehat{A}$
    \item $\widehat{A}\neq\varnothing$ but $A\cap\widehat{A}=\varnothing$
  \end{enumerate}
\end{problem}
\begin{solution}
  First we will prove that $\overline{A}=A\cup\widehat{A}$.
  \begin{sproof}
    First suppose that $x\in\overline{A}$.
    Then, by definition, every neighborhood of $x$ meets $A$.
    If $x$ is not in $A$, then that every neighborhood of $x$ meets $A$ means that every neighborhood of
    $x$ contains points of $A$ that aren't $x$, meaning $x\in\widehat{A}$.
    Hence $\overline{A}\subseteq A\cup\widehat{A}$.\\\indent
    Conversely, suppose $x\in A\cup\widehat{A}$.
    If $x\in A$, then every neighborhood of $x$ contains $x\in A$.
    If $x\in \widehat{A}$, then every neighborhood of $x$ contains a point in $A$.
    Hence $A\cup\widehat{A}\subseteq \overline{A}$, and so $\overline{A} = A\cup\widehat{A}$.
  \end{sproof}
  \newpage
  Next, we will prove that $A$ is closed iff $\widehat{A}\subseteq A$.
  \begin{sproof}
    $A$ is closed iff $A=\overline{A}$, meaning $A=A\cup\widehat{A}$, whence $\widehat{A}\subseteq{A}$.
  \end{sproof}
  Now, we produce each of the examples requested:
  \begin{enumerate}[(i)]
    \item Let $A$ be the singleton set $\{0\}$.
      Obviously every neighborhood of 0 contains 0, so 0 is not a limit point of $A$.
      If $x\neq0$, then the open interval $(x-|x|, x+|x|)$ does not contain $0$, 
      so $x$ is not a limit point of $A$.

    \item Let $A = [0, 1] \cup \{2\}$.
      Then $\widehat{A} = [0,1]\subseteq A$.

    \item Let $A=(0,1)$. 
      Then $\widehat{A}=[0,1]\supset A$.

    \item Let $A = \left\{ 1/n\  |\  n\in\mathbb{N} \right\}$.
      Then for any $1/n\in A$, the interval $\left(\frac{1}{n}-\delta,\frac{1}{n}+\delta\right)$ 
      with $\delta=\frac{1}{n} - \frac{1}{n+1}$ contains only $1/n\in A$, 
      and so $A\cap\widehat{A}=\varnothing.$
      However, by the Archimedean property of the real numbers, there exists for every 
      $\varepsilon>0$ an $m\in\mathbb{N}$ such that $\frac{1}{m} < \varepsilon$.
      Hence every open interval containing $0$ also contains a point in $A$, and so $0\in\widehat{A}$. 
  \end{enumerate}
\end{solution}

\end{document}
