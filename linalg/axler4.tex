%&pdflatex 
\documentclass[12pt]{article} 
\usepackage[margin=1in]{geometry} 
\usepackage{amsmath,amsthm,amssymb,amsfonts,tikz-cd} 

\newenvironment{problem}[2][Problem]{\begin{trivlist}
\item[\hskip \labelsep {\bfseries #1}\hskip \labelsep {\bfseries #2.}]}{\end{trivlist}}

\newenvironment{proposition}[1][Proposition]{\begin{trivlist}
\item[\hskip \labelsep {\bfseries #1.}]}{\end{trivlist}}

\newenvironment{definition}[1][Definition]{\begin{trivlist}
\item[\hskip \labelsep {\bfseries #1.}]}{\end{trivlist}}

\newcommand{\til}{\char`\~}
\newcommand{\bs}{\textbackslash} 
\newcommand{\nul}{\text{null}\ }
\newcommand{\range}{\text{range}\ }

\newenvironment{solution}
  {\renewcommand\qedsymbol{$\blacksquare$}\begin{proof}[Solution]}
{\end{proof}}

\newenvironment{sproof}{
  \renewcommand\qedsymbol{$\square$}
  \begin{proof}
  }{
  \end{proof}
}

\begin{document}

\title{Linear Algebra Done Right Exercises\\ \large Chapter 4}
\author{David Melendez}
\maketitle

\begin{problem}{5}
  Suppose $m$ is a nonnegative integer, $z_1,\dots,z_{m+1}$ are distinct elements of $\mathbb{F}$, and
  $w_1,\dots,w_{m+1} \in \mathbb{F}$. Prove that there exists a unique polynomial 
  $p\in \mathcal{P}_m(\mathbb{F})$ such that
  \begin{equation*}
    p(z_j) = w_j
  \end{equation*}
  for $j = 1,\dots,m+1$.
\end{problem}
\begin{proof}
  Let $T \in \mathcal{L}(\mathcal{P}_m(\mathbb{F}), \mathbb{F}^X)$ where $X = \{z_1,\dots,z_{m+1}\}$. 
  Define $T$ by
  \begin{equation*}
    T(p)(x_j) = p(x_j)\\
  \end{equation*}
  It is easy to show that this transformation is linear.\\
  Now we compute the null space of $T$.
  Note that when $Tp=0$, we have that $p(x)=0$ for all $x\in X$, 
  and thus $p$ has at least $|X| = m+1$ distinct zeroes. 
  But since the degree of $p$ is at most $m$, this must mean that $p$ is the zero polynomial.
  Hence $T$ is injective and $\dim\nul T = 0$. 
  We then have
  \begin{align*}
    \dim\range T &= \dim \mathcal{P}_m(\mathbb{F}) - \dim\nul T\\
    &= m + 1 \\
    &=\footnotemark \dim \mathbb{F}^X\text{.}
  \end{align*}
  \footnotetext{See the following Stack Exchange post:
  https://math.stackexchange.com/questions/2288812/finding-dimension-of-a-vector-space-v/2289241\#2289241}
  This shows that $\range T = \mathbb{F}$, and therefore that $T$ is an isomorphism between
  $\mathcal{P}_m(\mathbb{F})$ and $\mathbb{F}^X$. 
  Since $T^{-1}$ assigns a unique polynomial to each function 
  (i.e. set of ordered pairs distinct in the first slot) 
  as stated in the problem, this completes the proof.
\end{proof}
\begin{problem}{6}
  Suppose $p\in\mathcal{P}(\mathbb{C})$ has degree $m$. 
  Prove that $p$ has $m$ distinct zeros if and only if $p$ and its derivative $p'$ have no zeros in common.
\end{problem}
\begin{proof}
  We will prove the contrapositive.
  Suppose $z$ is a zero of both $p$ and $p'$.
  Note that
  \begin{align*}
    p(x) &= (x-z)q(x)\\
    p'(x) &= q(x) + (x-z)q'(x)
  \end{align*}
  We then have
  \begin{align*}
    0 &= p'(z) \\
    &= q(z)
  \end{align*}
  and hence $z$ is a zero of $q$. 
  This means that $(x-z)^2$ is a factor of $p$, and therefore it can have at most $m-1$ distinct zeros.
\end{proof}
\end{document}

