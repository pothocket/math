%&pdflatex 
\documentclass[12pt]{article} 
\usepackage[margin=1in]{geometry} 
\usepackage{amsmath,amsthm,amssymb,amsfonts,tikz-cd, mathrsfs} 

\newenvironment{problem}[2][Problem]{\begin{trivlist}
\item[\hskip \labelsep {\bfseries #1}\hskip \labelsep {\bfseries #2.}]}{\end{trivlist}}

\newenvironment{proposition}[1][Proposition]{\begin{trivlist}
\item[\hskip \labelsep {\bfseries #1.}]}{\end{trivlist}}

\newenvironment{definition}[1][Definition]{\begin{trivlist}
\item[\hskip \labelsep {\bfseries #1.}]}{\end{trivlist}}

\newcommand{\til}{\char`\~}
\newcommand{\bs}{\textbackslash} 

\newcommand{\catname}[1]{\normalfont\textbf{#1}}
\newcommand{\catsup}[2]{\normalfont\textbf{#1}^{#2}}
\newcommand{\catsub}[2]{\normalfont\textbf{#1}_{#2}}

\newcommand{\Hom}{\text{Hom}}
\newcommand{\Homc}[2]{\Hom_{\catname{#1}}(#2)}

\newcommand{\Obj}{\text{Obj}}
\newcommand{\Objc}[1]{\text{Obj}(\catname{#1})}
\newcommand{\Aut}{\text{Aut}}
\newcommand{\End}{\text{End}}

\newcommand{\id}{\text{id}}
\newcommand{\im}{\text{im }}

\newcommand{\lcm}[1]{\text{lcm}(#1)}

\newenvironment{solution}
  {\renewcommand\qedsymbol{$\blacksquare$}\begin{proof}[Solution]}
{\end{proof}}

\newenvironment{sproof}{
  \renewcommand\qedsymbol{$\square$}
  \begin{proof}
  }{
  \end{proof}
}

\newtheorem{lemma}{Lemma}

\theoremstyle{remark}
\newtheorem*{exmp}{Example}
\newtheorem*{cexmp}{Counterexample}


\begin{document}

\title{Algebra: Chapter 0 Exercises\\ \large Chapter 3, Section 2}
\author{David Melendez}
\maketitle

\begin{problem}{1}
Prove that if there is a homomorphism from a zero ring to a ring $R$, then $R$ is a zero ring.
\end{problem}
\begin{solution}
  Let $Z$ denote the zero ring, and let $\varphi : Z\to R$ be a ring homomorphism.
  Since $\varphi$ is a homomorphism, it must take the identity in $Z$ to the identity in $R$,
  so $\varphi(0) = 1_R$.
  But 0 is also the \textit{additive} identity in $Z$, meaning $\varphi(0) = 0$,
  and so $0 = 1$ in $R$.\\
  \indent If $r\in R$, we then have $1\cdot r = 0\cdot r = 0$, showing that $R$ is the zero ring.
\end{solution}

\begin{problem}{2}
  Let $R$ and $S$ be rings, and let $\varphi : R\to S$ be a function preserving both operations
  $+,\cdot$.
  \begin{enumerate}
    \item Prove that if $\varphi$ is surjective, then necessarily $\varphi(1_R) = 1_S$.
    \item Prove that if $\varphi\neq0$ and $S$ is an integral domain, then $\varphi(1_R)=1_S$.
  \end{enumerate}
\end{problem}

\begin{solution}\ 
  \begin{enumerate}
    \item First suppose $\varphi$ is surjective.
    Then, if $s\in S$, then there exists an $r\in R$ such that $\varphi(r)=s$.
    Note that 
    \begin{align*}
      \varphi(1_R)\cdot s &= \varphi(1_R)\cdot\varphi(r) \\
      &= \varphi(1_R\cdot r) \\
      &= \varphi(r) \\
      &= s.
    \end{align*}
    Since this is true for all $s\in S$ (as $\varphi$ is surjective), this implies that
    $\varphi(1_R) = 1_S$, as desired.

    \item Now, let $\varphi\neq 0$ and suppose $\varphi(1_R)\neq1_S$.
    This implies that $\varphi(1_R)-1_S\neq0$. 
    Since $\varphi$ is nonzero, there exists an $r\in R$ with $\varphi(r)\neq 0$.
    Note, then, that we have:
    \begin{align*}
      \varphi(r)\cdot(\varphi(1_R)-1_S) &=  \varphi(r)\cdot\varphi(1_R) - \varphi(r)\cdot1_S\\
      &= \varphi(r\cdot1_R) - \varphi(r)\\
      &= \varphi(r)-\varphi(r)\\
      &= 0,
    \end{align*}
    implying $S$ is not an integral domain since both of the terms in the original product are nonzero.
    Therefore, if $S$ is an integral domain and $\varphi\neq0$, then $\varphi(1_R)=1_S$.
  \end{enumerate}
\end{solution}

\begin{problem}

\end{problem}<++>

\end{document}
