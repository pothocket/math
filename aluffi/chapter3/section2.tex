%&pdflatex 
\documentclass[12pt]{article} 
\usepackage[margin=1in]{geometry} 
\usepackage{amsmath,amsthm,amssymb,amsfonts,tikz-cd, mathrsfs} 

\newenvironment{problem}[2][Problem]{\begin{trivlist}
\item[\hskip \labelsep {\bfseries #1}\hskip \labelsep {\bfseries #2.}]}{\end{trivlist}}

\newenvironment{proposition}[1][Proposition]{\begin{trivlist}
\item[\hskip \labelsep {\bfseries #1.}]}{\end{trivlist}}

\newenvironment{definition}[1][Definition]{\begin{trivlist}
\item[\hskip \labelsep {\bfseries #1.}]}{\end{trivlist}}

\newcommand{\til}{\char`\~}
\newcommand{\bs}{\textbackslash} 

\newcommand{\catname}[1]{\normalfont\textbf{#1}}
\newcommand{\catsup}[2]{\normalfont\textbf{#1}^{#2}}
\newcommand{\catsub}[2]{\normalfont\textbf{#1}_{#2}}

\newcommand{\Hom}{\text{Hom}}
\newcommand{\Homc}[2]{\Hom_{\catname{#1}}(#2)}

\newcommand{\Obj}{\text{Obj}}
\newcommand{\Objc}[1]{\text{Obj}(\catname{#1})}
\newcommand{\Aut}{\text{Aut}}
\newcommand{\End}{\text{End}}

\newcommand{\id}{\text{id}}
\newcommand{\im}{\text{im }}

\newcommand{\lcm}[1]{\text{lcm}(#1)}

\newenvironment{solution}
  {\renewcommand\qedsymbol{$\blacksquare$}\begin{proof}[Solution]}
{\end{proof}}

\newenvironment{sproof}{
  \renewcommand\qedsymbol{$\square$}
  \begin{proof}
  }{
  \end{proof}
}

\newtheorem{lemma}{Lemma}

\theoremstyle{remark}
\newtheorem*{exmp}{Example}
\newtheorem*{cexmp}{Counterexample}


\begin{document}

\title{Algebra: Chapter 0 Exercises\\ \large Chapter 3, Section 2}
\author{David Melendez}
\maketitle

\begin{problem}{2.1}
Prove that if there is a homomorphism from a zero ring to a ring $R$, then $R$ is a zero ring.
\end{problem}
\begin{solution}
  Let $Z$ denote the zero ring, and let $\varphi : Z\to R$ be a ring homomorphism.
  Since $\varphi$ is a homomorphism, it must take the identity in $Z$ to the identity in $R$,
  so $\varphi(0) = 1_R$.
  But 0 is also the \textit{additive} identity in $Z$, meaning $\varphi(0) = 0$,
  and so $0 = 1$ in $R$.\\
  \indent If $r\in R$, we then have $1\cdot r = 0\cdot r = 0$, showing that $R$ is the zero ring.
\end{solution}

\begin{problem}{2.2}
  Let $R$ and $S$ be rings, and let $\varphi : R\to S$ be a function preserving both operations
  $+,\cdot$.
  \begin{enumerate}
    \item Prove that if $\varphi$ is surjective, then necessarily $\varphi(1_R) = 1_S$.
    \item Prove that if $\varphi\neq0$ and $S$ is an integral domain, then $\varphi(1_R)=1_S$.
  \end{enumerate}
\end{problem}

\begin{solution}\ 
  \begin{enumerate}
    \item First suppose $\varphi$ is surjective.
    Then, if $s\in S$, then there exists an $r\in R$ such that $\varphi(r)=s$.
    Note that 
    \begin{align*}
      \varphi(1_R)\cdot s &= \varphi(1_R)\cdot\varphi(r) \\
      &= \varphi(1_R\cdot r) \\
      &= \varphi(r) \\
      &= s.
    \end{align*}
    Since this is true for all $s\in S$ (as $\varphi$ is surjective), this implies that
    $\varphi(1_R) = 1_S$, as desired.

    \item Now, let $\varphi\neq 0$ and suppose $\varphi(1_R)\neq1_S$.
    This implies that $\varphi(1_R)-1_S\neq0$. 
    Since $\varphi$ is nonzero, there exists an $r\in R$ with $\varphi(r)\neq 0$.
    Note, then, that we have:
    \begin{align*}
      \varphi(r)\cdot(\varphi(1_R)-1_S) &=  \varphi(r)\cdot\varphi(1_R) - \varphi(r)\cdot1_S\\
      &= \varphi(r\cdot1_R) - \varphi(r)\\
      &= \varphi(r)-\varphi(r)\\
      &= 0,
    \end{align*}
    implying $S$ is not an integral domain since both of the terms in the original product are nonzero.
    Therefore, if $S$ is an integral domain and $\varphi\neq0$, then $\varphi(1_R)=1_S$.
  \end{enumerate}
\end{solution}

\begin{problem}{2.6}
Verify the 'extension property' of polynomial rings, stated in Example 2.3.
\end{problem}
\begin{solution}\ 
  I will instead do the more general case, stating and proving a universal property for \textit{monoid} rings.
  \begin{proposition}
    Let $R$ be a ring, and $M$ a monoid.
    The monoid ring $R[M]$, as defined in the text, then satisfies the following universal property:\\
    \indent Let $\iota_M : (M,\cdot)\hookrightarrow (R[M],\cdot)$ be the monoid homomorphism
    $m\mapsto 1_Rm$, and let $\iota_R:R\hookrightarrow R[M]$ be the ring homomorphism
    $r\mapsto r1_M$.
    If $S$ is a ring, $\varphi:R\to S$ is a ring homomorphism, and $j:M\to S$ is a monoid homomorphism
    with respect to multiplication on $S$ such that $a\in\im j$ and $b\in\im\varphi$ implies $ab=ba$,
    then there exists a unique ring homomorphism $\overline\varphi:R[M]\to S$ such that the following
    diagram commutes:
    \[
      \begin{tikzcd}
        & R \ar{d}[swap]{\iota_R} \ar{drr}{\varphi} \\
        & R[M] \ar[dashed,near start]{rr}{\overline\varphi} && S \\
        & M \ar{u}{\iota_M} \ar{urr}[swap]{j}
      \end{tikzcd}
    \]
  \end{proposition}
  \begin{sproof}
  First, we show that $\overline\varphi$ is determined by the fact that it must be a ring homomorphism
  that makes the diagram above commute.
  \begin{align*}
    \overline\varphi\left( \sum_{m\in M} r_mm \right) &=\sum_{m\in M}\overline\varphi(r_mm) \\
    &= \sum_{m\in M} \overline\varphi(\iota_R(r_m)\cdot\iota_M(m)) \\
    &= \sum_{m\in M} \overline\varphi(\iota_R(r_m))\cdot\overline\varphi(\iota_M(m)) \\
    &= \sum_{m\in M} \varphi(r_m)\cdot j(m)
  \end{align*}
  \indent Since this function $\overline\varphi$ is the only function making the diagram commute, 
  we just have to prove that it is a ring homomorphism.
  The fact that $\overline\varphi$ preserves addition and the identity is clear enough, so we will
  just show that it preserves multiplication.
  If we let $p=\sum_{m\in M}a_mm$ and $q=\sum_{m\in M}b_mm$, we then have:
    \begin{align*}
      \overline\varphi(p)\cdot \overline\varphi(q) 
      &= \left( \sum_{m\in M} \varphi(a_m)\cdot j(m) \right)
         \left( \sum_{m\in M} \varphi(b_m)\cdot j(m) \right) \\
      &= \sum_{m\in M} \sum_{n\in M} \varphi(a_m)\cdot j(m)\cdot \varphi(b_n)\cdot j(n) \\
      &= \sum_{m\in M} \sum_{n\in M} \varphi(a_m)\cdot \varphi(b_n)\cdot j(m)\cdot j(n) \\
      &= \sum_{m\in M} \sum_{n\in M} \varphi(a_mb_n)\cdot j(mn) \\
      &= \sum_{\ell\in M} \sum_{mn=\ell} \varphi(a_mb_n)\cdot j(\ell) \\
      &= \overline\varphi(pq).
    \end{align*}
    Hence $\overline\varphi$ is a homomorphism, and thus it satisfies the universal property, as
    desired.
  \end{sproof}
  \indent This universal property is a generalization of the universal property for polynomial rings over
  one indeterminant mentioned in the text, which is really just the monoid ring $R[\mathbb{N}]$.
  Additionally, much to our pleasure, polynomial rings in $n$ indeterminants (which commute with each other)
  can be thought of
  as monoid rings $R\left[\mathbb{N}^n\right]$.
\end{solution}

\begin{problem}{2.8}
Prove that every subring of a field is an integral domain.
\end{problem}
\begin{solution}
  Let $k$ be a field and $R$ a subring of $k$.
  If $a\in R$ and $b\in R$, then $ab=0$ implies $a=0$ or $b=0$ since $a,b$ are also in $k$. 
  Hence $R$ is an integral domain.
  Note that $R$ might not be a field, since the multiplicative inverse of an element of $R$ might
  not be in $R$.
\end{solution}

\begin{problem}{2.9}
  The center of a ring $R$, denoted $Z(R)$, consists of the elements $a$ such that $ar=ra$ for all $r\in R$.
\end{problem}
Prove that $Z(R)$ is a subring of $R$.
\begin{proof}
  Suppose $a,b\in Z(R)$, and $r\in R$.
  We then have:
    \begin{align*}
      (a+b)r &= ar+br\\
      &= ra+rb\\
      &= r(a+b),
    \end{align*}
  and
    \begin{align*}
      (ab)r &= a(br)\\
      &= a(rb)\\
      &= (ar)b\\
      &= (ra)b\\
      &= r(ab).
    \end{align*}
  Of course 1 commutes with every element of $R$, so $Z(R)$ is a subring of $R$.
\end{proof}
Prove that the center of a division ring is a field.
\begin{proof}
  Suppose $R$ is a division ring.
  Clearly $Z(R)$ is commutative, so we just need to show that $a\in R$ implies $a^{-1}\in R$.
  This is easy: If $r\in R$, then:
    \begin{align*}
      ar=ra &\implies a^{-1}ar = a^{-1}ra \\
      &\implies r = a^{-1}ra \\
      &\implies ra^{-1} = a^{-1}r
    \end{align*}
  Hence every element $a\in Z(R)$ has a multiplicative inverse in $Z(R)$, making $Z(R)$ a field.
\end{proof}

\begin{problem}{2.10}
  The \textit{centralizer} of an element $a$ of a ring $R$ consists of the elements $r\in R$
  such that $ar=ra$.
\end{problem}
  Prove that the centralizer of $a$, denoted $C_R(a)$ is a subring of $R$, for every $a\in R$.
\begin{proof}
  This follows from an argument identital to the one above for the center of a ring.
\end{proof}
  Prove that the center of $R$ is the intersection of all its centralizers.
\begin{proof}
  Suppose $a\in C_R(r)$ for all $r\in R$.
  Then, by definition, $a$ commutes with every element of $r$, and so $a\in Z(R)$.
  Suppose conversely that $a\in Z(R)$.
  Then $r\in R$ implies $a$ commutes with $r$, again by definition, so $a\in C_R(r)$.
\end{proof}
  Prove that every centralizer in a division ring is a division ring.
\begin{proof}
  By the argument at the top of this page, an element $a$ being in a centralizer implies its
  inverse $a^{-1}$ is also in that centralizer.
\end{proof}

\begin{problem}{2.15}
For $m>1$, the abelian groups $(\mathbb{Z}, +)$ and $(m\mathbb{Z},+)$ are manifestly isomorphic:
the fucntion $\varphi:\mathbb{Z}\to m\mathbb{Z},\ n\mapsto mn$ is a group homomorphism.
\end{problem}
\end{document}
