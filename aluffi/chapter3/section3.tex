%&pdflatex 
\documentclass[12pt]{article} 
\usepackage[margin=1in]{geometry} 
\usepackage{amsmath,amsthm,amssymb,amsfonts,tikz-cd, mathrsfs} 

\newenvironment{problem}[2][Problem]{\begin{trivlist}
\item[\hskip \labelsep {\bfseries #1}\hskip \labelsep {\bfseries #2.}]}{\end{trivlist}}

\newenvironment{proposition}[1][Proposition]{\begin{trivlist}
\item[\hskip \labelsep {\bfseries #1.}]}{\end{trivlist}}

\newenvironment{definition}[1][Definition]{\begin{trivlist}
\item[\hskip \labelsep {\bfseries #1.}]}{\end{trivlist}}

\newcommand{\til}{\char`\~}
\newcommand{\bs}{\textbackslash} 

\newcommand{\catname}[1]{\normalfont\textbf{#1}}
\newcommand{\catsup}[2]{\normalfont\textbf{#1}^{#2}}
\newcommand{\catsub}[2]{\normalfont\textbf{#1}_{#2}}

\newcommand{\Hom}{\text{Hom}}
\newcommand{\Homc}[2]{\Hom_{\catname{#1}}(#2)}

\newcommand{\Obj}{\text{Obj}}
\newcommand{\Objc}[1]{\text{Obj}(\catname{#1})}
\newcommand{\Aut}{\text{Aut}}
\newcommand{\End}{\text{End}}

\newcommand{\id}{\text{id}}
\newcommand{\im}{\text{im }}

\newcommand{\lcm}[1]{\text{lcm}(#1)}

\newenvironment{solution}
  {\renewcommand\qedsymbol{$\blacksquare$}\begin{proof}[Solution]}
{\end{proof}}

\newenvironment{sproof}{
  \renewcommand\qedsymbol{$\square$}
  \begin{proof}
  }{
  \end{proof}
}

\newtheorem{lemma}{Lemma}

\theoremstyle{remark}
\newtheorem*{exmp}{Example}
\newtheorem*{cexmp}{Counterexample}


\begin{document}

\title{Algebra: Chapter 0 Exercises\\ \large Chapter 3, Section 3}
\author{David Melendez}
\maketitle

\begin{problem}{3.1}
  Prove that the image of a ring homomorphism $\varphi:R\to S$ is a subring of $S$.
  What can you say about $\varphi$ if its image is an ideal of $S$?
  What can you say about $\varphi$ if its kernel is a subring of $R$?
\end{problem}
\begin{solution}
  First we'll prove that $\im \varphi$ is a subring of $S$.
  \begin{sproof}
    Suppose $s_1=\varphi(r_1)$ and $s_2=\varphi(r_2)$ are elements of $\im\varphi$.
    We then have \\ $s_1+s_2=\varphi(r_1+r_2)$ and $s_1s_2=\varphi(r_1r_2)$ since $\varphi$
    is a homomorphism, so both of these are elements of $\im\varphi$.
    Additionally, $\varphi(1_R)=1_S$, making $\im\varphi$ a subring of $S$.
  \end{sproof}
  If $\im\varphi$ is an ideal of $S$, then $\varphi$ is surjective, since the only ideal of $S$
  containing the identity $1_S$ is $S$ itself.
  If $\ker\varphi$ is a subring of $R$, then it must contain $1_R$, which, combined with
  the fact that $\ker\varphi$ is an ideal, tells us that $\ker\varphi=R$. Thus $\varphi$
  must be the "zero" morphism $r\mapsto 0$, which isn't actually a ring homomorphism since it
  does not preserve the identity.
\end{solution}

\begin{problem}{3.2}
  Let $\varphi:R\to S$ be a ring homomorphism, and let $J$ be an ideal of $S$.
  Prove that $I=\varphi^{-1}(J)$ is an ideal of $R$.
\end{problem}
\begin{solution}
  Suppose $x\in I$ and $r\in R$.
  We then have $\varphi(rx) = \varphi(r)\varphi(x)$, which is in $J$ since $J$ is an ideal and
  $\varphi(x)\in J$.
  The same argument applies to $xr$ (as $J$ is a two-sided ideal), so $I$ is an ideal of $R$.
\end{solution}

\begin{problem}{3.3}
Let $\varphi:R\to S$ be a ring homomorphism, and let $J$ be an ideal of $R$.
\end{problem}
\begin{enumerate}
  \item Show that $\varphi(J)$ need not be an ideal of S.
    \begin{sproof}
      Let $R=\mathbb{C}$ and $S=\mathbb{H}$ (the quaternions), and let $\varphi$ be the
      inclusion \\$a+bi\mapsto a+bi$.
      The whole of $\mathbb{C}$ is of course an ideal of $\mathbb{C}$, but
      the "copy" of $\mathbb{C}$ in the quaternions $\varphi(\mathbb{C})$ is not an ideal
      of $\mathbb{H}$, since $(a+bi)j = aj+bk\notin\varphi(\mathbb{C})$.
    \end{sproof}
  \item Assume that $\varphi$ is surjective; then prove that $\varphi(J)$ \textit{is} an ideal of $S$.
    \begin{proof}
      
    \end{proof}<++>
  \item Assume that $\varphi$ is surjective, and let $I=\ker\varphi$; thus we may identify
  $S$ with $R/I$.
  Let $\overline J=\varphi(J)$, an ideal of $R/I$ by the previous point.
  Prove that
  \begin{equation*}
    \frac{R/I}{\overline{J}}\cong\frac{R}{I+J}.
  \end{equation*}
\end{enumerate}

\end{document}
