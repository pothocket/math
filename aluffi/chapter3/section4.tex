%&pdflatex 
\documentclass[12pt]{article} 
\usepackage[margin=1in]{geometry} 
\usepackage{amsmath,amsthm,amssymb,amsfonts,tikz-cd, mathrsfs, enumitem} 

\newenvironment{problem}[2][Problem]{\begin{trivlist}
\item[\hskip \labelsep {\bfseries #1}\hskip \labelsep {\bfseries #2.}]}{\end{trivlist}}

\newenvironment{proposition}[1][Proposition]{\begin{trivlist}
\item[\hskip \labelsep {\bfseries #1.}]}{\end{trivlist}}

\newenvironment{definition}[1][Definition]{\begin{trivlist}
\item[\hskip \labelsep {\bfseries #1.}]}{\end{trivlist}}

\newcommand{\til}{\char`\~}
\newcommand{\bs}{\textbackslash} 

\newcommand{\catname}[1]{\normalfont\textbf{#1}}
\newcommand{\catsup}[2]{\normalfont\textbf{#1}^{#2}}
\newcommand{\catsub}[2]{\normalfont\textbf{#1}_{#2}}

\newcommand{\Hom}{\text{Hom}}
\newcommand{\Homc}[2]{\Hom_{\catname{#1}}(#2)}

\newcommand{\Obj}{\text{Obj}}
\newcommand{\Objc}[1]{\text{Obj}(\catname{#1})}
\newcommand{\Aut}{\text{Aut}}
\newcommand{\End}{\text{End}}

\newcommand{\id}{\text{id}}
\newcommand{\im}{\text{im }}

\newcommand{\lcm}[1]{\text{lcm}(#1)}

\newenvironment{solution}
  {\renewcommand\qedsymbol{$\blacksquare$}\begin{proof}[Solution]}
{\end{proof}}

\newenvironment{sproof}{
  \renewcommand\qedsymbol{$\square$}
  \begin{proof}
  }{
  \end{proof}
}

\newtheorem{lemma}{Lemma}

\theoremstyle{remark}
\newtheorem*{exmp}{Example}
\newtheorem*{cexmp}{Counterexample}


\begin{document}

\title{Algebra: Chapter 0 Exercises\\ \large Chapter 3, Section 4\\ 
\large Ideals and quotients: Remarks and examples. Prime and maximal ideals}
\author{David Melendez}
\maketitle

\begin{problem}{4.1}
  Let $R$ be a ring, and let $\left\{ I_\alpha \right\}_{\alpha\in A}$ be a family of ideals of $R$.
  We let
  \begin{equation*}
    \sum_{\alpha\in A} I_\alpha = \left\{ \sum_{\alpha\in A}r_\alpha
    \text{ such that $r_\alpha\in I_\alpha$ and $r_\alpha=0$ for all but finitely many $\alpha$} 
    \right\}
  \end{equation*}
  Prove that $J=\sum_\alpha I_\alpha$ is an ideal of $R$ and that it is the smallest ideal
  containing all of the ideals $I_\alpha$.
\end{problem}
\begin{solution}
  First we prove that $J$ is an ideal of R. 
  \begin{sproof}
    Let $a,b\in J$, so that
    \begin{align*}
      a &= \sum_{\alpha\in A} r_\alpha \\
      b &= \sum_{\alpha\in A} s_\alpha,
    \end{align*}
    where each $r_\alpha,s_\alpha\in I_\alpha$ and all but finitely many $r_\alpha$ and $s_\alpha$
    are nonzero.
    We then have:
    \begin{align*}
      a+b &= \sum_{\alpha\in A} r_\alpha + \sum_{\alpha\in A} s_\alpha \\
      &= \sum_{\alpha\in A} r_\alpha+s_\alpha.
    \end{align*}
    Each term $r_\alpha+s_\alpha$ is in $I_\alpha$ since $r_\alpha,s_\alpha\in I_\alpha$ and
    $I_\alpha$ is an ideal, and clearly all but finitely many $r_\alpha+s_\alpha$ are nonzero
    since $(r_\alpha)_{\alpha\in A}$ and $(s_\alpha)_{\alpha\in A}$ both have that property, so
    $a+b\in J$.\\
    \indent Additionally, if $s\in R$ and $r\in J$ so that $r=\sum_{\alpha\in A}r_\alpha$ 
    (where all but finitely many $r$'s are zero),
    then we have
    \begin{align*}
      rs &= \left( \sum_{\alpha\in A} r_\alpha \right)s\\
      &= \sum_{\alpha\in A} r_\alpha s \\
      &\in J,
    \end{align*}
    where the last line is true because each $r_\alpha s\in I_\alpha$ as a result of each $I_\alpha$
    being a right-ideal of $R$, and the fact that if $r_\alpha$ is zero then $r_\alpha s$ is also zero, 
    implying that there are cofinitely many zero terms in this resulting sum as well.
    A similar argument shows that $J$ is a left-ideal of $R$ if each $I_\alpha$ is also a left-ideal.
  \end{sproof}

  Now, we will show that $J=\sum_{\alpha\in A} I_{\alpha}$ is the smallest ideal of $R$ containing
  each of the ideals $I_\alpha$ for $\alpha\in A$.
  \begin{sproof}
    We just proved that $J$ is an ideal of $R$, so now we just need to show that $J$ is a subset
    of any ideal containing each of the ideals $I_\alpha$.
    This is immediate: if $r\in J$ is such that $r=\sum_{\alpha\in A} r_\alpha$ for 
    $r_\alpha\in I_\alpha$, then of course any ideal of $R$ containing each $I_\alpha$
    contains $r$, since such an ideal is closed under addition.
  \end{sproof}
\end{solution}

\begin{problem}{4.2}
  Prove that the homomorphic image of a Noetherian ring is Noetherian.
  That is, prove that if $\varphi:R\to S$ is a surjective ring homomorphism and $R$ is Noetherian,
  then $S$ is Noetherian.
\end{problem}
\begin{solution}
  Suppose $I=(a_1,\dots,a_n)$ is an ideal of $R$ and $\varphi:R\to S$ is surjective.
  Then we have
  \begin{align*}
    \varphi(I) &= \varphi\left( \sum_{i=1}^{n} (a_i) \right) \\
    &= \sum_{i=1}^{n}\varphi((a_i)) \\
    &= \sum_{i=1}^{n}(\varphi(a_i)),
  \end{align*}
  and so $\varphi(I)$ is finitely generated.\\
  \indent To see that these operations are justified, note that if $g\in R$ and $J=(g)$ is an ideal,
  then we have
  \begin{align*}
    \varphi(J) &= \varphi\left(  \{rg : r\in R\}\right)\\
    &= \{\varphi(r)\varphi(g) : r\in R\} \\
    &= \{r\varphi(g) : r\in R\} \\ 
    &= (\varphi(g)),
  \end{align*}
  where the third equality follows from the surjectivity of $\varphi$.\\
  \indent Additionally, if $I,J$ are ideals of $R$, then we also have
  \begin{align*}
    \varphi(I+J) &= \varphi(\{i+j : i\in I, j\in J\}) \\
    &= \{\varphi(i)+\varphi(j) : i\in I, j\in J\} \\
    &= \varphi(I)+\varphi(J)
  \end{align*}
  \indent Note, then, that if $J$ is an ideal of $S$, then $\varphi^{-1}(J)$ is an ideal of $R$, allowing
  us to see that $J=\varphi(\varphi^{-1}(J))$ is finitely generated.
  Therefore, every ideal of $S$ is finitely generated, and so $S$ is Noetherian.
\end{solution}

\begin{problem}{4.3}
  Prove that the ideal $(2,x)$ of $\mathbb{Z}[x]$ is not principal.
\end{problem}
\begin{solution}
  First, we (quite clumsily) compute the ideal $(2,x)$ as follows:
  \begin{align*}
    (2,x) &= \{2p+xq : p,q\in\mathbb{Z}[x]\} \\
    &= \{(2a_0+2a_1x+\dots+2a_nx^n) + (b_1x+b_2x^2+\dots+b_mx^m):a_j,b_j\in\mathbb{Z}\}\\
    &= \{2a_0+a_1x+a_2x^2+\dots+a_nx^n : a_j\in\mathbb{Z}\}.
  \end{align*}
  In other words, the ideal $(2,x)$ consists of all the polynomials in $\mathbb{Z}[x]$ with an
  even constant term.\\
  \indent Note, then, if $(2,x)=(g)$ for some polynomial $g\in\mathbb{Z}[x]$,
  then there must be a polynomial $p\in\mathbb{Z}[x]$ such that $2=gp$, since $2\in(2,x)$.
  If this is the case, then we have $\deg g + \deg p = 0$, and so $\deg g = \deg p = 0$.
  This means that $g$ is constant, and hence is either $1$ or $2$. 
  In the former case, $(g)$ is the whole ring $\mathbb{Z}[x]$, and in the latter case,
  $(g)$ is the ideal $2\mathbb{Z}[x]$.
  Neither of these ideals equal the ideal of $(2,x)$, leading us to conclude that no single
  polynomial in $\mathbb{Z}[x]$ generates the ideal $(2,x)$.
\end{solution}

\begin{problem}{4.4}
  Prove that if $k$ is a field, then $k[x]$ is a PID. (Hint: Polynomial divison with remainder)
\end{problem}
\begin{solution}
  Let $I\subseteq k[x]$ be an ideal.
  If $I = 0 = (0)$, then clearly it is principal.
  Otherwise, let $p\in I$ be a monic polynomial of minimal degree $d$
    Let $I\subseteq k[x]$ be an ideal.
    If $I = 0 = (0)$, then clearly it is principal. \\
    \indent Otherwise, let $g\in I$ be a monic polynomial of minimal degree $d$.
    If $p\in I$, then we can apply division with remainder to find polynomials $q,r\in k[x]$
    such that $$p=gq+r,$$ where $\deg r < d$.
    Note that since $p\in I$ and $gq\in I$ by (right-) absorption, we then can see that
    $r=p-gq\in I$, since $I$ is closed under addition.
    But $d$ is the smallest degree of any nonzero polynomial in $I$ and $\deg r < d$; it
    then follows that $r = 0$, and so $$p = gq,$$ showing us that $I\subseteq(g)$.
    \\\indent Of course $(g)\subseteq I$, so we then have $I=(g)$, as desired.
\end{solution}

\begin{problem}{4.5}
  Let $I,J$ be ideals in a commutative ring $R$, such that $I+J=(1)$.
  Prove that $IJ=I\cap J$.
\end{problem}
\begin{solution}
  The simple fact that $IH\subseteq I\cap J$ was already proven in the text, so suppose
  $r\in I\cap J$.
  Since $I+J=(1)=R$, we know there exists an $i\in I$ and a $j\in J$ such that $1=i+j$.
  Note, then, that:
  \begin{align*}
    r &= r\cdot1 \\
    &= r\cdot(i+j) \\
    &= r\cdot i + r\cdot j.
  \end{align*}
  Since $r\in J$ and $R$ is commutative, we know that $ri\in IJ$, and since $r\in J$, we also know that
  $rj\in IJ$.
  Hence $r=ri+rj\in IJ$, and so $I\cap J\subseteq IJ$.
  Therefore, $IJ=I\cap J$, as desired.
\end{solution}

\begin{problem}{4.6}
  Let $I,J$ be ideals in a commutative ring $R$.
  Assume that $R/(IJ)$ is reduced (that is, it has no nonzero nilpotent
  elements).
  Prove that $IJ = I\cap J$.
\end{problem}
\begin{solution}
  We will proceed by proving the contrapositive.
  Since we already know that $IJ\subseteq I\cap J$ for any ideals $I,J$,
  assume that $I\cap J\not\subseteq IJ$.
  There then exists an $r\in I\cap J$ with $r\not\in IJ$; that is, 
  such that the coset $r+(IJ)$ is nonzero in $R/(IJ)$.
  Note, then, that $r^2=rr\in IJ$, since $r\in I$ and $r\in J$, and
  so $(r+IJ)^2 = 0$ in the ring $R/(IJ)$.
  Hence $R/(IJ)$ is not reduced, as desired. \\
  \indent Therefore, if $R/(IJ)$ is reduced, then $I\cap J\subseteq IJ$,
  and therefore $I\cap J = IJ$.
\end{solution}

\begin{problem}{4.7}
  Let $R=k$ be a field.
  Prove that every nonzero (principal) ideal in $k[x]$ is generated
  by a unique monic polynomial.
\end{problem}
\begin{solution}
  Let $I$ be an ideal of $k[x]$.
  Then, by exercise 4.4, there is a monic polynomial $p_1\in k[x]$ such
  that $I=(p_1)$.
  Let $p_2\in k[x]$ be such that $(p_2)=(p_1)=I$.
  Then, since $p_1\in (p_2)$ and $p_2\in (p_2)$, there exist $q_1$
  and $q_2$ such that $$p_1=q_1p_2$$ and $$p_2=q_2p_1.$$
  We then have $$p_1=q_1q_2p_1,$$ and hence $q_1q_2=1$, as $k[x]$
  is an integral domain.\\
  \indent Note, then, that 
  \begin{align*}
    0 &= \deg(q_1q_2)\\
    &=^1 \deg(q_1)+\deg(q_2),
  \end{align*}
  where equality $(1)$ follows from $k[x]$ being an integral domain and thus
  having no nonzero zero divisors.
  Therefore, $q_1$ and $q_2$ are both degree 0, that is, constants. \\
  \indent It then follows that if $p_2$ is monic, then $q_1 = 1$ and
  so $p_1=p_2$, showing that $I$ is generated by a unique \textit{monic}
  polynomial as desired.
\end{solution}

\begin{problem}{4.8}
  Let $R$ be a ring and $f(x)\in R[x]$ a monic polynomial.
  Prove that $f(x)$ is not a (left- or right-) zero divisor.
\end{problem}
\begin{solution}
  Let $g(x)\in R[x]$, be a nonzero polynomial and $a$ be the leading 
  coefficient of $g(x)$.
  Since $f(x)$ is monic, the leading coefficient of $f(x)g(x)$ and 
  $g(x)f(x)$ is $1\cdot a=a\cdot1=a$.
  Hence, if either one of these polynomials is zero, then $a=0$.
  Since $a\neq 0$ as $g(x)$ is nonzero, neither of these polynomials is zero.
  Hence $f(x)$ is not a left- or right- zero divisor.
\end{solution}

\newpage

\begin{problem}{4.10}
  Let $d$ be an integer that is not the square of an integer, and consider
  the subset of $\mathbb{C}$ defined by
  \begin{equation*}
    \mathbb{Q}(\sqrt{d}) := \{a+b\sqrt{d}\ |\ a,b\in\mathbb{Q}\}
  \end{equation*}
  \begin{enumerate}[label=(\alph*)]
    \item Prove that $\mathbb{Q}(\sqrt{d})$ is a subring of $\mathbb{C}$.
    \item Define a function $N : \mathbb{Q}(\sqrt{d})\to \mathbb{Q}$
      by $N(a+b\sqrt{d}) = a^2 - b^2d$.
      Prove that $N(zw) = N(z)N(w)$ and that $N(z)\neq0$ if 
      $z\in\mathbb{Q}(\sqrt{d}), z\neq0$.
  \end{enumerate}
  The function $N$ is a 'norm'; it is very useful in the study of 
  $\mathbb{Q}(\sqrt{d})$ and of its subrings.
  \begin{enumerate}[label=(\alph*)]
      \setcounter{enumi}{2}
    \item Prove that $\mathbb{Q}(\sqrt{d})$ is a field and in fact
      the smallest subfield of $\mathbb{C}$ containing both 
      $\mathbb{Q}$ and $\sqrt{d}$. (Use N.)
    \item Prove that $\mathbb{Q}(\sqrt{d})\cong\mathbb{Q}[t]/(t^2-d)$
  \end{enumerate}
\end{problem}
\begin{solution}
  First, we prove that $\mathbb{Q}(\sqrt{d})$ is a subring of $\mathbb{C}$.
  If $a+b\sqrt{d}$ and $x+y\sqrt{d}$ are elements of $\mathbb{Q}(\sqrt{d})$,
  then we have
  \begin{align*}
    (a+b\sqrt{d}) + (x+y\sqrt{d}) &= (a+b) + (x+y)\sqrt{d} \\
    &\in \mathbb{Q}(\sqrt{d})
  \end{align*}
  and
  \begin{align*}
    (a+b\sqrt{d})(x+y\sqrt{d}) &= ax + ay\sqrt{d} + bx\sqrt{d} + bdy \\
    &= (ax+bdy) + (ay+bx)\sqrt{d} \\
    &\in \mathbb{Q}(\sqrt{d}).
  \end{align*}
  Of course $1 = 1 + 0\sqrt{d}\in\mathbb{Q}(\sqrt{d})$, so 
  $\mathbb{Q}(\sqrt{d})$ is a ring. \\
  \indent Now, define $N : \mathbb{Q}(\sqrt{d})\to \mathbb{Q}$ as
  in the question.
  Let $z=z_1+z_2\sqrt{d}$ and $w=w_1+w_2\sqrt{d}$.
  We then have:
  \begin{align*}
    N(zw) &= N( (z_1w_1 + z_2w_2d) + (z_1w_2+z_2w_1)\sqrt{d}) \\
    &= (z_1w_1 + z_2w_2)^2 - (z_1w_2 + z_2w_1)^2d \\
    &= z_1^2w_1^2 - z_1^2w_2^2d - z_2^2w_1^2d + z_2^2w_2^2d^2 \\
    &= (z_1^2 - z_2^2d)(w_1^2-w_2^2d) \\
    &= N(z)N(w).
  \end{align*}
  Of course $z=0$ implies $N(z)=0$, so $N(z)\neq 0$ implies $z\neq0$. \\
  \indent For part (c), let $z = a+b\sqrt{d}$ be a nonzero element of 
  $\mathbb{Q}(\sqrt{d})$.
  Since $N(z)\in\mathbb{Q}$ and is nonzero because $z$ is nonzero, 
  we know that 
  \begin{align*}
    w &=  \frac{a-b\sqrt{d}}{N(z)} \\
    &= \frac{a}{N(z)} - \frac{b}{N(z)}\sqrt{d}\\
    &\in\mathbb{Q}(\sqrt{d}),
  \end{align*}
  and so we find that
  \begin{align*}
    zw &= \left(a+b\sqrt{d}\right)\left(\frac{a-b\sqrt{d}}{N(z)}\right) \\
    &= \frac{N(z)}{N(z)} \\
    &= 1,
  \end{align*}
  showing that $w$ is a multiplicative inverse of $z$, and hence
  that $\mathbb{Q}(\sqrt{d})$ is a field.\\
  \indent Any subfield of $\mathbb{C}$ that conatins $\mathbb{Z}$ and 
  $\sqrt{d}$ must contain $\mathbb{Q}(\sqrt{d})$, since subfields
  must be closed under the field operations.
  Hence $\mathbb{Q}(\sqrt{d})$ is the smallest such subfield. \\
  \indent Finally, Let 
  $\varphi : \mathbb{Q}[x]\to \mathbb{Q}\oplus\mathbb{Q}$
  be the group homomorphism (defined in the book) that sends
  $g(x)\in\mathbb{Q}[x]$ to the pair $(r_0,r_1)$, where $r_0,r_1$
  come from the remainder $r_0+r_1x$ when dividing $g(x)$
  by the polynomial $x^2-d$. \\
  \indent An argument in the book establishes that $\varphi$ is surjective
  with kernel equal to the principle ideal $(x^2-d)$, and so we have
  \begin{equation*}
    \frac{\mathbb{Q}[x]}{(x^2-d)}\cong \mathbb{Q}\oplus\mathbb{Q},
  \end{equation*}
  as abelian groups.\\
  \indent All we need to do, now, is endow $\mathbb{Q}\oplus\mathbb{Q}$
  with a ring structure that makes $\varphi$ a homomorphism, and show
  that this ring is isomorphic to $\mathbb{Q}(\sqrt{d})$.
  Define an operation $\cdot$ on $\mathbb{Q}\oplus\mathbb{Q}$, then,
  by 
  \begin{equation*}
    (a_0,a_1)\cdot(b_0,b_1) = 
    \varphi\left(\varphi^{-1}\left(a_0,a_1\right)
    \cdot\varphi^{-1}\left(b_0,b_1\right)\right),
  \end{equation*}
  where $\overline{p(x)}$ is the coset of $p(x)$ in the quotient ring
  $\mathbb{Q}[x]/(x^2-d)$. 
  Note that this is well-defined because $\varphi$ is a bijection.\\
  \indent To prove that $\varphi$ is then a ring homomorphism
  $\displaystyle\frac{\mathbb{Q}[x]}{(x^2-d)}\to \mathbb{Q}\oplus\mathbb{Q}$,
  note that:
  \begin{align*}
    \varphi^{-1}(a\cdot b) 
    &= \varphi^{-1}
      \left( \varphi(\varphi^{-1}(a)\cdot\varphi^{-1}(b) )\right) \\
      &= \varphi^{-1}(a)\cdot\varphi^{-1}(b),
  \end{align*}
  from which it then follows that $\varphi$ preserves multiplication. \\
  \indent The reader way wish to note, additionally, that the
  identity with respect to this multiplication defined on
  $\mathbb{Q}\oplus\mathbb{Q}$, is the pair $(0,1)$, as we have,
  for $a_0,a_1\in\mathbb{Q}$:
  \begin{align*}
    (1,0)\cdot(a_0,a_1) 
    &= \varphi\left(\varphi^{-1}(0,1)\cdot\varphi^{-1}(a_0,a_1)\right) \\
    &= \varphi( (\overline{1})\cdot(\overline{a_0+a_1x}) ) \\
    &= \varphi(\overline{1(a_0+a_1x)}) \\
    &= \varphi(a_0+a_1x) \\
    &= (a_0,a_1),
  \end{align*}
  where each step follows from simple considerations concerning polynomial
  division by a polynomial of degree 2, and the properties of ideals
  with respect to the ring operations.\\
  \indent Since the remainder when dividing $1$ by the polynomial
  $x^2-d$ is $1$, we then know that $\varphi(1) = (1,0)$, showing that
  $\varphi$ preserves the identity with respect to multiplication.\\
  \indent Since $\varphi$ is a bijection that preserves addition,
  multiplication, and the identity as defined on $\mathbb{Q}\oplus\mathbb{Q}$,
  we then know that $\mathbb{Q}\oplus\mathbb{Q}$ is a ring isomorphic
  to the ring $\mathbb{Q}[x]/(x^2-d)$.\\
  \indent To characterize this multiplication on $\mathbb{Q}\oplus\mathbb{Q}$,
  we find with some algebra that
  \begin{align*}
    (a_0+a_1)\cdot(b_0+b_1)
    &= \varphi\left( 
         \varphi^{-1}(a_0,a_1)\cdot\varphi^{-1}(b_0,b_1) \right) \\
    &= \varphi\left( (\overline{a_0+a_1x})
         \cdot(\overline{b_0+b_1x}) \right)\\
    &= \varphi\left( \overline{(a_0+a_1x)(b_0+b_1x}) \right) \\
    &= \varphi(\overline{a_0b_0+(a_0b_1+a_1b+0)x+a_1b_1x^2})\\
    &= \varphi(
    \overline{(a_0b_0+a_1b_1d) + (a_0b_1+a_1b_0)x + (x^2-d)(a_1b_1)})\\
    &= \varphi(\overline{(a_0b_0+a_1b_1d) + (a_0b_1+a_1b_0)x})\\
    &= (a_0b_0+a_1b_1d, a_0b_1+a_1b_0).
  \end{align*}
  \indent Inspection of the product $(a_0+a_1\sqrt{d})(b_0+b_1\sqrt{d})$
  quite easily shows that the mapping $(a_0,a_1)\mapsto a_0+a_1\sqrt{d}$
  is a ring isomorphism $\mathbb{Q}\oplus\mathbb{Q}\to \mathbb{Q}(\sqrt{d})$,
  and so we therefore get the isomorphism 
  $\mathbb{Q}[x]/(x^2-d)\cong\mathbb{Q}(\sqrt{d})$, as desired.
\end{solution}

\begin{problem}{4.11}
  Let $R$ be a commutative ring, $a\in R$, and $f_1(x),\dots,f_r(x)\in R[X]$.
  \begin{enumerate}[label=(\alph*)]
    \item Prove the equality of ideals 
      $$\left( f_1(x),\dots,f_r(x),x-a \right) = 
      \left( f_1(a),\dots,f_r(a),x-a \right).$$
    \item Prove the useful substitution trick
      \begin{equation*}
        \frac{R[x]}{\left( f_1(x),\dots,f_r(x),x-a \right)}
        \cong \frac{R}{\left( f_1(a),\dots,f_r(a)\right)}.
      \end{equation*}
  \end{enumerate}
\end{problem}
\begin{solution}
  First, suppose $f(x)\in R[x]$ and $a\in R$.
  If we divide $f(x) - f(a)$ by the polynomial $x-a$, we find that
  \begin{equation*}
    f(x) - f(a) = q(x)(x-a) + r(x),
  \end{equation*}
  where $q(x),r(x)\in R[x]$ and $\deg r(x) = 0$.
  Evaluating both sides at $a$, we then can see that 
  \begin{align*}
    f(a)-f(a) &= q(a)(a-a) + r(a)\\
    \Rightarrow  0 &= r(a),
  \end{align*}
  meaning $r(x)$ must be the zero polynomial since it has degree zero
  and evaluates to zero at $a$.
  Thus, we have:
  \begin{equation*}
    f(x) = q(x)(x-a) + f(a)
  \end{equation*}
  for all $f(x)\in R[x], a\in R$ and some polynomial $q(x)$.\\
  \indent Applying this to our problem, we then have polynomials
  $q_j$ such that
  \begin{equation*}
    f_j(x) = q_j(x)(x-a) + f_j(a),
  \end{equation*}
  for all $1\leq j\leq r$.\\
  \indent We then have, for all polynomials of the form shown on the 
  left-hand side in the ideal
  $\left( f_1(x),\dots,f_r(x),x-a \right)$,
  \begin{align*}
    \left( \sum_{j=0}^r p_j(x)f_j(x) \right) + g(x)(x-a) 
    &=  \left( \sum_{j=0}^r p_j(x)(q_j(x)(x-a)) + f_j(a) \right) + g(x)(x-a)\\
    &= \sum_{j=1}^r f_j(a) 
    + \left( g(x) + \sum_{j=1}^n q_j(a) \right)(x-a),
  \end{align*}
  and so $(f_1(x),\dots,f_r(x),x-a)\subseteq (f_1(a),\dots,f_r(a),x-a)$.
  A nearly identical argument gives us the inclusion in the other direction,
  and so we have the desired equality of ideals. \\
  \indent With this shiny new lemma, we can now see that\\
  \begin{align*}
    \frac{R[x]}{(f_1(x),\dots,f_r(x),x-a)} 
    &= \frac{R[x]}{(f_1(a),\dots,f_r(a),x-a)} \\
    &= \frac{R[x]}{(f_1(a),\dots,f_r(a))+(x-a)} \\
    &\cong \frac{R[x]/(x-a)}{(f_1(a),\dots,f_r(a))} \\
    &\cong \frac{R}{(f_1(a),\dots,f_r(a))} ,
  \end{align*}
  where the first equality follows from the lemma we just proved,
  the second equality follows from the definition of an ideal generated
  by multiple elements, the isomorphism in the third line follows
  from Exercise 3.3, and the isomorphism in the fourth line
  follows from the isomorphism 
  \begin{equation*}
    \frac{R[x]}{(x-a)}\cong R.
  \end{equation*}
\end{solution}
\begin{problem}{4.12}
  Let $R$ be a commutative ring and $a_1,\dots,a_n$ elements of $R$.
  Prove that
  \begin{equation*}
    \frac{R[x_1,\dots,x_n]}{(x_1-a_1,\dots,x_n-a_n)}\cong R
  \end{equation*}
\end{problem}
\begin{solution}
  We will proceed by induction.
  For the base case $n=1$, the isomorphism
  \begin{equation*}
    \frac{R[x]}{(x-a)}
  \end{equation*}
  has already been established.
  For the induction step, suppose we have the isomorphism
  \begin{equation*}
    \frac{R[x_1,\dots,x_n]}{(x_1-a_1,\dots,x_n-a_n)}
  \end{equation*}
  for $a_1,\dots,a_n\in R$, and let $a_{n+1}\in R$.
  We then have:
  \begin{align*}
    \frac{R[x_1,\dots,x_n,x_{n+1}]}{(x_1-a_1,\dots,x_n-a_n,x_{n+1}-a_{n+1})}
      &= \frac{R[x_1,\dots,x_n][x_{n+1}]}
      {(x_1-a_1,\dots,x_n-a_n)+(x_{n+1}-a_{n+1})}\\
      &\cong\frac{R[x_1,\dots,x_n][x_{n+1}]/(x_{n+1}-a_{n+1})}
      {(x_1-a_1,\dots,x_n-a_n)}\\
      &\cong \frac{R[x_1,\dots,x_n]}{(x_1-a_1,\dots,x_n-a_n)}\\
      &\cong R.
  \end{align*}
  \indent Here,  the equality in the first line follows from the definition
  of multivariate polynomial rings and finitely generated ideals,
  the isomorphism in the second line follows from Exercise 3.3,
  the isomorphism in the third line follows from the general isomorphism
  $R[x]/(x-a)\cong R$, and the isomorphism in the fourth line
  follows from the inductive hypothesis.
\end{solution}

\begin{problem}{4.13}
  Let $R$ be an integral domain.
  For all $k=1,\dots,n$, prove that $(x_1,\dots,x_k)$ is
  prime in $R[x_1,\dots,x_n]$.
\end{problem}
\begin{solution}
  This follows immediately from the previous exercise, and the fact that 
  the ideal $I\subseteq R$ is prime iff $R/I$ is an integral domain.
\end{solution}

\begin{problem}{4.14}
  Prove 'by hand' that maximal ideals are prime, \textit{without} using 
  quotient rings.
\end{problem}
\begin{solution}
  We will assume that the ring $R$ is commutative, since this is the only case in which the book
  deals with prime and maximal ideals (and because it makes things more convenient).
  We will also assume that $R$ is nonzero, as the result is immediate otherwise. \\
  \indent Let $I$ be a maximal ideal, and let $J$ be the subset of $R$ defined by\\
  \begin{equation*}
    J = \{a\in R\ |\ (\exists b\in R\setminus I): ab\in I \}
  \end{equation*}
  \indent Since $I$ is maximal, it does not contain the identity (by definition), and so
  $a\in I$ implies $a\cdot1_R\in I$, making it clear that $I\subseteq J$. \\
  \indent Note, then, that if $a\in J$ so that $b\in R\setminus I$ is such that $ab\in I$,
  then for any $r\in R$, we have:
  \begin{align*}
    (ra)b &= r(ab) \\
    &\in I,
  \end{align*}
  and so $J$ is actually an ideal of $R$.
  Because $1\notin J$ (as $1r\in I$ implies $r\in I$), 
  this means that $J=I$, since $J$ contains $I$ 
  and $I$ is maximal.\\
  \indent For the slam dunk, observe that if $a,b\in R$ are such that $ab\in I$ and $b\notin I$,
  then $a\in J$ by definition, and so $a\in I$, proving that $I$ is a prime ideal.
\end{solution}

\begin{problem}{4.15}
  Let $\varphi:R\to S$ be a homomorphism of commutative rings, and let $I\subseteq S$
  be an ideal.
  Prove that if $I$ is a prime ideal in $S$, then $\varphi^{-1}(I)$ is a prime ideal in $R$.
  Show that $\varphi^{-1}(I)$ is not necessarily maximal if $I$ is maximal.
\end{problem}
\begin{solution}
  Let $J\subseteq R$ be the preimage of $I$, $J=\varphi^{-1}(I)$.
  Note that $r\in R, a\in J$ implies $\varphi(ra)=\varphi(r)\varphi(a)\in I$,
  and so $J$ is an ideal of $R$. 
  Then, if $a,b\in R$ are such that $ab\in J$, then
  $\varphi(ab)=\varphi(a)\varphi(b)\in I$, which implies $\varphi(a)\in I$
  or $\varphi(b)\in I$ (since I is prime), which implies $a\in J$ or $b\in J$; 
  hence $J$ is prime.
\end{solution}

\begin{problem}{4.16}
  Let $R$ be a commutative ring, and let $P$ be a prime ideal of $R$.
  Suppose $0$ is the only zero-divisor of $R$ contained in $P$.
  Prove that $R$ is an integral domain.
\end{problem}
\begin{solution}
  Suppose $R$ is not an integral domain.
  Then there exist $a,b\in R$, both nonzero, such that $ab=0$.
  If $P$ is a prime ideal in $R$, then $ab\in P$ since $0$ is contained
  in every ideal of $R$, and so we know $a\in P$ or $b\in P$. 
  Hence, $0$ is not the only zero-divisor of $R$ contained in $P$,
  proving the contrapositive of the desired result.
\end{solution}

\begin{problem}{4.18}
  Let $R$ be a commutative ring, and let $N$ be its nilradical.
  Prove that $N$ is contained in every prime ideal of $R$.
\end{problem}
\begin{solution}
  Let $a\in N$, so that $a^n = 0$ for some integer $n$.
  If $P$ is a prime ideal of $R$, then we have
  \begin{align*}
    0 &= (a^n + P) \\
    &= (a+P)^n
  \end{align*}
  in the ring $R/P$, and so $(a+P) = 0$ as $R/P$ is an integral domain,
  meaning $a\in P$.
\end{solution}

\begin{problem}{4.19}
  Let $R$ be a commutative ring, let $P$ be a prime ideal in $R$, and let
  $I_j$ be ideals of $R$.
  \begin{enumerate}[label=(\alph*)]
    \item Assume that $I_1\cdots I_r\subseteq P$; prove that $I_J\subseteq P$
      for some $j$.
    \item By (a), if $P\supseteq \bigcap_{j=1}^{r} I_j$, then
      $P$ contains one of the ideals $I_j$.
      Prove or disprove: if $P\supseteq \bigcap_{j=1}^{\infty} I_j$,
      then $P$ contains one of the ideals $I_j$.
  \end{enumerate}
\end{problem}
\begin{solution}
  
\end{solution}<++>

\begin{problem}{4.22}
  Prove that $(x^2+1)$ is maximal in $\mathbb{R}[x]$.
\end{problem}
\begin{solution}
  The ring $\mathbb{R}[x]]/(x^2+1)$ is isomorphic to $\mathbb{C}$,
  which is a field, so $(x^2+1)$ is maximal.
\end{solution}

\begin{problem}{4.24}
  Prove that the ring $\mathbb{Z}[x]$ has Krull dimension $\geq2$.
\end{problem}
\begin{solution}
  We consider the ideals $(x)$ and $(x,p)$ (for $p\in\mathbb{Z}$ prime)
  of $\mathbb{Z}[x]$.
  Note that:
  \begin{equation*}
    \frac{\mathbb{Z}[x]}{x}\cong\mathbb{Z},
  \end{equation*}
  and
  \begin{align*}
    \frac{\mathbb{Z}[x]}{(x,p)} &\cong \frac{\mathbb{Z}[x]/(x)}{(p)} \\
    &\cong\frac{\mathbb{Z}}{(p)} \\
    &= \mathbb{Z}/p\mathbb{Z},
  \end{align*}
  and so $(x)$ is prime and $(x,p)$ is maximal (and thus prime). \\
  \indent The chain of proper inclusions of prime ideals
  \begin{equation*}
    0\subsetneq (x)\subsetneq (x,p)
  \end{equation*}
  then shows that $\mathbb{Z}[x]$ has Krull dimension at least 2.
\end{solution}
\end{document}
