%&pdflatex 
\documentclass[12pt]{article} 
\usepackage[margin=1in]{geometry} 
\usepackage{amsmath,amsthm,amssymb,amsfonts,tikz-cd, mathrsfs} 

\newenvironment{problem}[2][Problem]{\begin{trivlist}
\item[\hskip \labelsep {\bfseries #1}\hskip \labelsep {\bfseries #2.}]}{\end{trivlist}}

\newenvironment{proposition}[1][Proposition]{\begin{trivlist}
\item[\hskip \labelsep {\bfseries #1.}]}{\end{trivlist}}

\newenvironment{definition}[1][Definition]{\begin{trivlist}
\item[\hskip \labelsep {\bfseries #1.}]}{\end{trivlist}}

\newcommand{\til}{\char`\~}
\newcommand{\bs}{\textbackslash} 

\newcommand{\catname}[1]{\normalfont\textbf{#1}}
\newcommand{\catsup}[2]{\normalfont\textbf{#1}^{#2}}
\newcommand{\catsub}[2]{\normalfont\textbf{#1}_{#2}}

\newcommand{\Hom}{\text{Hom}}
\newcommand{\Homc}[2]{\Hom_{\catname{#1}}(#2)}

\newcommand{\Obj}{\text{Obj}}
\newcommand{\Objc}[1]{\text{Obj}(\catname{#1})}
\newcommand{\Aut}{\text{Aut}}
\newcommand{\End}{\text{End}}

\newcommand{\id}{\text{id}}
\newcommand{\im}{\text{im }}

\newcommand{\lcm}[1]{\text{lcm}(#1)}

\newenvironment{solution}
  {\renewcommand\qedsymbol{$\blacksquare$}\begin{proof}[Solution]}
{\end{proof}}

\newenvironment{sproof}{
  \renewcommand\qedsymbol{$\square$}
  \begin{proof}
  }{
  \end{proof}
}

\newtheorem{lemma}{Lemma}

\theoremstyle{remark}
\newtheorem*{exmp}{Example}
\newtheorem*{cexmp}{Counterexample}


\begin{document}

\title{Algebra: Chapter 0 Exercises\\ \large Chapter 3, Section 4\\ 
\large Ideals and quotients: Remarks and examples. Prime and maximal ideals}
\author{David Melendez}
\maketitle

\begin{problem}{4.1}
  Let $R$ be a ring, and let $\left\{ I_\alpha \right\}_{\alpha\in A}$ be a family of ideals of $R$.
  We let
  \begin{equation*}
    \sum_{\alpha\in A} I_\alpha = \left\{ \sum_{\alpha\in A}r_\alpha
    \text{ such that $r_\alpha\in I_\alpha$ and $r_\alpha=0$ for all but finitely many $\alpha$} 
    \right\}
  \end{equation*}
  Prove that $J=\sum_\alpha I_\alpha$ is an ideal of $R$ and that it is the smallest ideal
  containing all of the ideals $I_\alpha$.
\end{problem}
\begin{solution}
  First we prove that $J$ is an ideal of R. 
  \begin{sproof}
    Let $a,b\in J$, so that
    \begin{align*}
      a &= \sum_{\alpha\in A} r_\alpha \\
      b &= \sum_{\alpha\in A} s_\alpha,
    \end{align*}
    where each $r_\alpha,s_\alpha\in I_\alpha$ and all but finitely many $r_\alpha$ and $s_\alpha$
    are nonzero.
    We then have:
    \begin{align*}
      a+b &= \sum_{\alpha\in A} r_\alpha + \sum_{\alpha\in A} s_\alpha \\
      &= \sum_{\alpha\in A} r_\alpha+s_\alpha.
    \end{align*}
    Each term $r_\alpha+s_\alpha$ is in $I_\alpha$ since $r_\alpha,s_\alpha\in I_\alpha$ and
    $I_\alpha$ is an ideal, and clearly all but finitely many $r_\alpha+s_\alpha$ are nonzero
    since $(r_\alpha)_{\alpha\in A}$ and $(s_\alpha)_{\alpha\in A}$ both have that property, so
    $a+b\in J$.\\
    \indent Additionally, if $s\in R$ and $r\in J$ so that $r=\sum_{\alpha\in A}r_\alpha$ 
    (where all but finitely many $r$'s are zero),
    then we have
    \begin{align*}
      rs &= \left( \sum_{\alpha\in A} r_\alpha \right)s\\
      &= \sum_{\alpha\in A} r_\alpha s \\
      &\in J,
    \end{align*}
    where the last line is true because each $r_\alpha s\in I_\alpha$ as a result of each $I_\alpha$
    being a right-ideal of $R$, and the fact that if $r_\alpha$ is zero then $r_\alpha s$ is also zero, 
    implying that there are cofinitely many zero terms in this resulting sum as well.
    A similar argument shows that $J$ is a left-ideal of $R$ if each $I_\alpha$ is also a left-ideal.
  \end{sproof}

  Now, we will show that $J=\sum_{\alpha\in A} I_{\alpha}$ is the smallest ideal of $R$ containing
  each of the ideals $I_\alpha$ for $\alpha\in A$.
  \begin{sproof}
    We just proved that $J$ is an ideal of $R$, so now we just need to show that $J$ is a subset
    of any ideal containing each of the ideals $I_\alpha$.
    This is immediate: if $r\in J$ is such that $r=\sum_{\alpha\in A} r_\alpha$ for 
    $r_\alpha\in I_\alpha$, then of course any ideal of $R$ containing each $I_\alpha$
    contains $r$, since such an ideal is closed under addition.
  \end{sproof}
\end{solution}

\begin{problem}{4.2}
  Prove that the homomorphic image of a Noetherian ring is Noetherian.
  That is, prove that if $\varphi:R\to S$ is a surjective ring homomorphism and $R$ is Noetherian,
  then $S$ is Noetherian.
\end{problem}
\begin{solution}
  Suppose $I=(a_1,\dots,a_n)$ is an ideal of $R$ and $\varphi:R\to S$ is surjective.
  Then we have
  \begin{align*}
    \varphi(I) &= \varphi\left( \sum_{i=1}^{n} (a_i) \right) \\
    &= \sum_{i=1}^{n}\varphi((a_i)) \\
    &= \sum_{i=1}^{n}(\varphi(a_i)),
  \end{align*}
  and so $\varphi(I)$ is finitely generated.\\
  \indent To see that these operations are justified, note that if $g\in R$ and $J=(g)$ is an ideal,
  then we have
  \begin{align*}
    \varphi(J) &= \varphi\left(  \{rg : r\in R\}\right)\\
    &= \{\varphi(r)\varphi(g) : r\in R\} \\
    &= \{r\varphi(g) : r\in R\} \\ 
    &= (\varphi(g)),
  \end{align*}
  where the third equality follows from the surjectivity of $\varphi$.\\
  \indent Additionally, if $I,J$ are ideals of $R$, then we also have
  \begin{align*}
    \varphi(I+J) &= \varphi(\{i+j : i\in I, j\in J\}) \\
    &= \{\varphi(i)+\varphi(j) : i\in I, j\in J\} \\
    &= \varphi(I)+\varphi(J)
  \end{align*}
  \indent Note, then, that if $J$ is an ideal of $S$, then $\varphi^{-1}(J)$ is an ideal of $R$, allowing
  us to see that $J=\varphi(\varphi^{-1}(J))$ is finitely generated.
  Therefore, every ideal of $S$ is finitely generated, and so $S$ is Noetherian.
\end{solution}

\begin{problem}{4.3}
  Prove that the ideal $(2,x)$ of $\mathbb{Z}[x]$ is not principal.
\end{problem}
\begin{solution}
  First, we (quite clumsily) compute the ideal $(2,x)$ as follows:
  \begin{align*}
    (2,x) &= \{2p+xq : p,q\in\mathbb{Z}[x]\} \\
    &= \{(2a_0+2a_1x+\dots+2a_nx^n) + (b_1x+b_2x^2+\dots+b_mx^m):a_j,b_j\in\mathbb{Z}\}\\
    &= \{2a_0+a_1x+a_2x^2+\dots+a_nx^n : a_j\in\mathbb{Z}\}.
  \end{align*}
  In other words, the ideal $(2,x)$ consists of all the polynomials in $\mathbb{Z}[x]$ with an
  even constant term.\\
  \indent Note, then, if $(2,x)=(g)$ for some polynomial $g\in\mathbb{Z}[x]$,
  then there must be a polynomial $p\in\mathbb{Z}[x]$ such that $2=gp$, since $2\in(2,x)$.
  If this is the case, then we have $\deg g + \deg p = 0$, and so $\deg g = \deg p = 0$.
  This means that $g$ is constant, and hence is either $1$ or $2$. 
  In the former case, $(g)$ is the whole ring $\mathbb{Z}[x]$, and in the latter case,
  $(g)$ is the ideal $2\mathbb{Z}[x]$.
  Neither of these ideals equal the ideal of $(2,x)$, leading us to conclude that no single
  polynomial in $\mathbb{Z}[x]$ generates the ideal $(2,x)$.
\end{solution}

\begin{problem}{4.4}
  Prove that if $k$ is a field, then $k[x]$ is a PID. (Hint: Polynomial divison with remainder)
\end{problem}
\begin{solution}
  Let $I\subseteq k[x]$ be an ideal.
  If $I = 0 = (0)$, then clearly it is principal.
  Otherwise, let $p\in I$ be a monic polynomial of minimal degree $d$
    Let $I\subseteq k[x]$ be an ideal.
    If $I = 0 = (0)$, then clearly it is principal. \\
    \indent Otherwise, let $g\in I$ be a monic polynomial of minimal degree $d$.
    If $p\in I$, then we can apply division with remainder to find polynomials $q,r\in k[x]$
    such that $$p=gq+r,$$ where $\deg r < d$.
    Note that since $p\in I$ and $gq\in I$ by (right-) absorption, we then can see that
    $r=p-gq\in I$, since $I$ is closed under addition.
    But $d$ is the smallest degree of any nonzero polynomial in $I$ and $\deg r < d$; it
    then follows that $r = 0$, and so $$p = gq,$$ showing us that $I\subseteq(g)$.
    \\\indent Of course $(g)\subseteq I$, so we then have $I=(g)$, as desired.
\end{solution}

\begin{problem}{4.5}
  Let $I,J$ be ideals in a commutative ring $R$, such that $I+J=(1)$.
  Prove that $IJ=I\cap J$.
\end{problem}
\begin{solution}
  The simple fact that $IH\subseteq I\cap J$ was already proven in the text, so suppose
  $r\in I\cap J$.
  Since $I+J=(1)=R$, we know there exists an $i\in I$ and a $j\in J$ such that $1=i+j$.
  Note, then, that:
  \begin{align*}
    r &= r\cdot1 \\
    &= r\cdot(i+j) \\
    &= r\cdot i + r\cdot j.
  \end{align*}
  Since $r\in J$ and $R$ is commutative, we know that $ri\in IJ$, and since $r\in J$, we also know that
  $rj\in IJ$.
  Hence $r=ri+rj\in IJ$, and so $I\cap J\subseteq IJ$.
  Therefore, $IJ=I\cap J$, as desired.
\end{solution}<++>

\end{document}
