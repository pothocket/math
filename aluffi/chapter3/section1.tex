%&pdflatex 
\documentclass[12pt]{article} 
\usepackage[margin=1in]{geometry} 
\usepackage{amsmath,amsthm,amssymb,amsfonts,tikz-cd, mathrsfs} 

\newenvironment{problem}[2][Problem]{\begin{trivlist}
\item[\hskip \labelsep {\bfseries #1}\hskip \labelsep {\bfseries #2.}]}{\end{trivlist}}

\newenvironment{proposition}[1][Proposition]{\begin{trivlist}
\item[\hskip \labelsep {\bfseries #1.}]}{\end{trivlist}}

\newenvironment{definition}[1][Definition]{\begin{trivlist}
\item[\hskip \labelsep {\bfseries #1.}]}{\end{trivlist}}

\newcommand{\til}{\char`\~}
\newcommand{\bs}{\textbackslash} 

\newcommand{\catname}[1]{\normalfont\textbf{#1}}
\newcommand{\catsup}[2]{\normalfont\textbf{#1}^{#2}}
\newcommand{\catsub}[2]{\normalfont\textbf{#1}_{#2}}

\newcommand{\Hom}{\text{Hom}}
\newcommand{\Homc}[2]{\Hom_{\catname{#1}}(#2)}

\newcommand{\Obj}{\text{Obj}}
\newcommand{\Objc}[1]{\text{Obj}(\catname{#1})}
\newcommand{\Aut}{\text{Aut}}
\newcommand{\End}{\text{End}}

\newcommand{\id}{\text{id}}
\newcommand{\im}{\text{im }}

\newcommand{\lcm}[1]{\text{lcm}(#1)}

\newenvironment{solution}
  {\renewcommand\qedsymbol{$\blacksquare$}\begin{proof}[Solution]}
{\end{proof}}

\newenvironment{sproof}{
  \renewcommand\qedsymbol{$\square$}
  \begin{proof}
  }{
  \end{proof}
}

\newtheorem{lemma}{Lemma}

\theoremstyle{remark}
\newtheorem*{exmp}{Example}
\newtheorem*{cexmp}{Counterexample}


\begin{document}

\title{Algebra: Chapter 0 Exercises\\ \large Chapter 3, Section 1}
\author{David Melendez}
\maketitle

\begin{problem}{1}
Prove that is $0=1$ in a ring $R$, then $R$ is a zero-ring.
\end{problem}
\begin{proof}
If $x\in R$, then $x=1x=0x=0$.
\end{proof}

\begin{problem}{2}
  Let $S$ be a set, and define operations on the power set $\mathscr{P}(S)$ of $S$ by setting
  $\forall A,B\in\mathscr{P}(S)$:
  \begin{align*}
    A+B &:= (A\cup B) \setminus (A\cap B) \\
    A\cdot B &:= A\cap B
  \end{align*}
  Prove that $(\mathscr{P}(S), +, \cdot)$ is a commutative ring.
\end{problem}
\begin{solution}
  We will establish a bijection between $\mathscr{P}(S)$ and another ring that preserves the given
  operations.
  Let $(\mathbb{Z}/2\mathbb{Z})^S$ be a ring as defined in problem 1.3, and define a function
  $\varphi : \mathscr{P}(S)\to (\mathbb{Z}/2{Z})^S$ by
  \begin{equation*}
    \varphi(A)(x) = 
      \begin{cases}
        1 & x\in A\\
        0 & x\notin A
      \end{cases}
  \end{equation*}
  Case work shows that $\varphi$ preserves $+$ and $\cdot$, and further investigation 
  reveals a natural inverse for $\varphi$, showing that it is a bijection, and hence
  that $\mathscr{P}(S)$ is a ring (isomorphic to $(\mathbb{Z}/2\mathbb{Z})^S$, in fact).
\end{solution}

\begin{problem}{1.3}
  Let $R$ be a ring, and let $S$ be any set.
  Explain how to endow the set $R^S$ of set functions $S\to R$ of two operations $+,\cdot$
  so as to make $R^S$ into a ring, such that $R^S$ is just a copy of $R$ if $S$ is a singleton.
\end{problem}
\begin{solution}
  Define $+$ on $R^S$ by $(f_1+f_2)(s) = f_1(s) + f_2(s)$ (making $R^S$ into an abelian group, as $R$
  is an abelian group under $+$), and define $\cdot$ by $(f_1\cdot f_2)(s)=f_1(s)\cdot f_2(s)$.
  Each of the ring axioms are clearly satisfies by this.
  If $S$ is a singleton, then each $f:S\to R$ is uniquely identified by an element of $R$, and the
  operations coincide in the obvious way.
\end{solution}

\begin{problem}{1.5}
  Let $R$ be a ring.
  If $a,b$ are zero-divisors in $R$, is $a+b$ necessarily a zero-divisor?
\end{problem}
\begin{solution}
  No. For example, $[2]_6$ and $[3]_6$ are zero-divisors in $\mathbb{Z}/6\mathbb{Z}$, but 
  their sum $[5]_6$ is a unit in this ring, and hence is not a zero-divisor.
\end{solution}

\begin{problem}{1.6}
  An element $a$ of a ring $R$ is \textit{nilpotent} if $a^n=0$ for some $n$.
  \begin{enumerate}
    \item Prove that if $a$ and $b$ are nilpotent in $R$ and $ab=ba$, then $a+b$ is nilpotent.
    \item Is the hypothesis $ab=ba$ in the previous statement necessary for its conclusion
    to hold?
  \end{enumerate}
\end{problem}
\begin{solution}\ 
  \begin{enumerate}
    \item Suppose $a$ and $b$ are (nonzero, as this makes the conclusion obvious) 
      nilpotent elements of $R$, so that $a^m=b^n=0$ for positive integers $m,n$.
      Note, then, that, by the binomial theorem (as $R$ is commutative):
      \begin{equation*}
        (a+b)^{2mn} = \sum_{k=1}^{2mn} \binom{2mn}{k} a^{2mn-k}b^k \\
      \end{equation*}
      Notice that in each term of this sum, either $k\geq mn$ or $k\leq mn$, in which case
      $2mn-k\geq mn$.
      Since $mn\geq m$ and $mn\geq n$, this implies that in each term, either
      $b^k$ or $a^{2mn-k}$ is zero, meaning each term is zero, whence $a+b$ is nilpotent.  

    \item The hypothesis that $a$ and $b$ is indeed necessary.
        In the ring $\mathfrak{gl}_2(\mathbb{R})$, take the following matrices:
        \begin{equation*}
            A=\begin{bmatrix} 0 & 1 \\ 0 & 0 \end{bmatrix}\enspace;\enspace
            B=\begin{bmatrix} 0 & 0 \\ 1 & 0 \end{bmatrix}
        \end{equation*}
        We have $A^2=B^2=0$, but $A+B$ is of course not nilpotent.
  \end{enumerate}
\end{solution}

\begin{problem}{1.7}
  Prove that $[m]$ is nilpotent in $\mathbb{Z}/n{Z}$ is and only if $m$ is divisible
  by all prime factors of $n$.
\end{problem}
\begin{proof}
  First suppose $[m]$ is nilpotent in $\mathbb{Z}/n\mathbb{Z}$, so that 
  $[m]^k \equiv a \text{ mod } n$ for some positive integer $k$.
  This means that $n | m^k$, implying each prime factor of $n$ is also a 
  prime factor of $m^k$, and so is a prime factor of $m$ since $m^k$ and $m$ 
  have the same prime factors.\\
  \indent Conversely, suppose $m$ is divisible by all prime factors of $n$.
  Let $k$ be the largest exponent in the prime factorization of $n$.
  Then $n | m^k$, and so $[m]^k = [0]$.
\end{proof}

\begin{problem}{1.8}
  Prove that $x=\pm1$ are the only solutions to $x^2=1$ in an integral domain.
  Find a ring in which the equation $x^2=1$ has more than two solutions.
\end{problem}
\begin{solution}
  Note that:
    \begin{align*}
      x^2 = 1 & \Leftrightarrow x^2 - 1 = 0 \\
              & \Leftrightarrow (x+1)(x-1) = 0 \\
              & \Leftrightarrow x+1 = 0 \text{ or } x-1 = 0 \\
              & \Leftrightarrow x = \pm1
    \end{align*}
    where implications 1, 2, and 4 follow from the properties of rings, 
    and the third implication follows from our ring being an integral domain.\\
    \indent In the ring $\mathbb{Z}/3{Z}$, the equation $x^2=1$ has solutions $\pm1,\pm2$.
\end{solution}

\begin{problem}{10}
  Let $R$ be a ring.
  Prove that if $a\in R$ is a right-unit and has two or more left-zero-divisors, 
  then $a$ is \textit{not} a left-zero-divisor and \textit{is} a right-zero-divisor.
\end{problem}
\begin{solution}
  Let $u_1,u_2$ be distinct left inverses of $a$, so that $u_1a=u_2a=1$.
  We then have, for all $b\in R$,
    \begin{align*}
      ab=0 &\implies u_1ab=u_10\\
           &\implies b=0,
    \end{align*}
  and so $a$ is not a left-zero-divisor.
  Note, however, that 
    \begin{align*}
      (u_1-u_2)a &= u_1a-u_2a\\
                 &= 1-1\\
                 &= 0,
    \end{align*}
    and so $a$ is a right-zero-divisor since $u_1\neq u_2$.
\end{solution}

\begin{problem}{1.14}
  Let $R$ be a ring, and let $f(x),g(x)\in R[x]$ be nonzero polynomials.
  Prove that
  \begin{equation*}
    \deg(f(x)+g(x))\leq \max(\deg(f(x)), \deg(g(x))).
  \end{equation*}
  Assuming that $R$ is an integral domain, prove that
  \begin{equation*}
    \deg(f(x)\cdot g(x)) = \deg(f(x)) + \deg(g(x))
  \end{equation*}
\end{problem}
\begin{solution}
  Let $f(x)=\sum_i a_ix^i$ and $g(x)=\sum_i b_ix^i$.
  If $\deg(f(x))=\deg(g(x))=n$, then $i>n$ implies $a_i+b_i=0$, and so the largest $i$
  for which the coefficient $a_i+b_i$ is potentially nonzero coefficient is $i=n$.
  Hence $\deg(f(x)+g(x))\leq \max(\deg(f(x)), \deg(g(x)))$.\\
  \indent If $\deg(f(x))\neq\deg(g(x))$, then assume without loss of generality that\\
  $n = \deg(g(x)) > \deg(f(x))$.
  Of course $i > n$ then implies $a_i+b_i = 0$ and $a_n+b_n = a_n$, so the result holds.\\
  \indent Now, assume $R$ is an integral domain, and let $\deg(f(x)) = m$, $\deg(g(x))=n$.
  Since no $i\leq m, j\leq n$ implies $i+j\leq m+n$, we know then that the coefficient of
  $x^k$ in the product of $f$ and $g$ for $k>m+n$ is zero.
  Furthermore, this also implies that the coefficient of $x^{m+n}$ is $a_mb_n$ by the definition
  of the product of polynomials, which is nonzero since $a_m,b_n$ are nonzero and $R$ is an integral
  domain.
\end{solution}

\begin{problem}{1.15}
  Prove that $R[x]$ is an integral domain if and only if $R$ is an integral domain.
\end{problem}
\begin{solution}
  First suppose that $R[x]$ is an integral domain.
  Then every two constant polynomials $p = r_1, q=r_2$ in $R[x]$ satisfy $pq=0$ if and only if
  $p = 0$ or $q = 0$ as polynomials.
  But this is true if and only if $r_1=0$ or $r_2=0$ as elements of $R$, hence $R$ is an integral
  domain.\\
  \indent Conversely, suppose $R[x]$ is not an integral domain, that is, there exist nonzero polynomials
    \begin{align*}
      p &= \sum_ip_ix_i\\
      q &= \sum_iq_ix_i;
    \end{align*}
  (where $p_i$ and $q_i$ are elements of $R$) such that the polynomial $pq=0$.
  Let $m$ be the smallest positive integer such that $p_m$ is nonzero, and let $n$ be the smallest 
  positive integer such that $q_n$ is nonzero.
  Then $0 = (pq)_{m+n} = p_mq_n$ by the definition of the product of polynomials
  and the fact that $pq=0$, so $R$ is not an integral domain.
\end{solution}<++>
\end{document}
