%&pdflatex 
\documentclass[12pt]{article} 
\usepackage[margin=1in]{geometry} 
\usepackage{amsmath,amsthm,amssymb,amsfonts,tikz-cd, mathrsfs, enumitem} 

\newenvironment{problem}[2][Problem]{\begin{trivlist}
\item[\hskip \labelsep {\bfseries #1}\hskip \labelsep {\bfseries #2.}]}{\end{trivlist}}

\newenvironment{proposition}[1][Proposition]{\begin{trivlist}
\item[\hskip \labelsep {\bfseries #1.}]}{\end{trivlist}}

\newenvironment{definition}[1][Definition]{\begin{trivlist}
\item[\hskip \labelsep {\bfseries #1.}]}{\end{trivlist}}

\newcommand{\til}{\char`\~}
\newcommand{\bs}{\textbackslash} 

\newcommand{\catname}[1]{\normalfont\textbf{#1}}
\newcommand{\catsup}[2]{\normalfont\textbf{#1}^{#2}}
\newcommand{\catsub}[2]{\normalfont\textbf{#1}_{#2}}
\newcommand{\Alg}[1]{#1\catname{-Alg}}
\newcommand{\Mod}[1]{#1\catname{-Mod}}

\newcommand{\Hom}{\text{Hom}}
\newcommand{\Homc}[2]{\Hom_{\catname{#1}}(#2)}
\newcommand{\Homod}[2]{\Hom_{#1\catname{-Mod}}(#2)}

\newcommand{\Obj}{\text{Obj}}
\newcommand{\Objc}[1]{\text{Obj}(\catname{#1})}
\newcommand{\Aut}{\text{Aut}}
\newcommand{\End}{\text{End}}
\newcommand{\Endmod}[2]{\End_{#1\catname{-Mod}}(#2)}

\newcommand{\id}{\text{id}}
\newcommand{\im}{\text{im }}

\newcommand{\lcm}[1]{\text{lcm}(#1)}

\newcommand{\coker}{\text{coker }}

\newcommand{\N}{\mathbb{N}}
\newcommand{\Z}{\mathbb{Z}}
\newcommand{\Q}{\mathbb{Q}}
\newcommand{\R}{\mathbb{R}}
\newcommand{\C}{\mathbb{C}}

\newenvironment{solution}
  {\renewcommand\qedsymbol{$\blacksquare$}\begin{proof}[Solution]}
{\end{proof}}

\newenvironment{sproof}{
  \renewcommand\qedsymbol{$\square$}
  \begin{proof}
  }{
  \end{proof}
}

\newtheorem{lemma}{Lemma}

\theoremstyle{remark}
\newtheorem*{exmp}{Example}
\newtheorem*{cexmp}{Counterexample}

\begin{document}

\title{Algebra: Chapter 0 Exercises\\ \large Chapter 3, Section 6\\ 
\large Products, coproducts, etc. in $\Mod{R}$}
\author{David Melendez}
\maketitle

\begin{problem}{6.1}
  Prove that $R^{\oplus A}\cong F^R(A)$.
\end{problem}
\begin{proof}
  First, define $j:A\to R^{\oplus A}$ by $j(a)(b) = \delta_{ab}$,
  where $\delta$ is the Kronecker delta.
  We then have, for all $\alpha\in R^{\oplus A}$, that
  \begin{equation*}
    \alpha = \sum_{a\in A} \alpha(a)j(a),
  \end{equation*}
  since we have for all $x\in A$ that
  \begin{align*}
    \left( \sum_{a\in A} \alpha(a)j(a) \right)(x)
    &= \sum_{a\in A} (\alpha(a)j(a))(x) \\
    &= \sum_{a\in A} \alpha(a)(j(a)(x)) \\
    &= \sum_{a\in A} \alpha(a)\delta_{ax} \\
    &= \alpha(x).
  \end{align*}
  \indent Of course this representation of $\alpha$ as a linear combination
  of $j(a)$ for all $a\in A$ is unique, as the coefficients are clearly
  uniquely determined by the image of each $a\in A$ under $\alpha$. \\
  \indent Thus, if $N$ is an $R$-module, $f:A\to N$, 
  and $\varphi:R^{\oplus A}\to N$ is an $R$-module homomorphism 
  such that $\varphi j = f$, we then have, for all $\alpha\in R^{\oplus A}$,
  \begin{align*}
    \varphi(\alpha) &= \varphi\left( \sum_{a\in A}\alpha(a)j(a) \right) \\
    &= \sum_{a\in A}\varphi(\alpha(a)j(a)) \\
    &= \sum_{a\in A}\alpha(a)\varphi(j(a)) \\
    &= \sum_{a\in A}\alpha(a)f(a);
  \end{align*}
  thus such a homomorphism is unique, if it exists.
  Of course, this definition indeed defines a homomorphism that satisfies
  the desired property, as is easy to verify, and so $R^{\oplus A}$
  does satisfy the universal property for the free $R$-module over $A$.
\end{proof}

\begin{problem}{6.2}
  Prove or disprove that if $R$ is a ring and $M$ is a nonzero $R$-module,
  then $M$ is not isomorphic to $M\oplus M$.
\end{problem}
\begin{solution}
  As a counterexample, let $R$ be a ring and consider the $R$-module
  $M=R^{\oplus\N}$ (where $\N$ does not include 0), generated by
  the set $\{e_1,e_2,\dots\}$.
  Then, $M\oplus M$ is the cartesian product of $M$ with itself.
  Consider, then, the function $\varphi:M\to M\oplus M$, defined by
  \begin{equation*}
    \varphi\left( \sum_i r_ie_i \right)
    = \left( \sum_i r_{2i-1}e_i, \sum_i r_{2i}e_i \right).
  \end{equation*}
  As can be verified, $\varphi$ is an $R$-module homomorphism which is
  injective and surjective.
  Hence $M\cong M\oplus M$.
\end{solution}

\begin{problem}{6.3}
  Let $R$ be a ring, $M$ an $R$-module, and $p:M\to M$ an $R$-module
  homomorphism such that $p^2=p$ (Such a map is called a \textit{projection}).
  Prove that $M\cong\ker p\oplus\im p$.
\end{problem}
\begin{proof}
  Define the functions $\varphi:M\to\ker p\oplus\im p$
  and $\psi:\ker p\oplus\im p$ by 
  \begin{align*}
    \varphi(m) &= (m-p(m), p(m)) \\
    \psi(u,v) &= u + v.
  \end{align*}
  Note that $p(m)\in\im p$, and if $m\in M$, then 
  \begin{align*}
    p(m-p(m)) &= p(m)-p(p(m))  \\
    &= p(m)-p(m) \\
    &= 0;
  \end{align*}
  hence $m-p(m)\in \ker p$.
  Thus the definition of $\varphi$ makes sense.
  Past this, it is easy to verify that $\varphi$ and $\psi$ are
  $R$-module homomorphisms and that $\psi$ is a left and right inverse
  for $\varphi$; hence, $\varphi$ is an isomorphism between $M$
  and $\ker p\oplus\im p$.
\end{proof}

\begin{problem}{6.5}
  For any ring $R$ and any two sets $A_1,A_2$, prove that
  $\left(R^{\oplus A_1}\right)^{\oplus A_2}
  \cong R^{\oplus(A_1\times A_2)}$.
\end{problem}
\begin{proof}
  Let $\varphi: R^{\oplus(A_1\times A_2)}
  \to \left(R^{\oplus A_1}\right)^{\oplus A_2}$ be a function defined by 
  \begin{equation*}
    \Phi(\varphi)(a)(b) = \varphi(a,b).
  \end{equation*}
  Then $\Phi$ is an $R$-module isomorphism.
\end{proof}

\newpage

\begin{problem}{6.6}
  Let $R$ be a ring, and let $F=R^{\oplus n}$ be a finitely generated
  free $R$-module.
  Prove that $\Homod{R}{F,R}\cong F$.
\end{problem}
\begin{proof}
  Let $e_1,\dots,e_n$ be the generators of $F$, and
  for $0\leq i\leq n$, let $\psi_i:F\to R$ be defined by\\
  \begin{equation*}
    \psi_i\left( \sum_{j=1}^{n} r_je_j \right)
    = r_i.
  \end{equation*}
  Then each $\psi_i$ is well-defined and an $R$-module homomorphism.\\
  \indent Note, then, that for each $\varphi\in\Homod{R}{F,M}$
  and $v=\sum_i r_ie_i$, we have that
  \begin{align*}
    \varphi(v) &= \varphi\left( \sum_i r_ie_i \right) \\ 
    &= \sum_i \varphi(r_ie_i) \\
    &= \sum_i r_i\varphi(e_i) \\
    &= \sum_i \psi_i(v)\varphi(e_i) \\
    &= \left( \sum_i\varphi(e_i)\psi_i \right)(v);
  \end{align*}
  thus, if we let $s_i = \varphi(e_i)\psi_i$, then we have that
  $\varphi=\sum_i s_i\psi_i$,
  and so $\Homod{R}{F,R}$ is generated by $(\psi)_i$
  Indeed, each $\psi_i$ is in $\Homod{R}{F,R}$, and so 
  the module generated by them is contained within $\Homod{R}{F,R}$, as well.
  \\
  \indent We can then define a function $\Phi:\Homod{R}{F,R}\to F$ by
  \begin{equation}
    \Phi\left( \sum_i r_i\psi_i \right) = \sum_i r_ie_i,
  \end{equation}
  It is then easy to show that $\Phi$ is an $R$-module isomorphism.
\end{proof}

\begin{problem}{6.7}
  Let $A$ be any set.
  For any family $\{M_a\}_{a\in A}$ of modules over a ring $R$,
  define the \textit{product} $\prod_{a\in A}M_a$ and coproduct
  $\bigoplus_{a\in A} M_a$.
\end{problem}
\begin{solution}
  We define the product $P=\prod_{a\in A} M_a$ as follows:
  We say that $P$, along with a family of $R$-module 
  homomorphisms $\{\pi_a:P\to M_a\}_{a\in A}$
  is a product of the family $\{M_a\}_{a\in A}$ if for each 
  $R$-module $N$ and family of morphisms $\{\varphi_a:N\to M_a\}_{a\in A}$,
  there exists a unique $R$-module homomorphism 
  $\psi=\prod_{a\in A}\varphi_a:N\to P$ such that
  for all $a\in A$, we have $\pi_a\psi=\varphi_a$. \\
  \indent In the case where $M_a=R$ for all $a\in A$, we have that
  the set $R^A$ of functions from $A$ to $R$, along with the projections
  $\pi_a(g)=g(a)$ satisfies this universal property.
  Indeed, if $M$ is an $R$-module and we have a family of $R$-module
  homomorphisms $\{f_a:M\to R\}$, then we have that if $\psi:M\to R^A$ 
  is a function satisfying the condition $\pi_a\psi=f_a$, then
  \begin{align*}
    \psi(m)(a) &= \pi_a(\psi(m)) \\
    &= f_a(m);
  \end{align*}
  thus, $\psi(m)$ is the function taking $a$ to $f_a(m)$.
  It is easy to check that $\psi$ is an $R$-module homomorphism, and hence
  that it satisfies the desired universal property.
  \\
  \indent We define the coproduct $K=\bigoplus_{a\in A} M_a$ as follows:
  We say that $P$, along with a family of $R$-module 
  homomorphisms $\{\iota_a:M_a\to K\}_{a\in A}$
  is a coproduct of the family $\{M_a\}_{a\in A}$ if for each 
  $R$-module $N$ and family of morphisms $\{\varphi_a:M_a\to N\}_{a\in A}$,
  there exists a unique $R$-module homomorphism 
  $\psi=\bigoplus_{a\in A}\varphi_a:K\to N$ such that
  for all $a\in A$, we have $\psi\iota_a=\varphi_a$.
\end{solution}
  Prove that $\Z^\N\not\cong\Z^{\oplus\N}$. (Hint: Cardinality.)
\begin{proof}
  Note that $\Z^\N$ is the set of all infinite sequences of integers, which
  has cardinality equal to that of the reals.
  By contrast, $\Z^{\oplus\N}$ is countable (proof?).
\end{proof}

\begin{problem}{6.8}
  Let $R$ be a ring.
  If $A$ is any set, prove that $\Homod{R}{R^{\oplus A},R}$ satisfies
  the universal property for the \textit{product} of the family 
  $\{R_a\}_{a\in A}$, where $R_a\cong R$ for all $a$;
  thus, $\Homod{R}{R^{\oplus A},R}\cong R^A$.
  Conclude that $\Homod{R}{R^{\oplus A}, R}$ is not isomorphic to
  $R^{\oplus A}$ in general.
\end{problem}
\begin{solution}
  Alternatively, we can just prove directly using our characterization
  of the infinite product of a module with itself (done above) the desired
  isomorphism.\\ 
  \indent Let $\Phi:\Homod{R}{R^{\oplus A}, R}$ be the function defined by
  \begin{equation*}
    \Phi(\rho) = \rho j,
  \end{equation*}
  where $j$ is the usual inclusion from $A$ into $R^{\oplus A}$.
  It is easily verified that this is an $R$-module homomorphism.
  We can then see that since $R^{\oplus A}$ is the free $R$-module
  generated by $A$, that for every $f\in R^A$, there exists a unique
  $\rho^{\oplus A}\to R$ such that $\rho j=f$; that is,
  $\Phi$ is a bijection, and hence an isomorphism, as desired.\\
\end{solution}

\begin{problem}{6.9}
  Let $R$ be a ring, $F$ a nonzero free $R$-module, and let 
  $\varphi:M\to N$ be an $R$-module homomorphism.
  Prove that $\varphi$ is onto if and only if for all $R$-module homomorphisms
  $\alpha:F\to N$, there exists an $R$-module homomorphism $\beta:F\to M$
  such that $\alpha=\varphi\circ\beta$.
  (Free modules are \textit{projective})
\end{problem}
\begin{proof}
  First suppose $\varphi$ is surjective.
  Let $A$ be the set of generators of $F$ let $j:A\to F$ be the usual 
  inclusion, and let $f=\alpha j$.
  Note that for each $n\in\alpha j(A)$, there exists a (not necessarily unique)
  $m_n\in M$ such that $\varphi(m_n)=n$, since $\varphi$ is surjective.
  Define, then, a function $g:A\to M$ by $g(a) = m_{f(a)}$, and extend
  $g$ to a function $\beta:F\to M$.
  We then have that
  \begin{align*}
    \varphi\beta\sum_{a\in A} r_aj(a) &= \varphi\sum_{a\in A}r_a\beta j(a) \\
    &= \varphi\sum_{a\in A}r_ag(a) \\
    &= \varphi\sum_{a\in A}r_am_{f(a)} \\
    &= \sum_{a\in A}r_a\varphi(m_{f(a)}) \\
    &= \sum_{a\in A}r_af(a) \\
    &= \sum_{a\in A}r_a\alpha j(a) \\
    &= \alpha\sum_{a\in A}r_aj(a), 
  \end{align*}
  where the sum in the first expression is an arbitrary element of $F$.
  Thus $\varphi\beta=\alpha$, as desired.\\
  \indent Conversely, suppose that for all $R$-module homomorphisms
  $\alpha:F\to N$, there exists an $R$-module homomorphism $\beta:F\to M$
  such that $\alpha=\varphi\beta$.
  Then, suppose $n\in N$, and consider the $R$-module homomorphism
  $\alpha:F\to N$ extending the constant set-function $a\mapsto n$.
  We then have that there exists a $\beta:F\to M$ such that 
  $\alpha=\varphi\beta$.
  In particular, $\varphi\beta(j(a)) = \alpha(j(a)) = f(a) = n$,
  and so we have that $n\in\im\varphi$.
  Since $n$ was arbitrary, it then follows that $\varphi$ is surjective,
  as desired.
\end{proof}

\begin{problem}{6.10}
  Let $M,N,Z$ be $R$-modules, and let $\mu:M\to Z$ and $\nu:N\to Z$
  be homomorphisms of $R$-modules.\\
  \indent Prove that $\Mod{R}$ has 'fibered products':
  there exists an $R$-module $M\times_ZN$ with $R$-module homomorphisms
  $\pi_M:M\times_ZN\to M$ and $\pi_N:M\times_ZN\to N$ such that
  $\mu\pi_M=\nu\pi_N$, and which is universal with respect to this requirement.
  That is, for every $R$-module $P$ and $R$-module homomorphisms
  $\varphi_M:P\to M$,$\varphi_N:P\to N$ such that
  $\mu\varphi_M=\nu\varphi_N$, there exists a unique
  $R$-module homomorphism $\psi:P\to M\times_ZN$ making the diagram
  \[
    \begin{tikzcd}
      &&& M\ar{dr}{\mu} & \\
      & P \ar[dashed]{rr}{\psi} 
      \ar{urr}{\varphi_M} 
      \ar{drr}[swap]{\varphi_N}
      && M\times_ZN \ar{u}{\pi_M} \ar{d}{\pi_N} 
      & Z \\
      &&& N\ar{ur}[swap]{\nu} & \\
    \end{tikzcd}
  \]
  commute.
\end{problem}
\begin{solution}
  Define $M\times_ZN=\{(m,n)\in M\times N:\mu(m)=\nu(n)\}$.
  That this is an $R$-module follows immediately from
  $\mu,\nu$ being $R$-module homomorphisms.
  The usual projections also clearly satisfy $\mu\pi_M=\nu\pi_N$, and
  the desired unique homomorphism $\psi$ is defined by
  $\psi(p)=(\varphi_M(p),\varphi_N(p))$, which makes sense since
  we required that $\mu\varphi_M$ and $\nu\varphi_N$ agree.
  The case for fibered coproducts is similarly.
\end{solution}

\begin{problem}{6.12}
  Prove Proposition 6.2:
  For an $R$-module homomorphism $\varphi$, the following are equivalent:
  \begin{enumerate}[label=(\alph*)]
    \item $\varphi$ is a monomorphism 
    \item $\ker\varphi$ is trivial
    \item $\varphi$ is injective as a set function.
  \end{enumerate}
  Additionally, the following are equivalent:
  \begin{enumerate}[label=(\alph*)]
    \item $\varphi$ is an epimorphism 
    \item $\coker\varphi$ is trivial
    \item $\varphi$ is surjective as a set function.
  \end{enumerate}
\end{problem}
\begin{solution}
  For the first, part, let $\varphi:M\to N$ be an $R$-module homomorphism.
  If $\varphi$ is injective as a set-function, then $\varphi$ is mono
  as a set-function, and in particular as an $R$-module homomorphism.
  Thus (c) implies (a).
  Additionally, we know that an $R$-module homomorphism has trivial kernel
  if and only if it is injective, and so in particular (b) implies (c).\\
  \indent Assume, then, that $\varphi$ is a monomorphism; that is,
  for all $R$-modules $P$ and $R$-module homomorphisms 
  $\alpha_1,\alpha_2:P\to M$, we have that $\varphi\alpha_1=\varphi\alpha_2$
  implies $\alpha_1=\alpha_2$.
  Let $P=\ker\varphi$, and consider $\alpha_1=\iota$, the inclusion into $M$,
  and $\alpha_2=0$, the trivial homomorphism.
  We then have that $\varphi\circ0 = \varphi\circ\iota$, and so
  $\iota=0$; thus, $\ker\varphi=\im\iota=0$.
  Therefore, (a) implies (b), and we have the desired equivalence.\\
  \indent For the second part, first note that if $\varphi$ is surjective
  as a set function, then $\varphi$ is epi as a set-function, and in 
  particular as an $R$-module homomorphism.
  Thus (c) implies (a).
  Additionally, if $\varphi$ has trivial cokernel, then we have that
  $N/\im\varphi=0$, and so $\im\varphi = N$; thus, $\varphi$ is surjective,
  giving us that (b) implies (c).\\
  \indent Now, assume that $\varphi$ is an epimorphism, so that
  $\alpha_1\varphi=\alpha_2\varphi$ implies $\alpha_1=\alpha_2$ for all
  $R$-modules $P$ and homomorphisms $\alpha_1,\alpha_2:N\to P$.
  In particular, let $P=\coker\varphi$, $\alpha_1=\pi$ be the projection,
  and $\alpha_2=0$ be the zero homomorphism.
  Then $\pi\circ\varphi=0=0\circ\varphi$, and so $\pi=0$; hence 
  $\coker\varphi=\pi(N)=0(N)=0$.
  Therefore, (a) implies (b), and we have the desired equivalence.
\end{solution}

\newpage

\begin{problem}{6.13}
  Prove that every homomorphic image of a finitely generated module is
  finitely generated.
\end{problem}
\begin{solution}
  By definition, an $R$-module $M$ is finitely generated if and only if
  there exists a finite set $A$ and a function $\iota:A\to M$
  such that the $R$-module homomorphism $\gamma:F^R(A)\to M$
  induced by $\iota$ is surjective. \\
  \indent Suppose, then, that $M$ is finitely generated so that we have such 
  a set $A$ and a function $\iota$ which induce a surjection $\gamma$, and 
  let $\varphi:M\to N$ be a surjective $R$-module homomorphism.
  Stare at the following diagram
  \[
    \begin{tikzcd}
      & F^R(A) \ar{r}{\gamma} & M \ar{r}{\varphi} & N \\
      & A \ar{u}{j} \ar{ur}[swap]{\iota}
    \end{tikzcd}
  \]
  and note that since $\gamma j=\iota$, we then have that 
  $(\varphi\gamma)j=\varphi\iota$; thus, by the uniqueness clause of the
  universal property for free modules, $\varphi\gamma$ is the unique
  $R$-module homomorphism $F^R(A)\to N$ induced by $\varphi\iota$.
  Since $\gamma$ and $\varphi$ are surjective, we then have that
  $\varphi\gamma$ is surjective, and so $N$ is finitely generated, as desired.
\end{solution}

\begin{problem}{6.14}
  Prove that the ideal $I=(x_1,x_2,\dots)$ of the ring $R=\Z[x_1,x_2,\dots]$
  is not finitely generated (as an ideal, i.e. as an $R$-module).
\end{problem}
\begin{solution}
  Assume $I$ is finitely generated by a set $G=\{g_1,\dots,g_n\}\subseteq I$,
  and assume without loss of generality that each $g_i$ is a monomial.
  Let $N$ be the largest integer such that some $g_i$ is divisible by $x_N$,
  and consider the ring homomorphism $\varphi:R\to R$ (induced by the universal
  property for polynomial rings) that maps each integer to itself, 
  $x_i\mapsto0$ for $i<N$, and $x_i\mapsto x_i$ for $i\geq N$.
  If $x_N=\sum a_ig_i$ for some polynomials $a_i$, then we have
  \begin{align*}
    x_N &= \varphi(x_N) \\
    &= \varphi\left(\sum_ia_ig_i\right) \\
    &= \sum_i \varphi(a_i)\varphi(g_i) \\
    &= 0,
  \end{align*}
  where the last equality follows from $g_i$ not being divisibly by $x_N$.
  This is a contradiction; hence $I$ is not finitely generated.
\end{solution}

\begin{problem}{6.15}
  Let $R$ be a commutative ring.
  Prove that a commutative $R$-algebra $S$ is finitely generated as an
  algebra over $R$ if and only if it is finitely generated as a commutative
  algebra over $R$.
\end{problem}
\begin{proof}
  First suppose $S$ is finitely generated as a commutative $R$-algebra.
  Then, employing the universal property for free $R$-algebras,
  there exists a finite set $A$ and a set-function $\gamma:A\to S$
  such that the unique $R$-algebra homomorphism $R[A]\to S$ induced by
  $\gamma$ is a surjection. \\
  \indent Additionally, the natural inclusion $A\hookrightarrow R[A]$
  induces an $R$-algebra homomorphism $R\langle A\rangle\to R[A]$ which
  is clearly surjective.
  Consequently, we consider the following diagram,
  \[
    \begin{tikzcd}
      & R\langle A\rangle \ar[r, two heads, "\varphi"]
      & R[A] \ar[r, two heads, "\psi"]
      & S \\
      & A \ar[u, hook, "j"] 
          \ar[ur, hook, "j'"'] 
          \ar[urr, "\gamma"', bend right]
    \end{tikzcd}
  \]
  where $\varphi j=j'$ and $\psi j'=\gamma$.
  Note, then that we have $\psi\varphi j=\psi j'=\gamma$, and so
  $\psi\varphi$ is the unique $R$-algebra homomorphism 
  $R\langle A\rangle\to R[A]$ induced by $\gamma$.
  Since this homomorphism is the composition of two surjections, it itself
  is a surjection, and so $S$ is finitely generated as an $R$-algebra, 
  as desired. \\
  \indent Suppose conversely that $S$ is finitely generated as an $R$-algebra.
  Let $\mathfrak{c}$ be the centralizer ideal of $R\langle A\rangle$--that is,
  the ideal generated by all $ab-ba$ for $a,b\in R\langle A\rangle$.
  We then have that $R\langle A\rangle/\mathfrak{c}\cong R[A]$ (proof?).
  Observe, then, the following diagram,
  \[
    \begin{tikzcd}
      & \displaystyle\frac{R\langle A\rangle}{\mathfrak{c}} 
      \ar[dr, "\widetilde\varphi"] \\
      & R\langle A\rangle \ar[u, "\pi"] \ar[r, "\varphi"] & S \\
      & A \ar[u, "j"] \ar[ur, "\gamma"']
    \end{tikzcd}
  \]
  where $\pi$ is the projection, $\varphi$ is induced by $\gamma$, and
  $\widetilde\varphi$ is induced by the universal property for quotient
  algebras. \\
  \indent If we think of $R\langle A\rangle/\mathfrak{c}$ as $R[A]$, then
  $\pi j$ is the inclusion $A\hookrightarrow R[A]$, and we have that
  $\widetilde\varphi\pi j=\varphi j=\gamma$, and so $\widetilde\varphi$ is
  the morphism induced by $\gamma$ and the universal property for free
  commutative $R$-algebras.
  Since $\widetilde\varphi$ is surjective (because $\varphi$ is surjective),
  it then follows that $S$ is finitely generated as a commutative $R$-algebra,
  as desired.
\end{proof}

\begin{problem}{6.16}
  Let $R$ be a ring.
  A (left-)$R$ module is cyclic if $M=\langle m\rangle$ for some $m\in M$.
  Prove that simple modules are cyclic.
  Prove that an $R$-module $M$ is cyclic if and only if $M\cong R/I$ for some
  (left-)ideal $I$.
  Prove that every quotient of a cyclic module is cyclic.
\end{problem}
\begin{solution}
  Recall that a simple module is a module with only trivial (0 and itself)
  submodules.
  Suppose, then, that $M$ is a simple $R$-module.
  Let $m$ be any nonzero element of $M$, and let $N=\langle m\rangle$.
  Certainly $m\in N$ since $1m=m$, so $N$ is nonempty.
  Since $M$ is simple, it then follows that $N=M$ and so $M=\langle m\rangle$
  as desired.\\
  \indent For the next part, suppose that $M$ is cyclic, so that 
  $M=\langle m\rangle$.
  Define an $R$-module homomorphism $\varphi:R\to M$ by $\varphi(r)=rm$.
  Since $M$ is generated by $m$, we have that $\varphi$ is surjective,
  and so $R/\ker\varphi\cong M$.
  Thus simple $R$ modules are quotients of $R$. \\
  \indent Conversely, suppose $M\cong R/I$ as $R$-modules for some ideal $I$
  of $R$.
  We then have an isomorphism $\widetilde\varphi:R/I\to M$.
  Define, then, a homomorphism $\varphi:R\to I$ by 
  $\varphi(r)=\widetilde\varphi(r+I)$.
  Clearly $\varphi$ is surjective, and so for every $m\in M$, we have that
  there exists an $r\in R$ such that $\varphi(r)=m$.
  Note, then, that we have $m=\varphi(r)=r\cdot\varphi(1)$, and so
  $M$ is generated by $\varphi(1)$, showing that $M$ is cyclic as desired.\\
  \indent For the last part, it's enough to note that if $M$ is generated
  by $m$, then $\pi(m)$ generates quotients of $M$.
\end{solution}

\begin{problem}{6.17}
  Let $M$ be a cyclic $R$-module, so that $M\cong R/I$ for a (left-)ideal
  $I$, and let $N$ be another $R$-module.
  \begin{enumerate}[label=(\alph*)]
    \item Prove that $\Homod{R}{M,N}\cong\{n\in N:(\forall a\in I),an=0\}$.
    \item For $a,b\in\Z$, prove that 
      $\Homc{Ab}{\Z/a\Z,\Z/b\Z}\cong\Z/\gcd(a,b)\Z$.
  \end{enumerate}
\end{problem}
\begin{solution}
  For (a), let $P=\{n\in N:(\forall a\in I),an=0\}$, and define a function
  $\psi:P\to\Homod{R}{M,N}$ by $\psi(n)([r])=rn$.
  The function $\psi$ is well-defined as a result of the condition on $P$,
  and is an $R$-module homomorphism.
  Note that $n\in\ker\psi\implies(\forall r\in R)rn=0$, and so in particular,
  $1\cdot n=n=0$.
  Thus, $\psi$ is injective. \\
  \indent Note additionally that if $\varphi\in \Homod{R}{M,N}$, then for all
  $r\in R$, we have $\varphi([r])=\varphi(r\cdot[1])=r\varphi([1])$;
  thus, $\varphi$ is entirely determined by where it takes $[1]$.
  Therefore, if $\varphi([r])=rn$ for some $n\in N$, then we have
  $\varphi=\psi(n)$, and so $\psi$ is surjective, as desired. \\
  \indent The second result follows immediately if we let $M=\Z/a\Z$ and
  $N=\Z/b\Z$.
\end{solution}

\begin{problem}{6.18}
  Let $M$ be an $R$-module, and let $N$ be a submodule of $N$.
  Prove that if $N$ and $M/N$ are both finitely generated, then $M$
  is finitely generated.
\end{problem}
\begin{proof}
  Suppose $N$ is finitely generated by $n_1,\dots,n_k$,
  and $M/N$ is finitely generated by $[m_1],\dots,[m_\ell]$ for some 
  $m_1,\dots,m_\ell\in M$.
  We then have that $m=\sum_i m_i+N$. Note that $m-\sum_i m_i\in N$ since
  it is in the kernel of the projection $M\to M/N$, and so we have
  \begin{align*}
    m &= \sum_im_i - \left(m+\sum_im_i\right) \\
    &= \sum+im_i - \sum_in_i,
  \end{align*}
  since $N$ is finitely generated.
  Therefore, $M$ is generated by $m_1,\dots,m_\ell,n_1,\dots,n_k$, as desired.
\end{proof}

\end{document}
