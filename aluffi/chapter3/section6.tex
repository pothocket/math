%&pdflatex 
\documentclass[12pt]{article} 
\usepackage[margin=1in]{geometry} 
\usepackage{amsmath,amsthm,amssymb,amsfonts,tikz-cd, mathrsfs, enumitem} 

\newenvironment{problem}[2][Problem]{\begin{trivlist}
\item[\hskip \labelsep {\bfseries #1}\hskip \labelsep {\bfseries #2.}]}{\end{trivlist}}

\newenvironment{proposition}[1][Proposition]{\begin{trivlist}
\item[\hskip \labelsep {\bfseries #1.}]}{\end{trivlist}}

\newenvironment{definition}[1][Definition]{\begin{trivlist}
\item[\hskip \labelsep {\bfseries #1.}]}{\end{trivlist}}

\newcommand{\til}{\char`\~}
\newcommand{\bs}{\textbackslash} 

\newcommand{\catname}[1]{\normalfont\textbf{#1}}
\newcommand{\catsup}[2]{\normalfont\textbf{#1}^{#2}}
\newcommand{\catsub}[2]{\normalfont\textbf{#1}_{#2}}
\newcommand{\Alg}[1]{#1\catname{-Alg}}
\newcommand{\Mod}[1]{#1\catname{-Mod}}

\newcommand{\Hom}{\text{Hom}}
\newcommand{\Homc}[2]{\Hom_{\catname{#1}}(#2)}
\newcommand{\Homod}[2]{\Hom_{#1\catname{-Mod}}(#2)}

\newcommand{\Obj}{\text{Obj}}
\newcommand{\Objc}[1]{\text{Obj}(\catname{#1})}
\newcommand{\Aut}{\text{Aut}}
\newcommand{\End}{\text{End}}
\newcommand{\Endmod}[2]{\End_{#1\catname{-Mod}}(#2)}

\newcommand{\id}{\text{id}}
\newcommand{\im}{\text{im }}

\newcommand{\lcm}[1]{\text{lcm}(#1)}

\newcommand{\N}{\mathbb{N}}
\newcommand{\Z}{\mathbb{Z}}
\newcommand{\Q}{\mathbb{Q}}
\newcommand{\R}{\mathbb{R}}
\newcommand{\C}{\mathbb{C}}

\newenvironment{solution}
  {\renewcommand\qedsymbol{$\blacksquare$}\begin{proof}[Solution]}
{\end{proof}}

\newenvironment{sproof}{
  \renewcommand\qedsymbol{$\square$}
  \begin{proof}
  }{
  \end{proof}
}

\newtheorem{lemma}{Lemma}

\theoremstyle{remark}
\newtheorem*{exmp}{Example}
\newtheorem*{cexmp}{Counterexample}

\begin{document}

\title{Algebra: Chapter 0 Exercises\\ \large Chapter 3, Section 6\\ 
\large Products, coproducts, etc. in $\Mod{R}$}
\author{David Melendez}
\maketitle

\begin{problem}{6.1}
  Prove that $R^{\oplus A}\cong F^R(A)$.
\end{problem}
\begin{proof}
  First, define $j:A\to R^{\oplus A}$ by $j(a)(b) = \delta_{ab}$,
  where $\delta$ is the Kronecker delta.
  We then have, for all $\alpha\in R^{\oplus A}$, that
  \begin{equation*}
    \alpha = \sum_{a\in A} \alpha(a)j(a),
  \end{equation*}
  since we have for all $x\in A$ that
  \begin{align*}
    \left( \sum_{a\in A} \alpha(a)j(a) \right)(x)
    &= \sum_{a\in A} (\alpha(a)j(a))(x) \\
    &= \sum_{a\in A} \alpha(a)(j(a)(x)) \\
    &= \sum_{a\in A} \alpha(a)\delta_{ax} \\
    &= \alpha(x).
  \end{align*}
  \indent Of course this representation of $\alpha$ as a linear combination
  of $j(a)$ for all $a\in A$ is unique, as the coefficients are clearly
  uniquely determined by the image of each $a\in A$ under $\alpha$. \\
  \indent Thus, if $N$ is an $R$-module, $f:A\to N$, 
  and $\varphi:R^{\oplus A}\to N$ is an $R$-module homomorphism 
  such that $\varphi j = f$, we then have, for all $\alpha\in R^{\oplus A}$,
  \begin{align*}
    \varphi(\alpha) &= \varphi\left( \sum_{a\in A}\alpha(a)j(a) \right) \\
    &= \sum_{a\in A}\varphi(\alpha(a)j(a)) \\
    &= \sum_{a\in A}\alpha(a)\varphi(j(a)) \\
    &= \sum_{a\in A}\alpha(a)f(a);
  \end{align*}
  thus such a homomorphism is unique, if it exists.
  Of course, this definition indeed defines a homomorphism that satisfies
  the desired property, as is easy to verify, and so $R^{\oplus A}$
  does satisfy the universal property for the free $R$-module over $A$.
\end{proof}

\begin{problem}{6.2}
  Prove or disprove that if $R$ is a ring and $M$ is a nonzero $R$-module,
  then $M$ is not isomorphic to $M\oplus M$.
\end{problem}
\begin{solution}
  As a counterexample, let $R$ be a ring and consider the $R$-module
  $M=R^{\oplus\N}$ (where $\N$ does not include 0), generated by
  the set $\{e_1,e_2,\dots\}$.
  Then, $M\oplus M$ is the cartesian product of $M$ with itself.
  Consider, then, the function $\varphi:M\to M\oplus M$, defined by
  \begin{equation*}
    \varphi\left( \sum_i r_ie_i \right)
    = \left( \sum_i r_{2i-1}e_i, \sum_i r_{2i}e_i \right).
  \end{equation*}
  As can be verified, $\varphi$ is an $R$-module homomorphism which is
  injective and surjective.
  Hence $M\cong M\oplus M$.
\end{solution}

\begin{problem}{6.3}
  Let $R$ be a ring, $M$ an $R$-module, and $p:M\to M$ an $R$-module
  homomorphism such that $p^2=p$ (Such a map is called a \textit{projection}).
  Prove that $M\cong\ker p\oplus\im p$.
\end{problem}
\begin{proof}
  Define the functions $\varphi:M\to\ker p\oplus\im p$
  and $\psi:\ker p\oplus\im p$ by 
  \begin{align*}
    \varphi(m) &= (m-p(m), p(m)) \\
    \psi(u,v) &= u + v.
  \end{align*}
  Note that $p(m)\in\im p$, and if $m\in M$, then 
  \begin{align*}
    p(m-p(m)) &= p(m)-p(p(m))  \\
    &= p(m)-p(m) \\
    &= 0;
  \end{align*}
  hence $m-p(m)\in \ker p$.
  Thus the definition of $\varphi$ makes sense.
  Past this, it is easy to verify that $\varphi$ and $\psi$ are
  $R$-module homomorphisms and that $\psi$ is a left and right inverse
  for $\varphi$; hence, $\varphi$ is an isomorphism between $M$
  and $\ker p\oplus\im p$.
\end{proof}

\begin{problem}{6.5}
  For any ring $R$ and any two sets $A_1,A_2$, prove that
  $\left(R^{\oplus A_1}\right)^{\oplus A_2}
  \cong R^{\oplus(A_1\times A_2)}$.
\end{problem}
\begin{proof}
  Let $\varphi: R^{\oplus(A_1\times A_2)}
  \to \left(R^{\oplus A_1}\right)^{\oplus A_2}$ be a function defined by 
  \begin{equation*}
    \Phi(\varphi)(a)(b) = \varphi(a,b).
  \end{equation*}
  Then $\Phi$ is an $R$-module isomorphism.
\end{proof}

\newpage

\begin{problem}{6.6}
  Let $R$ be a ring, and let $F=R^{\oplus n}$ be a finitely generated
  free $R$-module.
  Prove that $\Homod{R}{F,R}\cong F$.
\end{problem}
\begin{proof}
  Let $e_1,\dots,e_n$ be the generators of $F$, and
  for $0\leq i\leq n$, let $\psi_i:F\to R$ be defined by\\
  \begin{equation*}
    \psi_i\left( \sum_{j=1}^{n} r_je_j \right)
    = r_i.
  \end{equation*}
  Then each $\psi_i$ is well-defined and an $R$-module homomorphism.\\
  \indent Note, then, that for each $\varphi\in\Homod{R}{F,M}$
  and $v=\sum_i r_ie_i$, we have that
  \begin{align*}
    \varphi(v) &= \varphi\left( \sum_i r_ie_i \right) \\ 
    &= \sum_i \varphi(r_ie_i) \\
    &= \sum_i r_i\varphi(e_i) \\
    &= \sum_i \psi_i(v)\varphi(e_i) \\
    &= \left( \sum_i\varphi(e_i)\psi_i \right)(v);
  \end{align*}
  thus, if we let $s_i = \varphi(e_i)\psi_i$, then we have that
  $\varphi=\sum_i s_i\psi_i$,
  and so $\Homod{R}{F,R}$ is generated by $(\psi)_i$
  Indeed, each $\psi_i$ is in $\Homod{R}{F,R}$, and so 
  the module generated by them is contained within $\Homod{R}{F,R}$, as well.
  \\
  \indent We can then define a function $\Phi:\Homod{R}{F,R}\to F$ by
  \begin{equation}
    \Phi\left( \sum_i r_i\psi_i \right) = \sum_i r_ie_i,
  \end{equation}
  It is then easy to show that $\Phi$ is an $R$-module isomorphism.
\end{proof}

\begin{problem}{6.7}
  Let $A$ be any set.
  For any family $\{M_a\}_{a\in A}$ of modules over a ring $R$,
  define the \textit{product} $\prod_{a\in A}M_a$ and coproduct
  $\bigoplus_{a\in A} M_a$.
\end{problem}
\begin{solution}
  We define the product $P=\prod_{a\in A} M_a$ as follows:
  We say that $P$, along with a family of $R$-module 
  homomorphisms $\{\pi_a:P\to M_a\}_{a\in A}$
  is a product of the family $\{M_a\}_{a\in A}$ if for each 
  $R$-module $N$ and family of morphisms $\{\varphi_a:N\to M_a\}_{a\in A}$,
  there exists a unique $R$-module homomorphism 
  $\psi=\prod_{a\in A}\varphi_a:N\to P$ such that
  for all $a\in A$, we have $\pi_a\psi=\varphi_a$. \\
  \indent In the case where $M_a=R$ for all $a\in A$, we have that
  the set $R^A$ of functions from $A$ to $R$, along with the projections
  $\pi_a(g)=g(a)$ satisfies this universal property.
  Indeed, if $M$ is an $R$-module and we have a family of $R$-module
  homomorphisms $\{f_a:M\to R\}$, then we have that if $\psi:M\to R^A$ 
  is a function satisfying the condition $\pi_a\psi=f_a$, then
  \begin{align*}
    \psi(m)(a) &= \pi_a(\psi(m)) \\
    &= f_a(m);
  \end{align*}
  thus, $\psi(m)$ is the function taking $a$ to $f_a(m)$.
  It is easy to check that $\psi$ is an $R$-module homomorphism, and hence
  that it satisfies the desired universal property.
  \\
  \indent We define the coproduct $K=\bigoplus_{a\in A} M_a$ as follows:
  We say that $P$, along with a family of $R$-module 
  homomorphisms $\{\iota_a:M_a\to K\}_{a\in A}$
  is a coproduct of the family $\{M_a\}_{a\in A}$ if for each 
  $R$-module $N$ and family of morphisms $\{\varphi_a:M_a\to N\}_{a\in A}$,
  there exists a unique $R$-module homomorphism 
  $\psi=\bigoplus_{a\in A}\varphi_a:K\to N$ such that
  for all $a\in A$, we have $\psi\iota_a=\varphi_a$.
\end{solution}
  Prove that $\Z^\N\not\cong\Z^{\oplus\N}$. (Hint: Cardinality.)
\begin{proof}
  Note that $\Z^\N$ is the set of all infinite sequences of integers, which
  has cardinality equal to that of the reals.
  By contrast, $\Z^{\oplus\N}$ is countable (proof?).
\end{proof}

\begin{problem}
  Let $R$ be a ring.
  If $A$ is any set, prove that $\Homod{R}{R^{\oplus A},R}$ satisfies
  the universal pro
\end{problem}
\end{document}
