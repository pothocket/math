%&pdflatex 
\documentclass[12pt]{article} 
\usepackage[margin=1in]{geometry} 
\usepackage{amsmath,amsthm,amssymb,amsfonts,tikz-cd, mathrsfs, enumitem} 

\newenvironment{problem}[2][Problem]{\begin{trivlist}
\item[\hskip \labelsep {\bfseries #1}\hskip \labelsep {\bfseries #2.}]}{\end{trivlist}}

\newenvironment{proposition}[1][Proposition]{\begin{trivlist}
\item[\hskip \labelsep {\bfseries #1.}]}{\end{trivlist}}

\newenvironment{definition}[1][Definition]{\begin{trivlist}
\item[\hskip \labelsep {\bfseries #1.}]}{\end{trivlist}}

\newcommand{\til}{\char`\~}
\newcommand{\bs}{\textbackslash} 

\newcommand{\catname}[1]{\normalfont\textbf{#1}}
\newcommand{\catsup}[2]{\normalfont\textbf{#1}^{#2}}
\newcommand{\catsub}[2]{\normalfont\textbf{#1}_{#2}}
\newcommand{\Alg}[1]{#1\catname{-Alg}}
\newcommand{\Mod}[1]{#1\catname{-Mod}}

\newcommand{\Hom}{\text{Hom}}
\newcommand{\Homc}[2]{\Hom_{\catname{#1}}(#2)}
\newcommand{\Homod}[2]{\Hom_{#1\catname{-Mod}}(#2)}

\newcommand{\Obj}{\text{Obj}}
\newcommand{\Objc}[1]{\text{Obj}(\catname{#1})}
\newcommand{\Aut}{\text{Aut}}
\newcommand{\End}{\text{End}}
\newcommand{\Endmod}[2]{\End_{#1\catname{-Mod}}(#2)}

\newcommand{\id}{\text{id}}
\newcommand{\im}{\text{im }}

\newcommand{\lcm}[1]{\text{lcm}(#1)}

\newcommand{\coker}{\text{coker }}

\newcommand{\N}{\mathbb{N}}
\newcommand{\Z}{\mathbb{Z}}
\newcommand{\Q}{\mathbb{Q}}
\newcommand{\R}{\mathbb{R}}
\newcommand{\C}{\mathbb{C}}

\newenvironment{solution}
  {\renewcommand\qedsymbol{$\blacksquare$}\begin{proof}[Solution]}
{\end{proof}}

\newenvironment{sproof}{
  \renewcommand\qedsymbol{$\square$}
  \begin{proof}
  }{
  \end{proof}
}

\newtheorem{lemma}{Lemma}

\theoremstyle{remark}
\newtheorem*{exmp}{Example}
\newtheorem*{cexmp}{Counterexample}

\begin{document}

\title{Algebra: Chapter 0 Exercises\\ \large Chapter 3, Section 7\\ 
\large Complexes and homology}
\author{David Melendez}
\maketitle

\begin{problem}{7.1}
  Assume the complex
  \[
  \begin{tikzcd}
    &\cdots \ar[r] &0 \ar[r,"f"] &M \ar[r,"g"] &0 \ar[r] &\cdots
  \end{tikzcd}
  \]
  is exact.
  Prove $M\cong0$.
  \begin{proof}
    By exactness, we have that $M=\ker g=\im f=0$.
  \end{proof}
\end{problem}

\begin{problem}{7.2}
  Assume that the complex
  \[
    \begin{tikzcd}
      &\cdots \ar[r] &0 \ar[r] &M \ar[r,"\varphi"] &M' \ar[r] &0 \ar[r] &\cdots
    \end{tikzcd}
  \]
  is exact.
  Prove that $M\cong M'$.
\end{problem}
\begin{proof}
  By exactness, we have that $\ker\varphi=\im\{0:0\to M\} = 0$ and $\im\varphi=\ker\{0:M'\to 0\}=M'$
  Thus $\varphi$ is an isomorphism, as desired.
\end{proof}

\begin{problem}{7.3}
  Assume the complex
  \[
    \begin{tikzcd}
      &0\ar[r] &L \ar[r,"\alpha"] &M \ar[r,"\varphi"] &M' \ar[r,"\beta"] &N \ar[r] &0
    \end{tikzcd}
  \]
  is exact.
  Show that up to natural identifications, $L=\ker\varphi$ and $N=\coker\varphi$.
\end{problem}
\begin{proof}
  By exactness, we have that $\ker\varphi=\im\alpha\cong L$ (since $\alpha$ is injective),
  and $\coker\varphi=M'/\im\varphi=M'/\ker\beta\cong N$, since $\beta$ is surjective.
\end{proof}

\begin{problem}{7.4}
  Construct short exact sequences of $\Z-modules$
  \[
    \begin{tikzcd}
      &(a) &0 \ar[r] &\Z^{\oplus\N} \ar[r,"f"] &\Z^{\oplus\N} \ar[r,"g"] &\Z \ar[r] &0 \\
      &(b) &0 \ar[r] &\Z^{\oplus\N} \ar[r,"f"] &\Z^{\oplus\N} \ar[r,"g"] &\Z^{\oplus\N} \ar[r] &0
    \end{tikzcd}
  \]
\end{problem}
\begin{solution}\ 
  \begin{enumerate}[label=(\alph*)]
    \item Let $f(n_1,n_2,\dots)=(0,n_1,n_2,\dots)$ and $g(n_1,n_2,\dots)=n_1$.
    \item Let $f(n_1,n_2,\dots)=(n_1,0,n_2,0,n_3,\dots)$ and $g(n_1,b_2,\dots)=(n_2,n_4,n_6,\dots)$.
  \end{enumerate}
\end{solution}

\begin{problem}{7.5}
  Assume that the complex
  \[
    \begin{tikzcd}
      &\cdots \ar[r] &L \ar[r,"f"] &M \ar[r,"g"] &N \ar[r] &\cdots
    \end{tikzcd}
  \]
  is exact and that $L$ and $N$ are Noetherian.
  Prove that $M$ is Noetherian.
\end{problem}
\begin{proof}
  First, nt hat since $L$ is Noetherian, we have that $\im f$ is Noetherian, since it is the 
  homomorphic image of a Noetherian module.
  Next, note hat $M/\im f=M/\ker g$ is a submodule of $N$, and hence is also Noetherian. \\
  \indent Therefore, since $\im f$ and $M/\im f$ are Noetherian, we have that $M$ is Noetherian,
  as desired.
\end{proof}

\begin{problem}{7.7}
  Let
  \[
    \begin{tikzcd}
      &0 \ar[r] &M \ar[r] &N \ar[r] &P \ar[r] &0
    \end{tikzcd}
  \]
  be a short exact equence of $R$-modules, and let $L$ be an $R$-module.
  \begin{enumerate}[label=(\roman*)]
    \item Prove that athere is an exact sequence \\
      \[
        \begin{tikzcd}
          &0 \ar[r] &\Homod{R}{P,L} \ar[r] &\Homod{R}{N, L} \ar[r] &\Homod{R}{M, L}
        \end{tikzcd}
      \]

    \item Let $M$ be  cyclic $R$-module, so that $M\cong R/I$ for a (left-)ideal $I$, and let
      $N$ be another $R$-module.
      Prove that $\Homod{R}{M, N}\cong \{n\in N | (\forall a\in I),an=0\}$.
      For $a,b\in\Z$, prove that $\Homc{Ab}{\Z/a\Z,\Z/b\Z}\cong\Z/\gcd(a,b)\Z$.

    \item Construct an example showing that the rightmost homomorphism in (i) need not be onto.
    \item Show that if the original sequence splits, then the rightmost homomorphism in (i)
      \textit{is} onto.
  \end{enumerate}
\end{problem}
\end{document}
