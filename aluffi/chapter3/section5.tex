%&pdflatex 
\documentclass[12pt]{article} 
\usepackage[margin=1in]{geometry} 
\usepackage{amsmath,amsthm,amssymb,amsfonts,tikz-cd, mathrsfs, enumitem} 

\newenvironment{problem}[2][Problem]{\begin{trivlist}
\item[\hskip \labelsep {\bfseries #1}\hskip \labelsep {\bfseries #2.}]}{\end{trivlist}}

\newenvironment{proposition}[1][Proposition]{\begin{trivlist}
\item[\hskip \labelsep {\bfseries #1.}]}{\end{trivlist}}

\newenvironment{definition}[1][Definition]{\begin{trivlist}
\item[\hskip \labelsep {\bfseries #1.}]}{\end{trivlist}}

\newcommand{\til}{\char`\~}
\newcommand{\bs}{\textbackslash} 

\newcommand{\catname}[1]{\normalfont\textbf{#1}}
\newcommand{\catsup}[2]{\normalfont\textbf{#1}^{#2}}
\newcommand{\catsub}[2]{\normalfont\textbf{#1}_{#2}}

\newcommand{\Hom}{\text{Hom}}
\newcommand{\Homc}[2]{\Hom_{\catname{#1}}(#2)}
\newcommand{\Homod}[2]{\Hom_{#1-\catname{Mod}}(#2)}

\newcommand{\Obj}{\text{Obj}}
\newcommand{\Objc}[1]{\text{Obj}(\catname{#1})}
\newcommand{\Aut}{\text{Aut}}
\newcommand{\End}{\text{End}}

\newcommand{\id}{\text{id}}
\newcommand{\im}{\text{im }}

\newcommand{\lcm}[1]{\text{lcm}(#1)}

\newenvironment{solution}
  {\renewcommand\qedsymbol{$\blacksquare$}\begin{proof}[Solution]}
{\end{proof}}

\newenvironment{sproof}{
  \renewcommand\qedsymbol{$\square$}
  \begin{proof}
  }{
  \end{proof}
}

\newtheorem{lemma}{Lemma}

\theoremstyle{remark}
\newtheorem*{exmp}{Example}
\newtheorem*{cexmp}{Counterexample}


\begin{document}

\title{Algebra: Chapter 0 Exercises\\ \large Chapter 3, Section 5\\ 
\large Modules over a ring}
\author{David Melendez}
\maketitle

\begin{problem}{5.5}
  Let $R$ be a commutative ring, viewed as an $R$-module over itself,
  and let $M$ be an $R$-module.
  Prove that $\Homod{R}{R,M}\cong M$ as $R$-modules.
\end{problem}
\begin{solution}
  Let $\varphi : M\to \Homod{R}{R,M}$ be the function defined by $$\varphi(m)(r)=rm.$$
  Then note that
  \begin{align*}
    \varphi(m+n)(r) &= r(m+n) \\
    &= rm + rn \\
    &= \varphi(m)(r) + \varphi(n)(r) \\
    &= (\varphi(m)+\varphi(n))(r),
  \end{align*}
  and 
  \begin{align*}
    \varphi(sm)(r) &= r(sm) \\
    &= (rs)m \\
    &= (sr)m \\
    &= s(rm) \\
    &= s\varphi(m)(r) \\
    &= (s\varphi(m))(r),
  \end{align*}
  and so $\varphi$ is an $R-\catname{Mod}$ homomorphism. \\
  Additionally, we have:
  \begin{align*}
    \varphi(m)(r+s) &= (r+s)m \\
    &= rm+sm \\
    &= \varphi(m)(r) + \varphi(n)(r)
  \end{align*}
  and 
  \begin{align*}
    \varphi(m)(rs) &= (rs)m \\
    &= r(sm) \\
    &= r\varphi(m)(s),
  \end{align*}
  and so $\varphi(m)$ is an $R-\catname{Mod}$ homomorphism for all $m\in M$. \\
  \indent Now, to prove that $\varphi$ is injective, note that if $\varphi(m) = 0$,
  then $m = 1_Rm = \varphi(m)(1) = 0$, and so $\varphi$ is injective.
  For surjectivity, we need the following insight: 
  For all $m\in M$ and $r\in R$, we have
  \begin{align*}
    \varphi(m)(r) &= \varphi(m)(r\cdot1_R) \\
    &=r\varphi(m)(1_R),
  \end{align*}
  and so if $\psi\in\Homod{R}{R,M}$, then we have, for all $r\in R$,
  \begin{align*}
    \psi(r) &= r\psi(1_R)\\
    &= \varphi(\psi(1_R)(r);
  \end{align*}
  thus, $\psi$ is in the image of $\varphi$ and $\varphi$ is surjective.
  Therefore, $\psi$ is an isomorphism and the modules are isomorphic as desired.
\end{solution}

\begin{problem}{5.6}
  Let $G$ be an abelian group.
  Prove that if $G$ has a structure of $\mathbb{Q}$-vector space, then it has only one such structure.
  (Hint: First prove that every element of $G$ has necessarily infinite order.
  Alternative hint: The unique ring homomorphism $\mathbb{Z}\to\mathbb{Q}$ is an epimorphism.)
\end{problem}
\begin{solution}
  Let $G$ be an abelian group.
  A $\mathbb{Q}$-vector space structure on $G$ is precisely a ring homomorphism
  $\sigma:G\to\Homc{Ab}{G}$.
  Let $\sigma_1,\sigma_2$, then, be two of these ring homomorphisms.
  Note that $\sigma_1$ and $\sigma_2$ agree on the integers, as if we view $\mathbb{Q}$ and
  $\Homc{Ab}{G}$ as $\mathbb{Z}$-modules, we then have, for all $n\in\mathbb{Z}$,
  \begin{align*}
    \varphi_1(n) &= \varphi_1(n\cdot1) \\
    &= n\cdot\varphi_1(1)\\
    &= n\cdot\id\\
    &= n\cdot\varphi_2(1)\\
    &= \varphi_2(n\cdot1) \\
    &= \varphi_2(n).
  \end{align*}
  Thus, if $\iota:\mathbb{Z}\to\mathbb{Q}$ is the unique ring homomorphism $\mathbb{Z}\to\mathbb{Q}$,
  i.e. the inclusion, we have $\sigma_1\iota = \sigma_2\iota$.
  Since $\iota$ is a ring epimorphism, this then implies that $\sigma_1=\sigma_2$, and so
  there is only one $\mathbb{Q}$-vector space structure on $\mathbb{G}$, as desired.
\end{solution}

\newpage

\begin{problem}{5.7}
  Let $K$ be a field, and let $k\subseteq K$ be a subfield of $K$.
  Show that $K$ is a vector space over $k$ (and in fact a $k$-algebra) in a natural way.
  In this situation, we say that $K$ is an \textit{extension} of $k$.
\end{problem}
\begin{solution}
  Note that the inclusion $\sigma:k\to\Homc{Ab}{K}$ is a ring homomorphism, and thus a natural
  $k$-vector space structure on $K$.
  This $\sigma$ also gives us a $k$-algebra structure on $K$ since the center of $K$ is $K$
  itself, and so $\im\sigma\subseteq Z(K)$.\\
  \indent More explicitly, the "scalar" multiplication $\kappa x$ for $\kappa\in k$ and $x\in K$
  is just multiplication within the field $K$, and the $k$-algebra structure on $K$
  also consists of multiplication as defined in the field $K$.
\end{solution}

\begin{problem}{5.8}
  What is the initial object of the category $R\catname{-Alg}$?
\end{problem}<++>

\end{document}
