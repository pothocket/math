%&pdflatex 
\documentclass[12pt]{article} 
\usepackage[margin=1in]{geometry} 
\usepackage{amsmath,amsthm,amssymb,amsfonts,tikz-cd} 

\newenvironment{problem}[2][Problem]{\begin{trivlist}
\item[\hskip \labelsep {\bfseries #1}\hskip \labelsep {\bfseries #2.}]}{\end{trivlist}}

\newenvironment{proposition}[1][Proposition]{\begin{trivlist}
\item[\hskip \labelsep {\bfseries #1.}]}{\end{trivlist}}

\newenvironment{definition}[1][Definition]{\begin{trivlist}
\item[\hskip \labelsep {\bfseries #1.}]}{\end{trivlist}}

\newcommand{\til}{\char`\~}
\newcommand{\bs}{\textbackslash} 

\newcommand{\catname}[1]{\normalfont\textbf{#1}}
\newcommand{\catsup}[2]{\normalfont\textbf{#1}^{#2}}
\newcommand{\catsub}[2]{\normalfont\textbf{#1}_{#2}}


\newcommand{\Hom}{\text{Hom}}
\newcommand{\Homc}[2]{\Hom_{\catname{#1}}(#2)}

\newcommand{\Obj}{\text{Obj}}
\newcommand{\Objc}[1]{\text{Obj}(\catname{#1})}
\newcommand{\Aut}{\text{Aut}}
\newcommand{\End}{\text{End}}

\newcommand{\id}{\text{id}}

\newenvironment{solution}
  {\renewcommand\qedsymbol{$\blacksquare$}\begin{proof}[Solution]}
{\end{proof}}

\newenvironment{sproof}{
  \renewcommand\qedsymbol{$\square$}
  \begin{proof}
  }{
  \end{proof}
}

\begin{document}

\title{Algebra: Chapter 0 Exercises\\ \large Chapter 2, Section 1}
\author{David Melendez}
\maketitle

\begin{problem}{1.3}
  Prove that $(gh)^{-1} = h^{-1}g^{-1}$ for all elements $g,h$ of a group $G$.
\end{problem}
\begin{proof}
  We have (by associativity) that $(gh)(g^{-1}h^{-1}) = e$. But $(gh)(gh)^{-1} = e$, so by cancellation 
  $(gh)^{-1}=h^{-1}g^{-1}$.
\end{proof}

\begin{problem}{1.4}
  Suupose that $g^2=e$ for all elements $g$ of a group $G$; prove that $G$ is commutative.
\end{problem}
\begin{proof}
  $gh = ghe = gh(hg)^2 = ghhghg = gghg = hg$
\end{proof}

\begin{problem}{1.5}
  Prove that ever row and every column of the 'multiplication table' of a group contains all elemtns of the group exactly once.
\end{problem}
\begin{solution}
  That every row of a group $G$'s multiplication table is 'sudoku complete' (if you will) is equivalent to the following:\\
  \begin{proposition}
    For every $g, h \in G$ $g\neq h$, there exists a unique $x \in G$ such that $gx=h$.
  \end{proposition}
  \begin{sproof}
    Putting $x=g^{-1}h$, we have $gx = gg^{-1}h = h$. 
    If any $y$ satisfies this property, we have 
    \begin{align*}
    gx = h = gy &\implies gx = gy\\
    &\implies g^{-1}x = g^{-1}y\\
    &\implies x = y
    \end{align*}
  \end{sproof}
  The proof for columns is entirely analogous.
\end{solution}

\newpage

\begin{problem}{1.6}
  Prove that there is only \textit{one} possible multiplication table for $G$ if $G$ has exactly $1, 2,$ or $3$ elements.
  Analyze the possible multiplication tables for groups with exactly 4 elements, and show that there are \textit{two} distinct tables, up to reordering the elements of $G$.
\end{problem}
\begin{solution} .\\
  \begin{enumerate}
    \item The proof for $|G|=1$ is trivial.
    \item For $|G|=2$ and $e, a\in G$, we have $ee = e$, $ea =a$, and $ae = e$.
      Since each element of a group must have an inverse, we must also have 
      $a = a^{-1}$ (since $e \neq a^{-1}$), so $a^2 = e$.
    \item For $|G| = 3$, consider the table:\\
      \[\begin{tabular}{c || l|c|r}
        $\cdot$ & e & a & b \\ \hline\hline
        e & e & a & b \\ \hline
        a & a & ? & ? \\ \hline
        b & b & ? & ? \\ 
      \end{tabular}\]
      We can complete the table like a sudoku puzzle using problem 1.5. 
      Since $ea = a$, we cannot have $a^2 = a$. 
      Since $eb=b$, we can't have $a^2=e$ since that would force $ab=b$. 
      Hence, $a^2=b$.
      \[\begin{tabular}{c || l|c|r}
        $\cdot$ & e & a & b \\ \hline\hline
        e & e & a & b \\ \hline
        a & a & b & ? \\ \hline
        b & b & ? & ? \\ 
      \end{tabular}\]
      The rest of the table is forced by problem 1.5.
      \[\begin{tabular}{c || l|c|r}
        $\cdot$ & e & a & b \\ \hline\hline
        e & e & a & b \\ \hline
        a & a & b & e \\ \hline
        b & b & e & a \\ 
      \end{tabular}\]
  \end{enumerate}
\end{solution}

\end{document}
