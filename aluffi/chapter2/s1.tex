%&pdflatex 
\documentclass[12pt]{article} 
\usepackage[margin=1in]{geometry} 
\usepackage{amsmath,amsthm,amssymb,amsfonts,tikz-cd} 

\newenvironment{problem}[2][Problem]{\begin{trivlist}
\item[\hskip \labelsep {\bfseries #1}\hskip \labelsep {\bfseries #2.}]}{\end{trivlist}}

\newenvironment{proposition}[1][Proposition]{\begin{trivlist}
\item[\hskip \labelsep {\bfseries #1.}]}{\end{trivlist}}

\newenvironment{definition}[1][Definition]{\begin{trivlist}
\item[\hskip \labelsep {\bfseries #1.}]}{\end{trivlist}}

\newcommand{\til}{\char`\~}
\newcommand{\bs}{\textbackslash} 

\newcommand{\catname}[1]{\normalfont\textbf{#1}}
\newcommand{\catsup}[2]{\normalfont\textbf{#1}^{#2}}
\newcommand{\catsub}[2]{\normalfont\textbf{#1}_{#2}}

\newcommand{\Hom}{\text{Hom}}
\newcommand{\Homc}[2]{\Hom_{\catname{#1}}(#2)}

\newcommand{\Obj}{\text{Obj}}
\newcommand{\Objc}[1]{\text{Obj}(\catname{#1})}
\newcommand{\Aut}{\text{Aut}}
\newcommand{\End}{\text{End}}

\newcommand{\id}{\text{id}}

\newcommand{\lcm}[1]{\text{lcm}(#1)}

\newenvironment{solution}
  {\renewcommand\qedsymbol{$\blacksquare$}\begin{proof}[Solution]}
{\end{proof}}

\newenvironment{sproof}{
  \renewcommand\qedsymbol{$\square$}
  \begin{proof}
  }{
  \end{proof}
}

\begin{document}

\title{Algebra: Chapter 0 Exercises\\ \large Chapter 2, Section 1}
\author{David Melendez}
\maketitle

\begin{problem}{1.3}
  Prove that $(gh)^{-1} = h^{-1}g^{-1}$ for all elements $g,h$ of a group $G$.
\end{problem}
\begin{proof}
  We have (by associativity) that $(gh)(g^{-1}h^{-1}) = e$. But $(gh)(gh)^{-1} = e$, so by cancellation 
  $(gh)^{-1}=h^{-1}g^{-1}$.
\end{proof}

\begin{problem}{1.4}
  Suupose that $g^2=e$ for all elements $g$ of a group $G$; prove that $G$ is commutative.
\end{problem}
\begin{proof}
  $gh = ghe = gh(hg)^2 = ghhghg = gghg = hg$
\end{proof}

\begin{problem}{1.5}
  Prove that ever row and every column of the 'multiplication table' of a group contains all elemtns of the group exactly once.
\end{problem}
\begin{solution}
  That every row of a group $G$'s multiplication table is 'sudoku complete' (if you will) is equivalent to the following:\\
  \begin{proposition}
    For every $g, h \in G$ $g\neq h$, there exists a unique $x \in G$ such that $gx=h$.
  \end{proposition}
  \begin{sproof}
    Putting $x=g^{-1}h$, we have $gx = gg^{-1}h = h$. 
    If any $y$ satisfies this property, we have 
    \begin{align*}
    gx = h = gy &\implies gx = gy\\
    &\implies g^{-1}x = g^{-1}y\\
    &\implies x = y
    \end{align*}
  \end{sproof}
  The proof for columns is entirely analogous.
\end{solution}

\newpage

\begin{problem}{1.6}
  Prove that there is only \textit{one} possible multiplication table for $G$ if $G$ has exactly $1, 2,$ or $3$ elements.
  Analyze the possible multiplication tables for groups with exactly 4 elements, and show that there are \textit{two} distinct tables, up to reordering the elements of $G$.
\end{problem}
\begin{solution} .\\
  \begin{enumerate}
    \item The proof for $|G|=1$ is trivial.
    \item For $|G|=2$ and $e, a\in G$, we have $ee = e$, $ea =a$, and $ae = e$.
      Since each element of a group must have an inverse, we must also have 
      $a = a^{-1}$ (since $e \neq a^{-1}$), so $a^2 = e$.
    \item For $|G| = 3$, consider the table:\\
      \[\begin{tabular}{c || l|c|r}
        $\cdot$ & e & a & b \\ \hline\hline
        e & e & a & b \\ \hline
        a & a & ? & ? \\ \hline
        b & b & ? & ? \\ 
      \end{tabular}\]
      We can complete the table like a sudoku puzzle using problem 1.5. 
      Since $ea = a$, we cannot have $a^2 = a$. 
      Since $eb=b$, we can't have $a^2=e$ since that would force $ab=b$. 
      Hence, $a^2=b$.
      \[\begin{tabular}{c || l|c|r}
        $\cdot$ & e & a & b \\ \hline\hline
        e & e & a & b \\ \hline
        a & a & b & ? \\ \hline
        b & b & ? & ? \\ 
      \end{tabular}\]
      The rest of the table is forced by problem 1.5.
      \[\begin{tabular}{c || l|c|r}
        $\cdot$ & e & a & b \\ \hline\hline
        e & e & a & b \\ \hline
        a & a & b & e \\ \hline
        b & b & e & a \\ 
      \end{tabular}\]
    \item Consider the table for $|G|=4$:
      \[\begin{tabular}{c || c|c|c|c}
            $\cdot$ & e & a & b & c\\ \hline\hline
            e & e & a & b & c \\ \hline
            a & a & ? & ? & ? \\ \hline
            b & b & ? & ? & ? \\ \hline
            c & c & ? & ? & ? 
        \end{tabular}\]
        For this table we have two distinct cases: 
        where $a^2=e$ and where $a^2=b$. 
        The case where $a^2=c$ is the same as where $a^2=b$ up to reordering.\\
        First consider $a^2=e:$
        \[\begin{tabular}{c || c|c|c|c}
            $\cdot$ & e & a & b & c\\ \hline\hline
            e & e & a & b & c \\ \hline
            a & a & e & ? & ? \\ \hline
            b & b & ? & ? & ? \\ \hline
            c & c & ? & ? & ? 
        \end{tabular}\]
        We can complete the rest of the table using problem 1.5:
        \[\begin{tabular}{c || c|c|c|c}
            $\cdot$ & e & a & b & c\\ \hline\hline
            e & e & a & b & c \\ \hline
            a & a & e & c & b \\ \hline
            b & b & c & e & a \\ \hline
            c & c & b & a & e 
        \end{tabular}\]
        Notice that we can also fill the table out this way:
        \[\begin{tabular}{c || c|c|c|c}
            $\cdot$ & e & a & b & c\\ \hline\hline
            e & e & a & b & c \\ \hline
            a & a & e & c & b \\ \hline
            b & b & c & a & e \\ \hline
            c & c & b & e & a 
        \end{tabular}\]
        As it turns out, this is equivalent to the case where $a^2=b$,
        but with $b$ and $a$ switched (that is, up to reordering):
        \[\begin{tabular}{c || c|c|c|c}
            $\cdot$ & e & a & b & c\\ \hline\hline
            e & e & a & b & c \\ \hline
            a & a & b & c & e \\ \hline
            b & b & c & e & a \\ \hline
            c & c & e & a & b 
        \end{tabular}\]
  \end{enumerate}
\end{solution}
\begin{problem}{1.8}
  Let $G$ be a finite abelian group, with exactly one element $f$ of order 2.
  Prove that $\prod_{g\in G} g = f$.
\end{problem}
\begin{proof}
  Since every element of $G$ has an inverse and the order of composition
  doesn't matter (since $G$ is abelian), we have, with each $g_j \in G$,
  \begin{align*}
    \prod_{g\in G} g &= e\cdot f\cdot 
    (g_1\cdot g_1^{-1})\cdot (g_2\cdot g_2^{-1})\cdots(g_n\cdot g_n^{-1})\\
    &= e\cdot f\cdot (e)(e)\cdots(e)\\
    &= f
  \end{align*}
  where $n = |G| - 2$
\end{proof}
\begin{problem}{1.9}
  Let $G$ be a finite group of order $n$ and let $m$ be the number of elements
  $g\in G$ of order exactly 2. Prove that $n-m$ is odd.
\end{problem}
\begin{proof}
  We can divide the elements of $G$ into three classes: elements of order 1,
  elements of order 2, and elements of order greater than 2:
  \begin{enumerate}
    \item The only group element of order 1 is the identity $e$.
    \item We have assumed that there are $m$ elements of order 2.
    \item Note that for every element $g$ with $|g|>2$, we also have a distinct
      $g^{-1}$, meaning that there are an even number of these elements.
  \end{enumerate}
  Taking these three classes into consideration, we have 
  $|G| = n = 1 + m + 2j$ where $j$ is a nonnegative integer. Hence
  $n-m=2j+1$ as desired.
\end{proof}
\begin{problem}{1.10}
  Suppose the order of $g$ is odd. What can you say about the order of $g^2$?
\end{problem}
\begin{solution}
  $|g^2| = \frac{\lcm{2,|g|}}{2}=|g|$
\end{solution}
\begin{problem}{1.11}
  Prove that for all $g,h$ in a group $G$, $|gh| = |hg|$.
\end{problem}
\begin{solution}
  Since $gh = h(gh)h^{-1}$, we just need to prove that $|aga^{-1}|=|g|$
  for $a,g\in G$ (as is given in the problem as a hint).
  \begin{sproof}
    Note that, with $n=|g|$,
    \begin{align*}
      (aga^{-1})^n &= ag(a^{-1}a)g(a^{-1}a)\cdots ga^{-1} \\
      &= a(g^n)a^{-1} \\
      &= aa^{-1} \\
      &= e
    \end{align*}
    Since $n$ is the smallest positive integer that makes the $g$'s 
    vanish like this, we have $|aga^{-1}| = n = |g|$. 
  \end{sproof}
\end{solution}
\begin{problem}{1.12}
  In the group of $2\times 2$ matrices, consider
  \begin{equation*}
     g = \begin{pmatrix} 0 & -1\\ 1 & 0 \end{pmatrix} \qquad, \qquad
     h = \begin{pmatrix} 0 & 1\\ -1 & -1 \end{pmatrix} 
  \end{equation*}
  Verify that $|g|=4$, $|h|=3$, and $|gh|=\infty$.
  \begin{solution}
    The first two are a trivial application of matrix multiplication.\\
    Consider the product 
    $gh = \begin{pmatrix}1&1\\0&1\end{pmatrix}$. We will work with its
    corresponding linear map \\
    \begin{proposition}
      $|gh| = \infty$
    \end{proposition}
    \begin{proof}
      Consider the corresponding linear map $T \in \mathcal{L}(\mathbb{R}^2)$.
      Let ${x,y}$ be a basis of $\mathbb{R}^2$. We then have, from the matrix,
      that
      \begin{align*}
        Tx &= x \\
        Ty &= x + y
      \end{align*}
      (This is enough to define $T$ since T is linear).\\
      It then follows that $T^n$ is as follows:
      \begin{align*}
        T^nx &= x \\
        T^ny &= nx + y
      \end{align*}
      Finding the order of $gh$ then boils down to solving 
      $T^n=I$ for $n$. Since $T^nx=x$, we just need to solve 
      $T^ny=y$.
      \begin{alignat*}{2}
        & & T^ny &= y \\
        &\implies &nx + y &= y \\
        &\implies &nx &= 0 \\
        &\implies &n &= 0 
      \end{alignat*}
    Since no integer other than 0 gives $T^n=(gh)^n=e$, we have $|gh|=\infty$.
    \end{proof}
  \end{solution}
\end{problem}
\begin{problem}{1.13}
  Give an example showing that $|gh|$ is not necessarily $\lcm{|g|,|h|}$
  even if $g$ and $h$ commute.
\end{problem}
\begin{solution}
  Let $h=g^{-1}$ and $g\neq e$.
  Then clearly $g$ and $h$ commute, but\\ $\lcm{|g|,|h|}=|g|\neq|gh|=1$.
\end{solution}
\begin{problem}{1.14}
  As a counterpoint to Exercise 1.13, prove that if $g$ and $h$ commute, and $\gcd(|g|,|h|)=1$, then
  $|gh|=|g||h|$. (Hint: let $N=|gh|$; then $g^N = (h^{-1})^N$. What can you say about this element?)
\end{problem}
\begin{proof}
  We will prove, with $|gh|=N, |g|=m, |h|=n$, that $N|mn$ and $mn|N$.\\
  Note that since $g$ and $h$ commute, $(gh)^{mn} = g^{nm}h^{mn} = e$. Hence, $N | mn$.\\
  Now, consider $(gh)^N$. Note that since $(gh)^N=e$, we have $g^N=(h^N)^{-1}=(h^{-1})^N$.
  Then, since $(g^N)^n\left( (h^{-1})^N \right)^n = h^{-Nn}=e$, we have $m|Nn$. Similarly, $n|Nm$.
  It then follows, since $\gcd(m, n) = 1$, that $m|N$ and $n|N$. 
  Finally, since $m$ and $n$ are coprime, it follows from this that $N$ must be a product of the prime factors
  of $n$, the prime factors of $m$, and some other positive integer, showing that $mn | N$, as desired.
\end{proof}
\end{document}
