%&pdflatex 
\documentclass[12pt]{article} 
\usepackage[margin=1in]{geometry} 
\usepackage{amsmath,amsthm,amssymb,amsfonts,tikz-cd, mathrsfs} 

\newenvironment{problem}[2][Problem]{\begin{trivlist}
\item[\hskip \labelsep {\bfseries #1}\hskip \labelsep {\bfseries #2.}]}{\end{trivlist}}

\newenvironment{proposition}[1][Proposition]{\begin{trivlist}
\item[\hskip \labelsep {\bfseries #1.}]}{\end{trivlist}}

\newenvironment{definition}[1][Definition]{\begin{trivlist}
\item[\hskip \labelsep {\bfseries #1.}]}{\end{trivlist}}

\newcommand{\til}{\char`\~}
\newcommand{\bs}{\textbackslash} 

\newcommand{\catname}[1]{\normalfont\textbf{#1}}
\newcommand{\catsup}[2]{\normalfont\textbf{#1}^{#2}}
\newcommand{\catsub}[2]{\normalfont\textbf{#1}_{#2}}

\newcommand{\Hom}{\text{Hom}}
\newcommand{\Homc}[2]{\Hom_{\catname{#1}}(#2)}

\newcommand{\Obj}{\text{Obj}}
\newcommand{\Objc}[1]{\text{Obj}(\catname{#1})}
\newcommand{\Aut}{\text{Aut}}
\newcommand{\End}{\text{End}}

\newcommand{\id}{\text{id}}

\newcommand{\lcm}[1]{\text{lcm}(#1)}

\newenvironment{solution}
  {\renewcommand\qedsymbol{$\blacksquare$}\begin{proof}[Solution]}
{\end{proof}}

\newenvironment{sproof}{
  \renewcommand\qedsymbol{$\square$}
  \begin{proof}
  }{
  \end{proof}
}

\begin{document}

\title{Algebra: Chapter 0 Exercises\\ \large Chapter 2, Section 5}
\author{David Melendez}
\maketitle

\begin{problem}{1}
  Let $\mathscr{F}^A$ be a category whose objects $(f, G)$ are pairs consisting of a group $G$ along with
  a function $f : A \to G$, and whose morphisms $(f_1, G_1)\to (f_2, G_2)$ are morphisms 
  $\varphi : G_1 \to G_2$
  such that the following diagram commutes:
  \[\begin{tikzcd}
    & G_1\ar{r}{\varphi} & G_2 \\
    & A \ar{u}{f_1} \ar{ur}[swap]{f_2}
    \end{tikzcd}\]
    Does this category have final objects?
\end{problem}
\begin{solution}
  Yes. If $T$ is the trivial group and $\kappa_e$ is the trivial homomorphism 
  (i.e. sends all elements to the identity in the destination group),
  then $(\varphi_e,T)$ is final in this group. 
  \begin{sproof}
    Let $(f, G)\in \Obj(\mathscr{F}^A)$, and suppose $\varphi$ is such that the following diagram commutes:
   \[\begin{tikzcd}
    & G\ar{r}{\varphi} & T \\
    & A \ar{u}{f} \ar{ur}[swap]{\kappa_e}
    \end{tikzcd}\]
    Clearly the only morphism $G\to T$ is the trivial morphism (i.e. $\varphi(g) = e$ for all $g\in G$),
    and this diagram does commute for such a $\varphi$ since $\varphi(f(a)) = e$.
    Hence $(\kappa_e,T)$ is final in $\mathscr{F}^A$.
  \end{sproof}
\end{solution}

\begin{problem}{5.2}
  Explain why $(\kappa_e,T)$ is not initial in $\mathcal{F}^A$ (unless $a=\emptyset$).
\end{problem}
\begin{solution}
  Let $(f, G)\in \Obj(\mathscr{F}^A)$, and consider the following diagram: 
   \[\begin{tikzcd}
       & T\ar[dashed]{r}{\varphi} & G \\
    & A \ar{u}{\kappa_e} \ar{ur}[swap]{f}
    \end{tikzcd}\]
    Since the only group homomorphism $T\to G$ is the trivial one, we must have 
    $f(a) = e_G$ for all $a\in A$ for this diagram to commute.
    This is the only case for all $(f,G)$ if $A=\emptyset$.
\end{solution}
\begin{problem}{5.3}
  Use the universal property of free groups to prove that the map $j : f \to F(A)$ is injective,
  for all sets $A$.
\end{problem}
\begin{solution}
  Recall that a free group along with inclusion $\iota, F(A)$ is initial in the category $\mathscr{F}^A$.
  For any $a,b\in A$ with $a\neq b$, consider the following diagram:
   \[\begin{tikzcd}
       & F(A)\ar[dashed]{r}{\varphi} & C_2 \\
       & A \ar{u}{\iota} \ar{ur}[swap]{f}
    \end{tikzcd}\]
    Define $f$ by $f(a) = e$, and $f(x) = g$ if $x\neq a$ 
    (where $e$ and $g$ are the identity and generator in $C_2$, respectively.
      We then have $f(a)\neq f(b)$, so $(\varphi\circ\iota)(a) \neq (\varphi\circ\iota)(b)$,
      and hence \\$\iota(a)\neq\iota(b)$, making $\iota$ injective.
\end{solution}
\begin{problem}{5.5}
  Verify explicitly that $H^{\oplus A}$ is a group.
\end{problem}
\begin{solution}
  Let $H$ be a group. 
  If $\alpha$ is a function from $A$ to $H$, define 
  $$\ker' \alpha := \{a\in A \ |\ \alpha(a) \neq e_H\}.$$
  We then define $H^{\oplus A}$ by
  \begin{equation*}
    H^{\oplus A} := \{\alpha : A \to H\ |\ \ker' \alpha \text{ is finite}\}
  \end{equation*}
  To define the group operation on $H^{\oplus A}$, let $\alpha_1, \alpha_2\in H^{\oplus A}$.
  We then define
  \begin{equation*}
    (\alpha_1\cdot\alpha_2)(a) = \alpha_1(a) \cdot \alpha_2(a)
  \end{equation*}
  This operation "inherits" associativity and inverses from the group $H$.
  Furthermore, this group is closed under inverses, since the inverse of the identity is the identity.
  To prove closure, we must show that
  \begin{equation*}
    \ker' (\alpha_1\cdot\alpha_2) \text{ is finite.}
  \end{equation*}
  Some examination shows us that
  \begin{equation*}
    \ker' (\alpha_1\cdot\alpha_2) = (\ker` \alpha_1 \cup \ker` \alpha_2) \setminus 
    \{a\in A\ |\ \alpha_1(a) = \alpha_2(a)^{-1}\},
  \end{equation*}
  which is clearly finite; hence $(\alpha_1\cdot\alpha_2)$ is in $H^{\oplus A}$.
\end{solution}
\begin{problem}{5.6}
  Prove that the group $F(\{x,y\})$ is a coproduct $\mathbb{Z}*\mathbb{Z}$ of 
  $\mathbb{Z}$ by itself in the category $\catname{Grp}$.
\end{problem}
\begin{proof}
  Let $S$ be the set $\{x,y\}$, let $G$ be a group,
  and suppose we have group homomorphisms as in the following diagram:
  \[ \begin{tikzcd}
    & \mathbb{Z} 
    \ar{r}{\epsilon_x} 
    \ar{dr}[swap]{f_x}
    & F(S)
    \ar[dashed]{d}{\varphi}
    & \mathbb{Z}
    \ar{l}[swap]{\epsilon_y}
    \ar{dl}{f_y}
    \\
    & & G &
  \end{tikzcd} \]
  where $\epsilon_x$ and $\epsilon_y$ are the "exponential maps" $\epsilon_g(n) = g^n$.
  If a $\varphi$ exists that makes this diagram commute, note that we must have
  \begin{align*}
    \varphi(x) &= \varphi(\epsilon_x(1)) \\
    &= f_x(1)
  \end{align*}
  and similarly
  \begin{align*}
    \varphi(y) &= \varphi(\epsilon_y(1)) \\
    &= f_y(1)
  \end{align*}
  Keeping this in mind, we turn our attention to the following diagram:
  \[\begin{tikzcd}
      & F(S) \ar[dashed]{r}{\varphi} & G \\
      & S \ar{u}{\iota} \ar{ur}[swap]{f} &
  \end{tikzcd}\]
  where $\iota:S\to F(S)$ is the usual inclusion and $f:S\to G$ is a set function mapping
  $x\mapsto f_1(1)$ and $y\mapsto f_2(1)$.\\
  The group homomorphism $\varphi$ making this diagram commute exists and is unique by the 
  universal property for free groups.
  Since this is the only group homomorphism satisfying those properties listed above required to 
  make the coproduct diagram commute, we simply must check that this homomorphism does indeed
  make that diagram commute.
  We have
  \begin{align*}
    \varphi(\epsilon_x(n)) &= \varphi(x^n) \\
    &= \varphi(x)^n \\
    &= f_x(1)^n \\
    &= f_x(n),
  \end{align*}
  and similarly for $f_y$. This completes the proof.
\end{proof}
\begin{problem}{5.10}
  Let $F=F^{ab}(A)$.
  \begin{enumerate}
    \item Define an equivalence relation $\sim$ on $F$ by setting 
      \begin{equation*}
        f'\sim f \Leftrightarrow (\exists g\in F):~~ f-f' = 2g.
      \end{equation*}
      Prove that $F/{\sim}$ is a finite set if and only if A is finite, and in that case 
      $|F/{\sim}| = 2^{|A|}$.
    \item Assume $F^{ab}(B) \cong F^{ab}(A)$. If $A$ is finite, prove that so is $B$, 
      and $A\cong B$ as sets.
  \end{enumerate}
\end{problem}
\begin{solution}
  \begin{enumerate}
    \item First suppose that $A$ is finite.
      We then know, by exercise 5.7 (which extends easily from 5.6), that we can think of elements of $F$ 
      as tuples of integers.
      Consider, then, $f, f'\in F$ with\\
      \begin{align*}
        f &= (a_1, \dots, a_n)\\
        f'&= (b_1,\dots, b_n)
      \end{align*}
      We then have $f\sim f'$ if and only if $b_j - a_j$ is even for each $j$, and hence if and only if
      each $b_j$ and $a_j$ have the same parity (i.e. are congruent modulo 2).
      There are, then, as many equivalence classes in $F/{\sim}$ as there are tuples 
      $(p_1, \dots, p_n)$ where each $p_j \in \mathbb{Z}/2\mathbb{Z}$, and hence 
      $|F/{\sim}| = |(Z/2\mathbb{Z})^n| = 2^n$ (which, of course also proves that $F/{\sim}$ is finite).\\
      Conversely, suppose that $A$ is infinite.
      We then have, for each distinct $x\in A$, a distinct equivalence class $[x]_{\sim}$
      (since $x\sim a$ iff $a = x^{n} a_1^{n_1} a_2^{n_2} \cdots$ with each $a_k\in A$ distinct from $x$
        and each other,
      each $n_k$ even, and $n$ odd); hence the cardinality of $F/{\sim}$ is at least the cardinality of $A$.
    \item 
  \end{enumerate}
\end{solution}
\end{document}
