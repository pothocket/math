%&pdflatex 
\documentclass[12pt]{article} 
\usepackage[margin=1in]{geometry} 
\usepackage{amsmath,amsthm,amssymb,amsfonts,tikz-cd, mathrsfs} 

\newenvironment{problem}[2][Problem]{\begin{trivlist}
\item[\hskip \labelsep {\bfseries #1}\hskip \labelsep {\bfseries #2.}]}{\end{trivlist}}

\newenvironment{proposition}[1][Proposition]{\begin{trivlist}
\item[\hskip \labelsep {\bfseries #1.}]}{\end{trivlist}}

\newenvironment{definition}[1][Definition]{\begin{trivlist}
\item[\hskip \labelsep {\bfseries #1.}]}{\end{trivlist}}

\newcommand{\til}{\char`\~}
\newcommand{\bs}{\textbackslash} 

\newcommand{\catname}[1]{\normalfont\textbf{#1}}
\newcommand{\catsup}[2]{\normalfont\textbf{#1}^{#2}}
\newcommand{\catsub}[2]{\normalfont\textbf{#1}_{#2}}

\newcommand{\Hom}{\text{Hom}}
\newcommand{\Homc}[2]{\Hom_{\catname{#1}}(#2)}

\newcommand{\Obj}{\text{Obj}}
\newcommand{\Objc}[1]{\text{Obj}(\catname{#1})}
\newcommand{\Aut}{\text{Aut}}
\newcommand{\End}{\text{End}}

\newcommand{\id}{\text{id}}

\newcommand{\lcm}[1]{\text{lcm}(#1)}

\newenvironment{solution}
  {\renewcommand\qedsymbol{$\blacksquare$}\begin{proof}[Solution]}
{\end{proof}}

\newenvironment{sproof}{
  \renewcommand\qedsymbol{$\square$}
  \begin{proof}
  }{
  \end{proof}
}

\begin{document}

\title{Algebra: Chapter 0 Exercises\\ \large Chapter 2, Section 5}
\author{David Melendez}
\maketitle

\begin{problem}{1}
  Let $\mathscr{F}^A$ be a category whose objects $(f, G)$ are pairs consisting of a group $G$ along with
  a function $f : A \to G$, and whose morphisms $(f_1, G_1)\to (f_2, G_2)$ are morphisms 
  $\varphi : G_1 \to G_2$
  such that the following diagram commutes:
  \[\begin{tikzcd}
    & G_1\ar{r}{\varphi} & G_2 \\
    & A \ar{u}{f_1} \ar{ur}[swap]{f_2}
    \end{tikzcd}\]
    Does this category have final objects?
\end{problem}
\begin{solution}
  Yes. If $T$ is the trivial group and $\kappa_e$ is the trivial homomorphism 
  (i.e. sends all elements to the identity in the destination group),
  then $(\varphi_e,T)$ is final in this group. 
  \begin{sproof}
    Let $(f, G)\in \Obj(\mathscr{F}^A)$, and suppose $\varphi$ is such that the following diagram commutes:
   \[\begin{tikzcd}
    & G\ar{r}{\varphi} & T \\
    & A \ar{u}{f} \ar{ur}[swap]{\kappa_e}
    \end{tikzcd}\]
    Clearly the only morphism $G\to T$ is the trivial morphism (i.e. $\varphi(g) = e$ for all $g\in G$),
    and this diagram does commute for such a $\varphi$ since $\varphi(f(a)) = e$.
    Hence $(\kappa_e,T)$ is final in $\mathscr{F}^A$.
  \end{sproof}
\end{solution}

\begin{problem}{5.2}
  Explain why $(\kappa_e,T)$ is not initial in $\mathcal{F}^A$ (unless $a=\emptyset$).
\end{problem}
\begin{solution}
  Let $(f, G)\in \Obj(\mathscr{F}^A)$, and consider the following diagram: 
   \[\begin{tikzcd}
       & T\ar[dashed]{r}{\varphi} & G \\
    & A \ar{u}{\kappa_e} \ar{ur}[swap]{f}
    \end{tikzcd}\]
    Since the only group homomorphism $T\to G$ is the trivial one, we must have 
    $f(a) = e_G$ for all $a\in A$ for this diagram to commute.
    This is the only case for all $(f,G)$ if $A=\emptyset$.
\end{solution}
\begin{problem}{5.3}
  Use the universal property of free groups to prove that the map $j : f \to F(A)$ is injective,
  for all sets $A$.
\end{problem}
\begin{solution}
  Recall that a free group along with inclusion $\iota, F(A)$ is initial in the category $\mathscr{F}^A$.
  For any $a,b\in A$ with $a\neq b$, consider the following diagram:
   \[\begin{tikzcd}
       & F(A)\ar[dashed]{r}{\varphi} & C_2 \\
       & A \ar{u}{\iota} \ar{ur}[swap]{f}
    \end{tikzcd}\]
    Define $f$ by $f(a) = e$, and $f(x) = g$ if $x\neq a$ 
    (where $e$ and $g$ are the identity and generator in $C_2$, respectively.
      We then have $f(a)\neq f(b)$, so $(\varphi\circ\iota)(a) \neq (\varphi\circ\iota)(b)$,
      and hence \\$\iota(a)\neq\iota(b)$, making $\iota$ injective.
\end{solution}
\begin{problem}{5.5}
  Verify explicitly that $H^{\oplus A}$ is a group.
\end{problem}
\begin{solution}
  Let $H$ be a group. 
  If $\alpha$ is a function from $A$ to $H$, define 
  $$\ker' \alpha := \{a\in A \ |\ \alpha(a) \neq e_H\}.$$
  We then define $H^{\oplus A}$ by
  \begin{equation*}
    H^{\oplus A} := \{\alpha : A \to H\ |\ \ker' \alpha \text{ is finite}\}
  \end{equation*}
  To define the group operation on $H^{\oplus A}$, let $\alpha_1, \alpha_2\in H^{\oplus A}$.
  We then define
  \begin{equation*}
    (\alpha_1\cdot\alpha_2)(a) = \alpha_1(a) \cdot \alpha_2(a)
  \end{equation*}
  This operation "inherits" associativity and inverses from the group $H$.
  Furthermore, this group is closed under inverses, since the inverse of the identity is the identity.
  To prove closure, we must show that
  \begin{equation*}
    \ker' (\alpha_1\cdot\alpha_2) \text{ is finite.}
  \end{equation*}
  Some examination shows us that
  \begin{equation*}
    \ker' (\alpha_1\cdot\alpha_2) = (\ker` \alpha_1 \cup \ker` \alpha_2) \setminus 
    \{a\in A\ |\ \alpha_1(a) = \alpha_2(a)^{-1}\},
  \end{equation*}
  which is clearly finite; hence $(\alpha_1\cdot\alpha_2)$ is in $H^{\oplus A}$.
\end{solution}
\end{document}
