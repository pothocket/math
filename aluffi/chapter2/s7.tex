%&pdflatex 
\documentclass[12pt]{article} 
\usepackage[margin=1in]{geometry} 
\usepackage{amsmath,amsthm,amssymb,amsfonts,tikz-cd, mathrsfs} 

\newenvironment{problem}[2][Problem]{\begin{trivlist}
\item[\hskip \labelsep {\bfseries #1}\hskip \labelsep {\bfseries #2.}]}{\end{trivlist}}

\newenvironment{proposition}[1][Proposition]{\begin{trivlist}
\item[\hskip \labelsep {\bfseries #1.}]}{\end{trivlist}}

\newenvironment{definition}[1][Definition]{\begin{trivlist}
\item[\hskip \labelsep {\bfseries #1.}]}{\end{trivlist}}

\newcommand{\til}{\char`\~}
\newcommand{\bs}{\textbackslash} 

\newcommand{\catname}[1]{\normalfont\textbf{#1}}
\newcommand{\catsup}[2]{\normalfont\textbf{#1}^{#2}}
\newcommand{\catsub}[2]{\normalfont\textbf{#1}_{#2}}

\newcommand{\Hom}{\text{Hom}}
\newcommand{\Homc}[2]{\Hom_{\catname{#1}}(#2)}

\newcommand{\Obj}{\text{Obj}}
\newcommand{\Objc}[1]{\text{Obj}(\catname{#1})}
\newcommand{\Aut}{\text{Aut}}
\newcommand{\End}{\text{End}}

\newcommand{\id}{\text{id}}

\newcommand{\lcm}[1]{\text{lcm}(#1)}

\newenvironment{solution}
  {\renewcommand\qedsymbol{$\blacksquare$}\begin{proof}[Solution]}
{\end{proof}}

\newenvironment{sproof}{
  \renewcommand\qedsymbol{$\square$}
  \begin{proof}
  }{
  \end{proof}
}

\newtheorem{lemma}{Lemma}

\begin{document}

\title{Algebra: Chapter 0 Exercises\\ \large Chapter 2, Section 7}
\author{David Melendez}
\maketitle

\begin{problem}{7.4}
  Prove that the relation defined in Exercise 5.10 on a free abelian group $F = F^{ab}(A)$ by 
  \begin{equation*}
    f \sim f' \Leftrightarrow (\exists g\in G): ~ f'-f = 2g
  \end{equation*}
  is compatible with the group structure.
  Determine the quotient $F/{\sim}$ as a better known group.
\end{problem}
\begin{solution}
  The relation $\sim$ is compatible with the group structure on $F$ if and only if, considering 
  Proposition 7.3,
  \begin{equation*}
    (\forall f,f',a\in F): ~ f\sim f'\implies a+f\sim a+f' 
  \end{equation*}
  (since $F$ is abelian).
  Suppose then, we have $f\sim f'$, that is $f' - f = 2g$.
  We then have $(f' + a) - (f + a) = 2g$, and hence $a+f'\sim a+f$.

  To determine $F/{\sim}$ as a better known group, we will establish an isomorphism between the group $F$ and the abelian group 
  $G := (\mathbb{Z}/2\mathbb{Z})^A$. \\
  Define the set function $\kappa : A\to G$ as follows:
  \begin{equation*}
    (\forall a,a'\in A):~ \kappa(a)(a') = [\delta_{a'a}]_2,
  \end{equation*}
  where $\delta$ is the kronecker delta function.
  We then use this function and the universal property for free (abelian) groups to induce a group
  homomorphism $\varphi : F\to G$ that makes the following diagram commute:
  \[ \begin{tikzcd}
    & F^{ab}(A) \ar[dashed]{r}{\varphi} & (\mathbb{Z}/2\mathbb{Z})^A \\
    & A \ar{u}{j} \ar{ur}[swap]{\kappa}
  \end{tikzcd} \]
  where $j : A\to F$ is the usual inclusion.
  We will now prove two lemmas essential to constructing the desired isomorphism.

  \newpage

  \begin{lemma}
    The homomorphism $\varphi$ is surjective.
  \end{lemma}
  \begin{sproof}
    Suppose $f\in (\mathbb{Z}/2\mathbb{Z})^A$, let $\bar{a} = j(a) \in F$, and allow 
    $f(x)y$ to be "multiplication" by a coset representitive of $f(x)$, i.e. 
    $[0]_2y= 0y$ and $[1]_2y = 1y$.
    We then have, with the symbol at the top of each sigma representing for clarity the abelian group
    in which the sum is taking place,
    \begin{align*}
      (\forall a,a'\in A) \varphi\left( \sum_{a\in A}^{F} f(a)\bar{a} \right)(a')
      &= \sum_{a\in A}^{G} \left( \varphi(f(a)\bar{a}) \right)(a') \\
      &= \sum_{a\in A}^{G} \left( f(a)(\varphi(\bar{a})(a')) \right) \\
      &= \sum_{a\in A}^{\mathbb{Z}/2\mathbb{Z}} \left( f(a)([\delta_{a'a}]_2) \right) \\
      &= f(a')
    \end{align*}
    Hence every $f\in G$ is in the image of $\varphi$.
  \end{sproof}
  \begin{lemma}
    The homomorphism $\varphi$ agrees with the relation $\sim$; that is, 
    \begin{equation*}
      f\sim f'\implies \varphi(f) = \varphi(f')
    \end{equation*}
    \begin{sproof}
      Suppose $f,f'\in F$ and $f\sim f'$.
      We then have, for all $a\in A$ and for some $g\in (\mathbb{Z}/2\mathbb{Z})^A$:
      \begin{align*}
        (\varphi(f') - \varphi(f))(a) &= \varphi(f'-f)(a) \\
        &= (2g)(a) \\
        &= 2(g(a)) \\
        &= [0]_2
      \end{align*}
      (as $\mathbb{Z}/2\mathbb{Z}$ has order 2); hence $\varphi(f') = \varphi(f)$, as desired.
    \end{sproof}
  \end{lemma}
  \begin{lemma}
    If we define $H = [e_F]_{\sim}$ then we have $F/H = F/{\sim}$, and $\ker(\varphi) = H$.
  \end{lemma}
  \begin{sproof}
    The first statement follows from Proposition 7.4, Proposition 7.7, and the definition of quotient
    by a normal subgroup. \\
    For the second statement, first suppose that $h\in H$. 
    It then follows, since $h\sim e_F$, that $\varphi(h) = \varphi(e_F) = e_G$, so $h\subseteq F$.
    For the other direction, suppose $f\in\ker(\varphi)$, and 
    \begin{equation*}
      f = \sum_{a\in A} n_a \bar{a}
    \end{equation*}
    We then have, for all $a'\in A$,
    \begin{align*}
      \varphi(f)(a') &= \varphi\left( \sum_{a\in A} n_a\bar{a} \right)(a') \\
      &= \sum_{a\in A}\left( n_a\varphi(\bar{a})(a') \right) \\
      &= \sum_{a\in A}\left( n_a\kappa(a)(a') \right) \\
      &= \sum_{a\in A}\left( n_a[\delta_{a'a}]_2 \right) \\
      &= [n_{a'}]_2
    \end{align*}
    Since $\varphi(f) = e_G$, we know that each $n_{a'}$ is congruent to 0 modulo two; that is, even.
    Hence, we have:
    \begin{align*}
      f - e_F &= f \\
      &= 2\sum_{a\in A} \frac{n_a}{2} \bar{a}
    \end{align*}
    This shows that $f\sim e_f$ and thus $f \in H$, giving us $\ker(\varphi)\subseteq H$.
    Therefore $\ker(\varphi) = H$, as desired.
  \end{sproof}
  With this all established, we can finally find our isomorphism.
  We now construct a homomorphism 
  $\widetilde{\varphi} : F/H\to G$ using the universal property for quotient by an equivalence relation:
  \[\begin{tikzcd}
      & F \ar{d}[swap]{\pi} \ar{r}{\varphi} & G \\
    & F/H \ar[dashed]{ur}[swap]{\widetilde{\varphi}} &
  \end{tikzcd}\]
  Here, $\pi$ is the quotient map $\pi(f) = [f]_{\sim}$, and $\widetilde{\varphi}$ is the unique homomorphism
  making the diagram commute.
  The first isomorphism theorem shows that $\widetilde{\varphi}$ is an isomorphism,
  and so $F/{\sim} \cong (\mathbb{Z}/2\mathbb{Z})^A$.
\end{solution}
\end{document}
