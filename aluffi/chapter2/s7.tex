%&pdflatex 
\documentclass[12pt]{article} 
\usepackage[margin=1in]{geometry} 
\usepackage{amsmath,amsthm,amssymb,amsfonts,tikz-cd, mathrsfs} 

\newenvironment{problem}[2][Problem]{\begin{trivlist}
\item[\hskip \labelsep {\bfseries #1}\hskip \labelsep {\bfseries #2.}]}{\end{trivlist}}

\newenvironment{proposition}[1][Proposition]{\begin{trivlist}
\item[\hskip \labelsep {\bfseries #1.}]}{\end{trivlist}}

\newenvironment{definition}[1][Definition]{\begin{trivlist}
\item[\hskip \labelsep {\bfseries #1.}]}{\end{trivlist}}

\newcommand{\til}{\char`\~}
\newcommand{\bs}{\textbackslash} 

\newcommand{\catname}[1]{\normalfont\textbf{#1}}
\newcommand{\catsup}[2]{\normalfont\textbf{#1}^{#2}}
\newcommand{\catsub}[2]{\normalfont\textbf{#1}_{#2}}

\newcommand{\Hom}{\text{Hom}}
\newcommand{\Homc}[2]{\Hom_{\catname{#1}}(#2)}

\newcommand{\Obj}{\text{Obj}}
\newcommand{\Objc}[1]{\text{Obj}(\catname{#1})}
\newcommand{\Aut}{\text{Aut}}
\newcommand{\End}{\text{End}}

\newcommand{\id}{\text{id}}

\newcommand{\lcm}[1]{\text{lcm}(#1)}

\newenvironment{solution}
  {\renewcommand\qedsymbol{$\blacksquare$}\begin{proof}[Solution]}
{\end{proof}}

\newenvironment{sproof}{
  \renewcommand\qedsymbol{$\square$}
  \begin{proof}
  }{
  \end{proof}
}

\newtheorem{lemma}{Lemma}

\theoremstyle{remark}
\newtheorem*{exmp}{Example}
\newtheorem*{cexmp}{Counterexample}

\begin{document}

\title{Algebra: Chapter 0 Exercises\\ \large Chapter 2, Section 7}
\author{David Melendez}
\maketitle

\begin{problem}{7.4}
  Prove that the relation defined in Exercise 5.10 on a free abelian group $F = F^{ab}(A)$ by 
  \begin{equation*}
    f \sim f' \Leftrightarrow (\exists g\in G): ~ f'-f = 2g
  \end{equation*}
  is compatible with the group structure.
  Determine the quotient $F/{\sim}$ as a better known group.
\end{problem}
\begin{solution}
  The relation $\sim$ is compatible with the group structure on $F$ if and only if, considering 
  Proposition 7.3,
  \begin{equation*}
    (\forall f,f',a\in F): ~ f\sim f'\implies a+f\sim a+f' 
  \end{equation*}
  (since $F$ is abelian).
  Suppose then, we have $f\sim f'$, that is $f' - f = 2g$.
  We then have $(f' + a) - (f + a) = 2g$, and hence $a+f'\sim a+f$.

  To determine $F/{\sim}$ as a better known group, we will establish an isomorphism between the group $F$ and the abelian group 
  $G := (\mathbb{Z}/2\mathbb{Z})^A$. \\
  Define the set function $\kappa : A\to G$ as follows:
  \begin{equation*}
    (\forall a,a'\in A):~ \kappa(a)(a') = [\delta_{a'a}]_2,
  \end{equation*}
  where $\delta$ is the kronecker delta function.
  We then use this function and the universal property for free (abelian) groups to induce a group
  homomorphism $\varphi : F\to G$ that makes the following diagram commute:
  \[ \begin{tikzcd}
    & F^{ab}(A) \ar[dashed]{r}{\varphi} & (\mathbb{Z}/2\mathbb{Z})^A \\
    & A \ar{u}{j} \ar{ur}[swap]{\kappa}
  \end{tikzcd} \]
  where $j : A\to F$ is the usual inclusion.
  We will now prove two lemmas essential to constructing the desired isomorphism.

  \newpage

  \begin{lemma}
    The homomorphism $\varphi$ is surjective.
  \end{lemma}
  \begin{sproof}
    Suppose $f\in (\mathbb{Z}/2\mathbb{Z})^A$, let $\bar{a} = j(a) \in F$, and allow 
    $f(x)y$ to be "multiplication" by a coset representitive of $f(x)$, i.e. 
    $[0]_2y= 0y$ and $[1]_2y = 1y$.
    We then have, with the symbol at the top of each sigma representing for clarity the abelian group
    in which the sum is taking place,
    \begin{align*}
      (\forall a,a'\in A) \varphi\left( \sum_{a\in A}^{F} f(a)\bar{a} \right)(a')
      &= \sum_{a\in A}^{G} \left( \varphi(f(a)\bar{a}) \right)(a') \\
      &= \sum_{a\in A}^{G} \left( f(a)(\varphi(\bar{a})(a')) \right) \\
      &= \sum_{a\in A}^{\mathbb{Z}/2\mathbb{Z}} \left( f(a)([\delta_{a'a}]_2) \right) \\
      &= f(a')
    \end{align*}
    Hence every $f\in G$ is in the image of $\varphi$.
  \end{sproof}
  \begin{lemma}
    The homomorphism $\varphi$ agrees with the relation $\sim$; that is, 
    \begin{equation*}
      f\sim f'\implies \varphi(f) = \varphi(f')
    \end{equation*}
    \begin{sproof}
      Suppose $f,f'\in F$ and $f\sim f'$.
      We then have, for all $a\in A$ and for some $g\in (\mathbb{Z}/2\mathbb{Z})^A$:
      \begin{align*}
        (\varphi(f') - \varphi(f))(a) &= \varphi(f'-f)(a) \\
        &= (2g)(a) \\
        &= 2(g(a)) \\
        &= [0]_2
      \end{align*}
      (as $\mathbb{Z}/2\mathbb{Z}$ has order 2); hence $\varphi(f') = \varphi(f)$, as desired.
    \end{sproof}
  \end{lemma}
  \begin{lemma}
    If we define $H = [e_F]_{\sim}$ then we have $F/H = F/{\sim}$, and $\ker(\varphi) = H$.
  \end{lemma}
  \begin{sproof}
    The first statement follows from Proposition 7.4, Proposition 7.7, and the definition of quotient
    by a normal subgroup. \\
    For the second statement, first suppose that $h\in H$. 
    It then follows, since $h\sim e_F$, that $\varphi(h) = \varphi(e_F) = e_G$, so $h\in \ker(\varphi)$,
    and hence $H\subseteq \ker(\varphi)$.
    For the other direction, suppose $f\in\ker(\varphi)$, and 
    \begin{equation*}
      f = \sum_{a\in A} n_a \bar{a}
    \end{equation*}
    We then have, for all $a'\in A$,
    \begin{align*}
      \varphi(f)(a') &= \varphi\left( \sum_{a\in A} n_a\bar{a} \right)(a') \\
      &= \sum_{a\in A}\left( n_a\varphi(\bar{a})(a') \right) \\
      &= \sum_{a\in A}\left( n_a\kappa(a)(a') \right) \\
      &= \sum_{a\in A}\left( n_a[\delta_{a'a}]_2 \right) \\
      &= [n_{a'}]_2
    \end{align*}
    Since $\varphi(f) = e_G$, we know that each $n_{a'}$ is congruent to 0 modulo two; that is, even.
    Hence, we have:
    \begin{align*}
      f - e_F &= f \\
      &= 2\sum_{a\in A} \frac{n_a}{2} \bar{a}
    \end{align*}
    This shows that $f\sim e_f$ and thus $f \in H$, giving us $\ker(\varphi)\subseteq H$.
    Hherefore $\ker(\varphi) = H$, as desired.
  \end{sproof}
  With this all established, we can finally find our isomorphism.
  We now construct a homomorphism 
  $\widetilde{\varphi} : F/H\to G$ using the universal property for quotient by an equivalence relation:
  \[\begin{tikzcd}
      & F \ar{d}[swap]{\pi} \ar{r}{\varphi} & G \\
    & F/H \ar[dashed]{ur}[swap]{\widetilde{\varphi}} &
  \end{tikzcd}\]
  Here, $\pi$ is the quotient map $\pi(f) = [f]_{\sim}$, and $\widetilde{\varphi}$ is the unique homomorphism
  making the diagram commute.
  The first isomorphism theorem shows that $\widetilde{\varphi}$ is an isomorphism,
  and so $F/{\sim} \cong (\mathbb{Z}/2\mathbb{Z})^A$.
\end{solution}

\begin{problem}{7.6}
  Let $G$ be a group, and let $n$ be a positive integer.
  Consider the relation
  \begin{equation*}
    a\sim b \Leftrightarrow (\exists g\in G){:}~ab^{-1} = g^n
  \end{equation*}
  \begin{enumerate}
    \item Show that in general $\sim$ is not an equivalence relation.
    \item Prove that $\sim$ is an equivalence relation if $G$ is commutative, 
      and determine the corresponding subgroup of $G$.
  \end{enumerate}
  \begin{solution}~
    \begin{enumerate}
      \item Reflexivity and symmetry are easy to prove, so we will show that $\sim$ is not always transitive.
        \begin{exmp}
          Consider the following matrices in $\text{GL}_2(\mathbb{R})$, and let $n=2$ for the relation:
          \begin{align*}
            A = \begin{pmatrix} 1 & 1 \\ 0 & 1 \end{pmatrix}^2 
              = \begin{pmatrix} 1 & 2 \\ 0 & 1 \end{pmatrix}  \\
            B = \begin{pmatrix} 1 & 0 \\ -1 & 1 \end{pmatrix}^2 
              = \begin{pmatrix} 1 & 0 \\ -2 & 1 \end{pmatrix} 
          \end{align*}
          Clearly we have $A\sim I$ and $B\sim I$, so transitivity would imply that $AB\sim I$.
          Hence we should have
          \begin{align*}
            AB^{-1} &= \begin{pmatrix} 1&2\\0&1 \end{pmatrix}\begin{pmatrix}1&0\\2&1\end{pmatrix} \\
            &= \begin{pmatrix}4&1\\1&0\end{pmatrix}
          \end{align*}
          Since this matrix has a negative eigenvalue, it has no square roots in $\text{GL}_2(\mathbb{R})$.
        \end{exmp}
      \item Suppose $a,b,c\in G$ and:
        \begin{align*}
          ab^{-1} &= g_1^n \\
          bc^{-1} &= g_2^n
        \end{align*}
        so $a\sim b$ and $b\sim c$.
        We then have:
        \begin{align*}
          ac^{-1} &= (ab^{-1})(bc^{-1}) \\
          &= g_1^ng_2^n \\
          &= (g_1g_2)^n
        \end{align*}
        Hence $a\sim c$, making $\sim$ transitive.
        The corresponding subgroup of $G$ is the equivalence class $[e]_{\sim}$; that is the set of all
        $g^n$ for $g\in G$.
    \end{enumerate}
  \end{solution}
\begin{problem}{7.7}
    Let $G$ be a group, $n$ a positive integer, and let 
    $H\subseteq G$ be the subgroup generated by all elements of order $n$ in $G$.
    Prove that $H$ is normal.
  \end{problem}
  \begin{solution}
    Let $N\subseteq G$ be the set of all elements of order $n$ in $G$, and let $H=\langle N\rangle$.
    Any $h\in H$ can then be written as a product of powers of generators in $N$, as follows:
    \begin{equation*}
      h = \prod_{i=1}^m h_i^{k_i}
    \end{equation*}
    Given any $g\in G$, we have
    \begin{align*}
      ghg^{-1} &= g\left(\prod_{i=1}^m h_i^{k_i}\right)g^{-1} \\
      &= \prod_{i=1}^m gh_i^{k_i}g^{-1} \\
      &= \prod_{i=1}^m (gh_ig^{-1})^{k_i} \\
      &\in  H
    \end{align*}
    where the ostensibly innocuous second and third equalities follow from the fact that\\
    $(gag^{-1})(gbg^{-1}) = g(ab)g^{-1}$.
  \end{solution}
\end{problem}
\begin{problem}{7.10}
  Let $G$ be a group, and $h\subseteq G$ a subgroup.
  With notation in Exercise 6.7, show that $H$ is normal in $G$ if and only if
  $\forall\gamma\in \text{Inn}(G), \gamma(H)\subseteq H$.  \\
  Conclude that if $H$ is normal in $G$ then there is an interesting homomorphism \\
  $\text{Inn}(G)\to \text{Aut}(H)$
\end{problem}
\begin{solution}
  By exercise 7.3:
  \begin{align*}
    H\text{ normal } &\Leftrightarrow (\forall g\in G){:}~gHg^{-1}\subseteq H \\
    &\Leftrightarrow (\forall g\in G){:}~ \gamma_g(H)\subseteq H \\
    &\Leftrightarrow (\forall \gamma\in\text{Inn}(G)){:}~ \gamma(H)\subseteq H \\
  \end{align*}
  Every inner automorphism of $G$ is an automorphism of $H$ \\ 
  ($\gamma_g(h)=e\implies ghg^{-1}=e\implies h=e$), so the inclusion morphism 
  $\iota:\text{Inn}(G)\to \Aut(H)$ is a perfectly fine homomorphism.
\end{solution}
\begin{problem}{7.11}
  Let $G$ be a group, and let $[G,G]$ be the subgroup of $G$ generated by all elements of the form
  $aba^{-1}b^{-1}$.
  \begin{enumerate}
    \item Prove that $[G,G]$ is normal in $G$.
    \item Prove that $G/[G,G]$ is commutative.
  \end{enumerate}
\end{problem}
\begin{solution}~
  \begin{enumerate}
    \item Suppose $\gamma\in \text{Inn}(G)$ and $h\in [G,G]$ so that 
      $h = \prod_{i=1}^{n} a_ib_ia_i^{-1}b_i^{-1}$.
      We then have:
      \begin{align*}
        \gamma_g(h) &= \gamma\left( \prod_{i=1}^{n}a_ib_ia_i^{-1}b_i^{-1} \right) \\
        &= \prod_{i=1}^{n}\gamma(a_ib_ia_i^{-1}b_i^{-1}) \\
        &= \prod_{i=1}^{n}\left( \gamma(a_i)\gamma(b_i)\gamma(a_i)^{-1}\gamma(b_i)^{-1} \right) \\
        &\in [G,G]
      \end{align*}
      Hence $\gamma([G,G])\subseteq [G,G]$, so $[G,G]$ is normal.
    \item Suppose $g_1, g_2 \in [G,G]$. We then have:
      \begin{align*}
        (g_1g_2)(g_2g_1)^{-1} &= g_1g_2g_1^{-1}g_2^{-1} \\
        &\in [G,G] 
      \end{align*}
      Hence, with $H:=[G,G]$, 
      \begin{align*}
        (g_1H)(g_2H) &= (g_1g_2)H \\
        &= (g_2g_1)H \\
        &= (g_1H)(g_2H)
      \end{align*}
      for all $g_1H,g_2H\in G/[G,G]$, making $G/[G,G]$ commutative.
  \end{enumerate}
\end{solution}
\begin{problem}{7.12}
  Let $F=F(A)$ be a free group, and let $f:A\to G$ be a set-function from the set $A$
  to a commutative group $G$.
  Prove that $f$ induces a unique homomorphism $F/[F,F]\to G$, where $[F,F]$ is the commutator subgroup
  of $F$.
  Conclude that $F/[F,F]\cong F^{ab}(A)$.
\end{problem}
\begin{solution}
  For this, we simply invoke the universal property for free groups and quotients, and "merge" them together.
  Let $A$ be a set, $G$ a group, $f:A\to G$ be any set function, $j:A\to F(A)$ be the canonical injection,
  and $\pi$ be the canonical projection.
  Invoking the universal properties for free groups and quotients respectively, 
  we find a unique $\sigma:F\to G$ such that $\sigma j = f$. 
  Since $G$ is abelian, we know that $[F,F] \subseteq \ker(\varphi)$, since, for all $a,b\in F(A)$,
  \begin{align*}
    \varphi([a,b]) &= [\varphi(a), \varphi(b)] \\
    &= e_G
  \end{align*}
  so we can invoke the universal property for quotients (by Theorem 7.12) to find a 
  unique $\varphi:F/[F,F]\to G$ such that $\varphi\pi = \sigma$.
  This is all illustrated in the following diagram:
  \[\begin{tikzcd}
      & F/[F,F] \ar[dashed]{dr}{\varphi} & \\
    & F(A) \ar{u}{\pi} \ar[dashed]{r}{\sigma} & G \\
    & A \ar{u}{j} \ar{ur}[swap]{f} &
  \end{tikzcd}\]
  Hence, given a set function $f:A\to G$, there is a unique group homomorphism $\varphi$ such that
  \begin{align*}
    \varphi\pi = \sigma &\implies \varphi\pi j = \sigma j \\
    &\implies \varphi(\pi j) = f 
  \end{align*}
  Hence the pair $(F/[F,F], \pi j)$ satisfies the universal property for free abelian groups, 
  so $F/[F,F]\cong F^{ab}(A)$.
\end{solution}
\begin{problem}{7.13}
  Let $A,B$ be sets and $F(A), F(B)$ the corresponding free groups.
  Assume $F(A)\cong F(B)$.
  Prove that if $A$ is finite, then so is $B$, and $A\cong B$.
\end{problem}
\begin{solution}
  Since $F(A)\cong F(B)$, we know there is some morphism $j:A\to F(B)$ such that $(A, j)$
  satisfies the universal property for the free group over B. 
  Hence, applying the same argument from the previous exercise (7.12), we can examine the following diagram
  \[\begin{tikzcd}
      & F(B)/[F(B),F(B)] \ar[dashed]{dr}{\varphi} & \\
    & F(B) \ar{u}{\pi} \ar[dashed]{r}{\sigma} & G \\
    & A \ar{u}{j} \ar{ur}[swap]{f} &
  \end{tikzcd}\]
  to find that $F(B)/[F(B),F(B)]$ satisfies the universal property for the abelian free group over $A$.
  Hence, we have:
  \begin{align*}
    F^{ab}(B) &\cong F(B)/[F(B),F(B)] \\
    &\cong F^{ab}(A)
  \end{align*}
  Therefore, by Exercise 5.10, $A\cong B$.
\end{solution}
\end{document}
