%&pdflatex 
\documentclass[12pt]{article} 
\usepackage[margin=1in]{geometry} 
\usepackage{amsmath,amsthm,amssymb,amsfonts,tikz-cd} 

\newenvironment{problem}[2][Problem]{\begin{trivlist}
\item[\hskip \labelsep {\bfseries #1}\hskip \labelsep {\bfseries #2.}]}{\end{trivlist}}

\newenvironment{proposition}[1][Proposition]{\begin{trivlist}
\item[\hskip \labelsep {\bfseries #1.}]}{\end{trivlist}}

\newenvironment{definition}[1][Definition]{\begin{trivlist}
\item[\hskip \labelsep {\bfseries #1.}]}{\end{trivlist}}

\newcommand{\til}{\char`\~}
\newcommand{\bs}{\textbackslash} 

\newcommand{\catname}[1]{\normalfont\textbf{#1}}
\newcommand{\catsup}[2]{\normalfont\textbf{#1}^{#2}}
\newcommand{\catsub}[2]{\normalfont\textbf{#1}_{#2}}

\newcommand{\Hom}{\text{Hom}}
\newcommand{\Homc}[2]{\Hom_{\catname{#1}}(#2)}

\newcommand{\Obj}{\text{Obj}}
\newcommand{\Objc}[1]{\text{Obj}(\catname{#1})}
\newcommand{\Aut}{\text{Aut}}
\newcommand{\End}{\text{End}}

\newcommand{\id}{\text{id}}

\newcommand{\lcm}[1]{\text{lcm}(#1)}

\newenvironment{solution}
  {\renewcommand\qedsymbol{$\blacksquare$}\begin{proof}[Solution]}
{\end{proof}}

\newenvironment{sproof}{
  \renewcommand\qedsymbol{$\square$}
  \begin{proof}
  }{
  \end{proof}
}

\begin{document}

\title{Algebra: Chapter 0 Exercises\\ \large Chapter 2, Section 3}
\author{David Melendez}
\maketitle

\begin{problem}{3.1}
  Let $\varphi : G\to H$ be a morphism in a category \catname{C} with products.
  Explain why there is a unique morphism 
  $$(\varphi\times\varphi) : G\times G\to H\times H$$
    compatible in the evident way with the natural projections.
\end{problem}
\begin{solution}
  I'm going to assume the author means the following:\\
  Since $H\times H$ is a product in \catname{C}, we have that 
  $G\times G$ along with two ``copies'' of $\varphi : G\to H$
  admits a unique morphism $(\varphi\times\varphi) : G\times G\to H\times H$
  that makes the following diagram commute:
  \[\begin{tikzcd}
    &&&H\\
    &G\times G 
    \ar{r}{\varphi\times\varphi} 
    \ar[bend left]{urr}{\varphi\pi_{G_1}}
    \ar[bend right]{drr}[swap]{\varphi\pi_{G_2}}
    &H\times H 
    \ar{ur}{\pi_{H_1}}
    \ar{dr}[swap]{\pi_{H_2}}
    &\\
    &&&H
  \end{tikzcd}\]
  where $\pi_{G_1}$ and $\pi_{G_2}$ are the canonical projections of
  $G\times G$ onto $G$.
\end{solution}
\begin{problem}{3.2}
  Let $\varphi : G\to H$ and $\psi : H\to K$ be morphisms in a category with
  products, and consider morphisms between the products 
  $G\times G$, $H\times H$, and $K\times K$ as in Exercise 3.1. 
  Prove that
  $$(\psi\varphi)\times (\psi\varphi) 
  = (\psi\times\psi)(\varphi\times\varphi)$$
\end{problem}
\begin{solution}
  To demonstrate this result, first stare at this diagram, mapping 
  $G\times G$ to the product $H\times H$ and 
  $H\times H$ to the product $K\times H$.
  \[\begin{tikzcd}
    & G\ar{r}{\varphi} & H\ar{r}{\psi} & K \\
    & G\times G \ar{r}{\varphi\times\varphi}
    \ar{u}\ar{d} 
    & H\times H \ar{r}{\psi\times\psi}
    \ar{u}\ar{d} 
    & K\times K \ar{u} \ar{d} \\
    & G\ar{r}[swap]{\varphi} & H\ar{r}[swap]{\psi} & K 
\end{tikzcd}\]
Compare it to the following diagram, mapping 
$G\times G$ straight to $K\times K$:
\[\begin{tikzcd}
  & G\ar{r}{\psi\varphi} & K\\
  & G\times G \ar{r}{(\psi\varphi)\times(\psi\varphi)} 
  \ar{u} \ar{d} 
  & K\times K \ar{u} \ar{d} \\
  & G\ar{r}[swap]{\psi\varphi} & K
\end{tikzcd}\]
Notice that we have morphisms 
$(\psi\varphi) \times (\psi\varphi)$ and 
$(\psi\times\psi)(\varphi\times\varphi)$
from $G\times G$ to $K\times K$, both determined by the universal property for
products with regards to $K\times K$. 
These must be equal per Problem 3.1.
\end{solution}
\begin{problem}{3.3}
  Show that if $G$ and $H$ are both \textit{abelian} groups, then
  $G\times H$ satisfies the universal property for coproducts in 
  \catname{Ab}.
\end{problem}
\begin{solution}
  That $G\times H$ satisfies the universal property for coproducts in \catname{Ab} means the following:\\
  Every triple $(X, \delta_G, \delta_H)$ (as in the diagram) admits a unique morphism $\sigma$ 
  such that the following diagram commutes:
  \[\begin{tikzcd}
    & & G \ar{dl}[swap]{f_G} \ar{d}{\delta_G}\\
    & X & G\times H \ar{l}[swap]{\sigma} \\
    & & H \ar{ul}{f_H} \ar{u}[swap]{\delta_G}
  \end{tikzcd}\]
  That is, $$\sigma\delta_G = f_G$$ $$\sigma\delta_H = f_H$$
  Define $\sigma : G\times H \to X$ by 
  $$\sigma(g,h) = f_G(g)\cdot f_H(h)$$
  Also define $\delta_G : G\to G\times H$ $\delta_H : H\to G\times H$ by
  \begin{align*}
    \delta_G(g) = (g, e_H) \\
    \delta_G(h) = (e_G, h) 
  \end{align*}
  The proof that these are homomorphisms is trivial.
  To show that $\sigma$ makes the diagram commute, note that
  \begin{align*}
    \sigma(\delta_G(g)) &= \sigma(g,e_H) \\
    &= f_G(g)\cdot f_H(e_H) \\
    &= f_G(h)
  \end{align*}
  The same proof applies to $H$. \\
  This is where our groups being abelian comes in. We will show that $\sigma$ is a homomorphism:
  \begin{align*}
    \sigma\left( (g_1,h_1)(g_2,h_2) \right) &= \sigma(g_1g_2,h_1h_2) \\
    &= f_G(g_1g_2)\cdot f_H(h_1h_2) \\
    &= f_G(g_1)g_G(g_2)f_H(h_1)f_H(h_2) \\
    &= \left( f_G(g_1)f_H(h_1) \right)\left( f_G(g_2)f_h(h_2) \right) \\
    &= \sigma(g_1,h_1)\cdot\sigma(g_2,h_2)
  \end{align*}
  done
\end{solution}
\begin{problem}{3.4}
  Let $G,H be groups$, and assume that $G\cong H\times H$. Can you conclude that H is trivial? (Hint: no.)
\end{problem}
\begin{solution}
\end{solution}
\end{document}
