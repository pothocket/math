%&pdflatex 
\documentclass[12pt]{article} 
\usepackage[margin=1in]{geometry} 
\usepackage{amsmath,amsthm,amssymb,amsfonts,tikz-cd, mathrsfs} 

\newenvironment{problem}[2][Problem]{\begin{trivlist}
\item[\hskip \labelsep {\bfseries #1}\hskip \labelsep {\bfseries #2.}]}{\end{trivlist}}

\newenvironment{proposition}[1][Proposition]{\begin{trivlist}
\item[\hskip \labelsep {\bfseries #1.}]}{\end{trivlist}}

\newenvironment{definition}[1][Definition]{\begin{trivlist}
\item[\hskip \labelsep {\bfseries #1.}]}{\end{trivlist}}

\newcommand{\til}{\char`\~}
\newcommand{\bs}{\textbackslash} 

\newcommand{\catname}[1]{\normalfont\textbf{#1}}
\newcommand{\catsup}[2]{\normalfont\textbf{#1}^{#2}}
\newcommand{\catsub}[2]{\normalfont\textbf{#1}_{#2}}

\newcommand{\Hom}{\text{Hom}}
\newcommand{\Homc}[2]{\Hom_{\catname{#1}}(#2)}

\newcommand{\Obj}{\text{Obj}}
\newcommand{\Objc}[1]{\text{Obj}(\catname{#1})}
\newcommand{\Aut}{\text{Aut}}
\newcommand{\End}{\text{End}}

\newcommand{\id}{\text{id}}

\newcommand{\lcm}[1]{\text{lcm}(#1)}

\newenvironment{solution}
  {\renewcommand\qedsymbol{$\blacksquare$}\begin{proof}[Solution]}
{\end{proof}}

\newenvironment{sproof}{
  \renewcommand\qedsymbol{$\square$}
  \begin{proof}
  }{
  \end{proof}
}

\newtheorem{lemma}{Lemma}

\theoremstyle{remark}
\newtheorem*{exmp}{Example}
\newtheorem*{cexmp}{Counterexample}


\begin{document}

\title{Algebra: Chapter 0 Exercises\\ \large Chapter 2, Section 8}
\author{David Melendez}
\maketitle

\begin{problem}{8.1}
  If a group $H$ may be realized as a subgroup of two groups $G_1$ and $G_2$, and
  \begin{equation*}
    \frac{G_1}{H} \cong \frac{G_2}{H},
  \end{equation*}
  does it follow that $G_1\cong G_2$?
  Give a proof or counterexample
\end{problem}
\begin{solution}
  No.
  As a counterexample, take $G_1 = D_6$, $G_2 = C_6$, and $H=C_3$. 
  In this case, we have $D_6/C_3\cong C_2\cong C_6/C_3$, but $D_6\not\cong C_3$.
\end{solution}

\begin{problem}{8.2}
  Suppose $G$ is a group, and $H\subseteq G$ is a subgroup of index 2.
  Prove that $H$ is normal in $G$.
\end{problem}
\begin{solution}
  Consider the function $\varphi : G\to C_2$ defined by
  \begin{equation*}
    \varphi(g) = \begin{cases}
      0 & g\in H \\
      1 & g\not\in H
    \end{cases}
  \end{equation*}
  To check that this is a homomorphism, suppose $g_1,g_2\not\in H$.
  In particular, $g_2^{-1}\not\in H$, so 
  \begin{equation*}
    g_1H = g_2^{-1}H,
  \end{equation*}
  since there are only two left cosets of $H$ in $G$, so
  \begin{equation*}
    g_1g_2\in H
  \end{equation*}
  and hence
  \begin{align*}
    \varphi(g_1g_2) &= 0 \\
    &= 1+1\\
    &= \varphi(g_1) + \varphi(g_2).
  \end{align*}
  Clearly $\ker\ \varphi=H$, so $H$ is normal in $G$.
\end{solution}

\begin{problem}{8.7}
  Let $\langle A|\mathscr{R}\rangle$ resp. $\langle A'|\mathscr{R}'\rangle$ be presentations for two groups
  $G$ resp. $G'$; we may assume that $A$ and $A'$ are disjoint.
  Prove that the groups $G*G'$ presented by $$\langle A\cup A' | \mathscr{R}\cup\mathscr{R}'\rangle$$
  satisfies the universal property for the coproduct of $G$ and $G'$ in $\catname{Grp}$.
\end{problem}
\begin{solution}
  Let $G, G'$ and $A,A',\mathscr{R},\mathscr{R'}$ be as described above, let
  $R$ resp. $R'$ resp. $\hat{R}$ be the normal closures of 
  $\mathscr{R}$ resp. $\mathscr{R}'$ $\mathscr{R\cup R'}$, and let $H$ be any group.
  Consider the diagram below, in which we use the universal properties of free groups and
  quotient groups to construct two morphisms $\tau: G\to G*G'$ and $\tau':G'\to G*G'$:
  \[\begin{tikzcd}
    & & H & \\
    & \frac{F(A)}{R} \ar{ur}{\varphi} \ar[dashed]{r}{\tau}
    & \frac{F(A\cup A')}{\hat{R}} \ar[dashed]{u}[swap]{\hat{\varphi}}
    & \frac{F(A)}{R'} \ar{ul}[swap]{\varphi'} \ar[dashed]{l}[swap]{\tau'} \\
    & F(A) \ar{u}{\pi} \ar{ur}{q} \ar[dashed]{r}{p}
    & F(A\cup A') \ar{u}{\hat{\pi}}
    & F(A') \ar{u}[swap]{\pi'} \ar{ul}[swap]{q'} \ar[dashed]{l}[swap]{p'} \\
    & A \ar{u}{j} \ar{ur}{u} \ar{r}{\iota} 
    & A\cup A' \ar{u}{\hat{j}}
    & A' \ar{u}[swap]{j'} \ar{ul}[swap]{u'} \ar{l}[swap]{\iota'} 
  \end{tikzcd}\]
  Here, $\iota$ and $\iota'$ are canonical inclusions (since $A$ and $A'$ are disjoint), 
  $j,\hat{j},j'$ are the canonical inclusions into the free groups, $u$ resp. $u'$ are defined by
  $\hat{j}\iota$ resp. $\hat{j}\iota'$, and $p$ and $p'$ are obtained by applying the universal property
  of free groups. \\
  The morphisms $\pi, \hat\pi,$ and $\pi'$ are the canonical projections, $q$ resp $q'$ are defined by
  $\hat\pi p$ resp. $\hat\pi p'$, and $\tau$ and $\tau'$ are obtained by invoking the universal property
  of quotient groups. \\
  Finally, $\varphi,\varphi'$ are any morphisms, and we propose that there exists a unique morphism
  $\hat\varphi$ such that $\hat\varphi\tau=\varphi$ and $\hat\varphi\tau' = \varphi'$.
  For this, we simply must prove that if we define $\hat\varphi$ by those two relations, then
  $\hat\varphi$ is well defined. \\
  Hence, suppose $\tau(w_1R) = \tau(w_2R)$.
  We then have 
  \begin{equation*}
    \tau(\pi(w_1)) = \tau(\pi(w_2)),
  \end{equation*}
  and hence
  \begin{equation*}
    q(w_1w_2^{-1}) = 0.
  \end{equation*}
  Note that
  \begin{align*}
    \ker q &= \ker (\hat\pi p) \\
    &= \ker(p) \cup p^{-1}(\ker \hat\pi) \\
    &= p^{-1}(\ker \hat\pi) \\
    &= p^{-1}(\hat{R}) \\
    &= R.
  \end{align*}
  This tells us that $w_1w_2^{-1}\in R$, and so $w_1R = w_2R$; hence $\tau$ is injective.
  The same reasoning applies to $\tau'$.
  Since $F(A\cup A')/\hat{R}$ is generated by the images of $\tau$ and $\tau'$,  this
  $\hat\varphi$ is well-defined, and hence unique.
\end{solution}

\begin{problem}{8.12}
  Prove 'by hand' (that is, by using Proposition 6.2), that if $H,K$ are subgroups of $G$, then
  $HK$ is a subgroup of $G$ if $H$ is normal.
\end{problem}
\begin{solution}
  Let $h_1,h_2$ and $k_1,k_2$ be in $H$ and $K$, respectively.
  We then have
  \begin{align*}
    (h_1k_1)(h_2k_2)^{-1} &= h_1k_1k_2^{-1}h_2^{-1} \\
    &= k_1(k_1^{-1}h_1k_1)(k_2^{-1}h_2^{-1}k_2)k_2^{-1} \\
    &= k_1h'h''k_1^{-1} \\
    &\in H.
  \end{align*}
  Hence $HK$ is a group by Proposition 6.2.
\end{solution}
\begin{problem}{8.13}
  Let $G$ be a finite commutative group, and assume $|G|$ is odd.
  Prove that every element of $G$ is a square.
\end{problem}
\begin{solution}
  Let $g\in G$.
  Since $|G|$ is odd, $|g|$ is also odd (by Lagrange's Theorem), so $|g|+1$ is even.
  We then have
  \begin{align*}
    \left(g^{\frac{|g|+1}{2}}\right)^2 &= g^{|g|+1} \\
    &= g
  \end{align*}
  Hence $g$ is a square.
\end{solution}
\begin{problem}{8.14}
  Generalize the result of Exercise 8.13: if $G$ is a group of order $n$, and $k$ is an integer
  relatively prime to $n$, then the function $G\to G$, $g\mapsto g^k$ is surjective.
\end{problem}
\begin{proof}
  Suppose $g\in g$.
  If $\gcd(k,n)=1$, then by Lagrange's Theorem, we have $\gcd(|g|,k)=1$ as well.
  Hence there exist integers $a$ and $b$ such that $a|g|+bk=1$.
  Let $a,b$ be integers that satisfy this property.
  We then have:
  \begin{align*}
    g &= g^{a|g|+bk} \\
    &= g^{a|g|}g^{bk} \\
    &= g^{bk} \\
    &= (g^b)^k \\
  \end{align*}
  Hence every $g\in G$ is in the image of the map mentioned above.
\end{proof}
\begin{problem}{8.15}
  Let $a,n$ be positive integers.
  Prove that $n$ divides $\phi(a^n-1)$, where $\phi$ is Euler's $\phi$-function.
\end{problem}
\begin{solution}
  Let $m=a^n-1$, and consider the group $G=(\mathbb{Z}/m\mathbb{Z})^*$.
  If $a=1$ then the question is nonsense, and if $n=1$ then clearly $1|\phi(a-1)$.
  Hence, assume that $m>1$ and $n>1$.
  We know that $a\in G$ because $a^n-1>a$ and $\gcd(a,a^n-1)=1$.
  We also know that $$a^n\equiv 1\mod m.$$
  Since $x<n$ implies
  \begin{equation*}
    1 < a \leq a^x < a^n-1,
  \end{equation*}
  it then follows that $|a| = n$, and hence $n~|~|G| = \phi(a^n-1)$.
\end{solution}
\begin{problem}{8.16}
  Generalize Fermat's Little Theorem to congruences modulo arbitrary integers.
\end{problem}
\begin{solution}~
  \begin{proposition}[Euler's Theorem]
    Let $a,b$ be positive integers.
    Then
    \begin{equation*}
      a^{\phi(n)} \equiv 1\mod n.
    \end{equation*}
  \end{proposition}
  \begin{sproof}
    We have $[a^{\phi(n)}]_n = [1]_n$ in $(\mathbb{Z}/n\mathbb{Z})^*$.
  \end{sproof}
\end{solution}

\begin{problem}{8.17}
  Assume $G$ is a finite abelian group, and let $p$ be a prime divisor of $|G|$.
  Prove that there exists an element in $G$ of order $p$.
\end{problem}
\begin{solution}
  Let $G_0=G$. 
  We will soon define each $G_k$ to be a quotient $G_{k-1}/H$, where $H$ is a subgroup of prime order.
  We will proceed by (strong) induction on this subscript $k$, proving that, for $0\leq k<\Omega(|G|)$:
  \begin{enumerate}
    \item If $\Omega(n)$ is the number of prime divisors of $n$ including multiplicity, then 
      \begin{equation*}
        \Omega(|G_{k}|) = \Omega(|G|) - k.
      \end{equation*}
    \item There exists an element of $|G_k|$ of prime order.
    \item If $g\in G_{k}$, has prime order $p$, then there exists an element of $G$ of order $p$.
  \end{enumerate}
  \begin{sproof}~\\
    \textbf{Base case (n=0):} Let $G_0=G$.
    \begin{enumerate}
      \item $\Omega(|G_0|)=\Omega(|G|)=\Omega(|G|)-0$
      \item Let $g_0\in G_0$, and for some prime divisor $q_0$ of $|g_0|$, let $h_0=g_0^{\frac{|g_0|}{q_0}}$.
        We then have $|h_0| = q_0$, hence (2) holds.
      \item Trivial.
    \end{enumerate}
    \textbf{Induction ($n=k+1$):} Suppose (1), (2), and (3) hold for $n\leq k$.
    Let $H_k = \langle h_k\rangle$ (which is normal because $G$ is abelian), and define $G_{k+1} = G_k/H_k$.
    \begin{enumerate}
      \item We have:
        \begin{align*}
          \Omega(|G_{k+1}|) &= \Omega\left(\left|\frac{G_k}{H}\right|\right) \\
          &= \Omega\left( \frac{|G_k|}{|H|} \right) \\
          &= \Omega\left( \frac{|G_k|}{q} \right) \\
          &= \Omega(|G_k|) - 1 \\
          &= \Omega(G) - k - 1
        \end{align*}
        as desired.
      \item Same as the base case - let $q_{k+1}$ be a prime divisor of $|G_{k+1}|$, 
        let $g_{k+1}\in G_{k+1}$, and define $h_{k+1}:=g^{\frac{|g_{k+1}|}{q_{k+1}}}$.
        Then $|h_{k+1}|=q_{k+1}$.
      \item Suppose $gH_k\in G_{k+1}$ has order $p$, where $p$ is a prime divisor of $|G_{k+1}|$.
        It then follows that $(gH_k)^p = e$, and so $g^p=h_k^m$ for some integer $m$.
        Since $e=(h^k)^x = g^{px}$ for some $x=|h^k|$, we then have $|g^x|=p$ in $G_k$.
        By our inductive hypothesis, then, we have an element of $G$ of order $p$, as desired.
    \end{enumerate}
    The set $\{q_0, \dots, q_n\}$ where $n=\Omega(G)-1$ is the set of all prime divisors of $|G|$,
    since at each step of the process we have removed a prime divisor before choosing a new one..
    At the end of this process, we have proven that $G$ contains an element of order $q_n$ for each $n$,
    and hence an element of order $p$ for each prime divisor $p$ of $|G|$.
  \end{sproof}
\end{solution}

\begin{problem}{8.18}
  Let $G$ be an abelian group of order $2n$, where $n$ is odd.
  Prove that $G$ has exactly one element of order $2$.
\end{problem}
\begin{solution}
  Let $g,h$ be distinct elements of $G$. 
  If $g$ and $h$ both have order 2, then the subgroup generated by $g$ and $h$ 
  equals $\{e_G,g,h,gh\}$, which has order 4.
  But 4 is not a divisor of $|G|=2n$ for $n$ odd (otherwise $n$ would be even), so $g$ and $h$ do not both have
  order 2. \\
  This does not necessarily hold if $G$ is not commutative. 
  A counterexample is the dihedral group $D_6$, where $f\neq rfr^{-1}$ both have order 2.
\end{solution}
\begin{problem}{8.19}
  Let $G$ be a finite group, and let $d$ be a proper divisor of $|G|$.
  Is it necessarily true that there exists an element of $G$ of order $d$?
\end{problem}
\begin{solution}
  No.
  The group $S_4$ has no elements of order 8, or more generally, of order greater than 4.
\end{solution}
\end{document}
