%&pdflatex 
\documentclass[12pt]{article} 
\usepackage[margin=1in]{geometry} 
\usepackage{amsmath,amsthm,amssymb,amsfonts,tikz-cd, mathrsfs} 

\newenvironment{problem}[2][Problem]{\begin{trivlist}
\item[\hskip \labelsep {\bfseries #1}\hskip \labelsep {\bfseries #2.}]}{\end{trivlist}}

\newenvironment{proposition}[1][Proposition]{\begin{trivlist}
\item[\hskip \labelsep {\bfseries #1.}]}{\end{trivlist}}

\newenvironment{definition}[1][Definition]{\begin{trivlist}
\item[\hskip \labelsep {\bfseries #1.}]}{\end{trivlist}}

\newcommand{\til}{\char`\~}
\newcommand{\bs}{\textbackslash} 

\newcommand{\catname}[1]{\normalfont\textbf{#1}}
\newcommand{\catsup}[2]{\normalfont\textbf{#1}^{#2}}
\newcommand{\catsub}[2]{\normalfont\textbf{#1}_{#2}}

\newcommand{\Hom}{\text{Hom}}
\newcommand{\Homc}[2]{\Hom_{\catname{#1}}(#2)}

\newcommand{\Obj}{\text{Obj}}
\newcommand{\Objc}[1]{\text{Obj}(\catname{#1})}
\newcommand{\Aut}{\text{Aut}}
\newcommand{\End}{\text{End}}

\newcommand{\id}{\text{id}}

\newcommand{\lcm}[1]{\text{lcm}(#1)}

\newenvironment{solution}
  {\renewcommand\qedsymbol{$\blacksquare$}\begin{proof}[Solution]}
{\end{proof}}

\newenvironment{sproof}{
  \renewcommand\qedsymbol{$\square$}
  \begin{proof}
  }{
  \end{proof}
}

\newtheorem{lemma}{Lemma}

\theoremstyle{remark}
\newtheorem*{exmp}{Example}
\newtheorem*{cexmp}{Counterexample}

\begin{document}

\title{Algebra: Chapter 0 Exercises\\ \large Chapter 2, Section 8}
\author{David Melendez}
\maketitle

\begin{problem}{8.1}
  If a group $H$ may be realized as a subgroup of two groups $G_1$ and $G_2$, and
  \begin{equation*}
    \frac{G_1}{H} \cong \frac{G_2}{H},
  \end{equation*}
  does it follow that $G_1\cong G_2$?
  Give a proof or counterexample
\end{problem}
\begin{solution}
  No.
  As a counterexample, take $G_1 = D_6$, $G_2 = C_6$, and $H=C_3$. 
  In this case, we have $D_6/C_3\cong C_2\cong C_6/C_3$, but $D_6\not\cong C_3$.
\end{solution}

\begin{problem}{8.2}
  Suppose $G$ is a group, and $H\subseteq G$ is a subgroup of index 2.
  Prove that $H$ is normal in $G$.
\end{problem}
\begin{solution}
  Consider the function $\varphi : G\to C_2$ defined by
  \begin{equation*}
    \varphi(g) = \begin{cases}
      0 & g\in H \\
      1 & g\not\in H
    \end{cases}
  \end{equation*}
  To check that this is a homomorphism, suppose $g_1,g_2\not\in H$.
  In particular, $g_2^{-1}\not\in H$, so 
  \begin{equation*}
    g_1H = g_2^{-1}H,
  \end{equation*}
  since there are only two left cosets of $H$ in $G$, so
  \begin{equation*}
    g_1g_2\in H
  \end{equation*}
  and hence
  \begin{align*}
    \varphi(g_1g_2) &= 0 \\
    &= 1+1\\
    &= \varphi(g_1) + \varphi(g_2).
  \end{align*}
  Clearly $\ker\ \varphi=H$, so $H$ is normal in $G$.
\end{solution}
\end{document}
