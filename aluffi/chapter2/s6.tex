%&pdflatex 
\documentclass[12pt]{article} 
\usepackage[margin=1in]{geometry} 
\usepackage{amsmath,amsthm,amssymb,amsfonts,tikz-cd, mathrsfs} 

\newenvironment{problem}[2][Problem]{\begin{trivlist}
\item[\hskip \labelsep {\bfseries #1}\hskip \labelsep {\bfseries #2.}]}{\end{trivlist}}

\newenvironment{proposition}[1][Proposition]{\begin{trivlist}
\item[\hskip \labelsep {\bfseries #1.}]}{\end{trivlist}}

\newenvironment{definition}[1][Definition]{\begin{trivlist}
\item[\hskip \labelsep {\bfseries #1.}]}{\end{trivlist}}

\newcommand{\til}{\char`\~}
\newcommand{\bs}{\textbackslash} 

\newcommand{\catname}[1]{\normalfont\textbf{#1}}
\newcommand{\catsup}[2]{\normalfont\textbf{#1}^{#2}}
\newcommand{\catsub}[2]{\normalfont\textbf{#1}_{#2}}

\newcommand{\Hom}{\text{Hom}}
\newcommand{\Homc}[2]{\Hom_{\catname{#1}}(#2)}

\newcommand{\Obj}{\text{Obj}}
\newcommand{\Objc}[1]{\text{Obj}(\catname{#1})}
\newcommand{\Aut}{\text{Aut}}
\newcommand{\End}{\text{End}}

\newcommand{\id}{\text{id}}

\newcommand{\lcm}[1]{\text{lcm}(#1)}

\newenvironment{solution}
  {\renewcommand\qedsymbol{$\blacksquare$}\begin{proof}[Solution]}
{\end{proof}}

\newenvironment{sproof}{
  \renewcommand\qedsymbol{$\square$}
  \begin{proof}
  }{
  \end{proof}
}

\newtheorem{lemma}{Lemma}

\begin{document}

\title{Algebra: Chapter 0 Exercises\\ \large Chapter 2, Section 6}
\author{David Melendez}
\maketitle

\begin{problem}{6.2}
  Prove that the set of upper-triangular matrices form a subgroup of $\text{GL}_2(\mathbb{Z})$.
\end{problem}
\begin{solution}
  Let $V$ be an $n$-dimensional vector space over $\mathbb{C}$, 
  and let $\mathcal{M} : \mathcal{L}(V)\to \mathbb{C}^{n,n}$ be the isomorphism that takes a linear
  operator to its matrix with respect to some basis $$\mathcal{B}=\{v_1,\dots,v_n\}.$$
  Let $A, B$ be upper-triangular matrices, and let $S,T$ (respectively) be $\mathcal{M}^{-1}(A)$ and
  $\mathcal{M}^{-1}(B)$. 
  We then know, due to a theorem in Axler, that $\text{span}(v_1,\dots,v_k)$ is invariant under
  $S$ and $T$ for $1\leq k\leq n$ (this property is equivalent to the matrix being upper-triangular). 
  Using this property and the invertibility of $T$, it then follows that
  \begin{align*}
    T^{-1}(a_1v_1+\dots+a_kv_k) &= T^{-1}(T(b_1v_1+\dots+b_kv_k)) \\
    &= b_1v_1 + \dots + b_kv_k
  \end{align*}
  meaning $\text{span}(v_1,\dots,v_k)$ is invariant under $T^{-1}$, and hence that $T^{-1}$ 
  is upper-triangular.
  Similarly, 
  \begin{align*}
    (ST^{-1})(a_1v_1+\dots+a_kv_k) &= S(b_1v_1+\dots+b_kv_k) \\
    &= c_1v_1 + \dots + c_kv_k,
  \end{align*}
  making $ST^{-1}$ upper-triangular and completing the proof.
\end{solution}
\begin{problem}{6.4}
  Let $G$ be a commutative group, and let $n>0$ be an integer.
  Prove that $\{g^n | g \in G\}$ is a subgroup of $G$.
  Prove that this is not necessarily the case if $G$ is not commutative.
\end{problem}
\begin{solution}
  Let $G$ be a commutative group, and let $G' = \{g^n | g\in G\}$  where $n$ is any positive integer.
  To prove that this is a group, suppose $g = g_1^n$ and $h=g_2^n$ are elements of $G'$.
  We then have:
  \begin{align*}
    gh^{-1} &= (g_1^n)(g_2^n)^{-1} \\
    &= (g_1)^n(g_2^{-1})^n \\
      &= (g_1g_2^{-1} )^n \\
      &\in G'
  \end{align*}
  Hence $G'$ is a subgroup of $G$. \\
  As a counterexample in the case that $G$ is not commutative, let $G = F(\{x,y\})$, the free group generated
  by $x$ and $y$, and let $n=2$. 
  In order for $G'$ to be closed under its operation, we would need to have $g\in G$ such that
  $$g^2=x^2y^2.$$
  That such a $g$ does not exist is turning out to be harder to prove than I suspected, so I'll come back to this later.
\end{solution}
\begin{problem}{6.6}$ $
  \begin{enumerate}
    \item Let $H,H'$ be subgroups of a group $G$.
      Prove that $H\cup H'$ is a subgroup of $G$ only if $H\subseteq H'$ or $H'\subseteq H$.
    \item On the other hand, let $H_0\subseteq H_1\subseteq H_2\subseteq\cdots$ 
      be subgroups of a group $G$.
      Prove that $G'=\bigcup_{i\geq0}H_i$ is a subgroup of $G$.
  \end{enumerate}
\end{problem}
\begin{solution} $ $
  \begin{enumerate}
    \item Suppose $H\cup H'$ is a subgroup of G. 
      We then have, by closure, for all $h\in h$ and $h\in h'$, that $hh' = g$ for some $g\in H\cup H'$.
      If $g\in H$, we have $h' = h^{-1}g \in H$, meaning $H'\subseteq H$.
      Alternatively, if $g\in H'$, we have $h = g(h')^{-1} \in H'$ meaning $H\subseteq H'$.
    \item Let $h_1\in H_j$ and $h_2\in H_k$ (both in $G'$, of course),
        and assume without loss of generality that $j\leq k$.
        By the sequence of subset relations, we know that $h_1\in H_k$, so \\$h_1h_2^{-1}\in H_k \subseteq G'$,
        completing the proof.
  \end{enumerate}
\end{solution}
\begin{problem}{6.8}
  Prove that an abelian group $G$ is finitely generated if and only if there is a surjective homomorphism
  \begin{equation*}
    \bigoplus_{i=1}^n \mathbb{Z} \twoheadrightarrow G
  \end{equation*}
  for some $n$.
\end{problem}
\begin{solution}
  First, suppose that an abelian group $G$ is finitely generated.
  This, by definition, means that there exists a finite subset $A$ of $G$ such that $\langle A\rangle=G$;
  in other words, $G$ is the image of the homomorphism $\varphi_A$ obtained by applying the universal property
  for the free abelian group over $A$ as follows:
  \[
  \begin{tikzcd}
    & F^{ab}(A) \ar[dashed]{r}{\varphi_A} & G \\
    & A \ar{u}{j} \ar{ur}[swap]{\iota}
  \end{tikzcd} \]
  where $\iota$ and $j$ are the inclusion maps into $G$ and $F^{ab}(A)$, respectively.
  We know by exercise $5.7$ that $Z=\bigoplus_{i=1}^n\mathbb{Z}\cong F^{ab}(A)$, so we have a surjection
  \[\begin{tikzcd}
      & Z \ar{r}{\sim} 
      & F^{ab}(A) \ar[twoheadrightarrow]{r}{\varphi_A} 
      & G
  \end{tikzcd}\]
  For the proof in the other direction, suppose we have a surjective homomorphism \\
  $\varphi : Z \to G$.
  For integers $0 \leq m\leq n$, let $\beta_m$ be the $n$-tuple with 0 in every slot except for the $m$th
  slot, where there is a 1.
  Define $A$ to be the set $\{\varphi(\beta_1),\dots,\varphi(\beta_n)\}$
  (since the coproduct and product in \catname{Ab} are the same), and let $a_m$ be the $m$th element
  of $A$ as listed above. Define the isomorphism $i : F^{ab}(A) \to Z$ by
  \begin{equation*}
    i(m_1a_1 + \cdots + m_na_n) = (m_1,\dots,m_n),
  \end{equation*}
  and let $f : F^{ab}(A) \to G$ be defined by $f = i\varphi$.
  Finally, let $j$ and $\iota$ be the standard inclusions into $F^{ab}(A)$ and $G$, respectively..
  the following diagram illustrates these morphisms:
  \[\begin{tikzcd}
      & & Z \ar[two heads]{dr}{\varphi} \\
      & F^{ab}(A) \ar{ur}{i} \ar[dashed]{rr}{\varphi_A} && G\\
      & A \ar{u}{j} \ar{urr}[swap]{\iota} &&
  \end{tikzcd}\]
  Define $\alpha_m$ to be $j(a_m)$, let $a = m_1a_1 + \cdots + m_na_n$, and let $\alpha=j(a)$.
  We then have:
  \begin{align*}
    (f\circ j)(a) &= f(\alpha)\\
    &= (\varphi\circ i)(m_1\alpha_1 + \cdots + m_n\alpha_n) \\
    &= \varphi(m_1, \dots, m_n) \\
    &= m_1\varphi(\beta_1) + \cdots + m_n\varphi(\beta_n) \\
    &= m_1\iota(a_1) + \cdots + m_n\iota(a_n) \\
    &= \iota(a)
  \end{align*}
  Since the morphism $\varphi_A$ is the only morphism that satisfies this property (by the universality of 
    $F^{ab}(A)$, we know that $f = \varphi_A$. 
    But $f$ (and hence $\varphi_A$) is surjective, giving us \\
    $G = \text{im}(\varphi_A) = \langle A\rangle$, as desired.
\end{solution}
\begin{problem}{6.9}
  Prove that every finitely generated subgroup of $\mathbb{Q}$ is cyclic.
  Prove that $\mathbb{Q}$ is not finitely generated.
\end{problem}
\begin{solution}
  First, I will state (but not prove) a lemma from number theory
  that will be useful in this proof:
  \begin{lemma}
    The diophantine equation
    \begin{equation*}
      a_1x_1 + \cdots + a_nx_n = d
    \end{equation*}
    has a solution if and only if $$\gcd(x_1,\dots,x_n)~|~d.$$
  \end{lemma}
  This actually follows quite easily from the case involving two terms, according to a post on stack exchange.
  \footnote{https://math.stackexchange.com/questions/145346/diophantine-equations-with-multiple-variables}
  Now, on to the problem:
  \begin{proposition}
    Let $r_1=\frac{p_1}{q_1}, \dots, r_n=\frac{p_n}{q_n}$ be (reduced) rational numbers.
    Then the group $G=\langle r_1,\dots,r_n\rangle$ is cyclic; in fact, we have
    \begin{equation*}
      M := \left\langle \frac{\gcd(P)}{\lcm{Q}}\right\rangle = G
    \end{equation*}
    where $P=\{p_1,\dots,p_n\}$ and $Q=\{q_1,\dots,q_n\}$.
  \end{proposition}
  \begin{sproof}
    First, we will prove that $M \subseteq G$ by proving that the generator of $M$ written above is in $G$.
    Consider the following equation:
    \begin{equation*}
      a_1r_1 + \cdots + a_nr_n = \frac{\gcd(P)}{\lcm{Q}}
    \end{equation*}
    If we let $c=\lcm{Q}$, this is equivalent to the equation
    \begin{equation*}
      a_1(cr_1) + \cdots + a_n(cr_n) = \gcd(P),
    \end{equation*}
    which, by Lemma 1, has a solution if and only if 
    \begin{equation*}
      \gcd(cr_1, \dots, cr_n)~|~\gcd(P).
    \end{equation*}
    This is true because each $cr_k$ is a multiple of $p_k$.
    Now that we know we can obtain a generator of $M$ from $G$ with coefficients $a_1,\dots,a_n$,
    we know we can obtain any element of $M$ by just multiplying the coefficients by some constant,
    so $M \subseteq G$. \\
    For the other direction, suppose we have some $a_1r_1+\cdots+a_nr_n\in G$.
    That this element is in $M$ is equivalent to the fact that following equation holds for some integer $m$:
    \begin{equation*}
      a_1r_1+\cdots+a_nr_n = m\frac{\gcd(P)}{\lcm{Q}}
    \end{equation*}
  multiplying by $\lcm{Q}$, we have (for $c=\lcm{Q})$):
    \begin{equation*}
      a_1(cr_1) + \cdots + a_n(cr_n) = m\gcd(P)
    \end{equation*}
    which clearly holds due to the reasoning above. 
    Additionally, we know that dividing the left side by $\gcd(P)$ would yield an integer since
    each $cr_k$ is a multiple of $p_k$, so there does exist an $m\in\mathbb{Z}$ such that the equation holds.
    Hence $G \subseteq M$, completing the proof.
  \end{sproof}
  The next part of the question is simple.
  Due to the proof above, $\mathbb{Q}$ being finitely generated would imply that it is cyclic.
  Suppose this is so, and designate a generator $g\in\mathbb{Q}$.
  We know, by the density of $\mathbb{Q}$ in $\mathbb{R}$,
  that there exists a rational strictly between any $kg$ and $(k+1)g$,
  meaning that there are rationals not in $\langle g\rangle$; therefore $\mathbb{Q}$ is not cyclic, and hence
  not finitely generated.
\end{solution}
\end{document}
