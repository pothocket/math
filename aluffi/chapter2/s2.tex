%&pdflatex 
\documentclass[12pt]{article} 
\usepackage[margin=1in]{geometry} 
\usepackage{amsmath,amsthm,amssymb,amsfonts,tikz-cd} 

\newenvironment{problem}[2][Problem]{\begin{trivlist}
\item[\hskip \labelsep {\bfseries #1}\hskip \labelsep {\bfseries #2.}]}{\end{trivlist}}

\newenvironment{proposition}[1][Proposition]{\begin{trivlist}
\item[\hskip \labelsep {\bfseries #1.}]}{\end{trivlist}}

\newenvironment{definition}[1][Definition]{\begin{trivlist}
\item[\hskip \labelsep {\bfseries #1.}]}{\end{trivlist}}

\newcommand{\til}{\char`\~}
\newcommand{\bs}{\textbackslash} 

\newcommand{\catname}[1]{\normalfont\textbf{#1}}
\newcommand{\catsup}[2]{\normalfont\textbf{#1}^{#2}}
\newcommand{\catsub}[2]{\normalfont\textbf{#1}_{#2}}

\newcommand{\Hom}{\text{Hom}}
\newcommand{\Homc}[2]{\Hom_{\catname{#1}}(#2)}

\newcommand{\Obj}{\text{Obj}}
\newcommand{\Objc}[1]{\text{Obj}(\catname{#1})}
\newcommand{\Aut}{\text{Aut}}
\newcommand{\End}{\text{End}}

\newcommand{\id}{\text{id}}

\newcommand{\lcm}[1]{\text{lcm}(#1)}

\newenvironment{solution}
  {\renewcommand\qedsymbol{$\blacksquare$}\begin{proof}[Solution]}
{\end{proof}}

\newenvironment{sproof}{
  \renewcommand\qedsymbol{$\square$}
  \begin{proof}
  }{
  \end{proof}
}

\begin{document}

\title{Algebra: Chapter 0 Exercises\\ \large Chapter 2, Section 2}
\author{David Melendez}
\maketitle

\begin{problem}{2.2}
  If $d\leq n$, then $S_n$ contains elements of order $d$.
\end{problem}
\begin{proposition}
  Let $c_d$, called a \textit{d-cycle} in $S_n$, be defined as follows:
  \begin{equation*}
    c_d(m) = 
    \begin{cases}
      d   & m = 1 \\
      d-1 & 1 < m \leq d \\
      m   & m > d
    \end{cases}
  \end{equation*}
  For example, if we're working in $S_6$, then 
  $c_4 = \begin{pmatrix} 1&2&3&4&5&6 \\ 4&1&2&3&5&6 \end{pmatrix}$. \\
  Then $|c_d| = d$ for $1 \leq d \leq n$.
\end{proposition}
\begin{proof}
  Note that if $0 < k < d$, then 
  $c_d^k(d) = d-k \geq 1$ 
  ($c_d^k$ never ``reaches'' the point at which it cycles from 1 to $d$ since $k<d$), 
  so $|c_d| \geq d$. Then, we have, for $m\leq d$,
  \begin{align*}
    c_d^d(m) &= (c_d^m \cdot c_d^{d-m})(m)\\
    &= c_d^{d-m}(d) \\
    &= d - (d - m) \\
    &= m
  \end{align*}
  Clearly $c_d^d(m) = m$ if $m>d$, so $c_d^d$ is the identity, as desired.
\end{proof}

\begin{problem}{2.5}
  Describe generators and relations for all dihedral groups $D_{2n}$.
\end{problem}
\begin{solution}
  We will define the dihedral group $D_{2n}$ as follows:
  \begin{equation*}
    D_{2n} = \left<x, y\ |\ x^2 = y^n = (xy)^2 = e\right>
  \end{equation*}
  \begin{proposition}
    With this definition of $D_{2n}$, every combination 
    $x^{i_1}y^{i_2}x^{i_4}y^{i_5}\cdots$ equals $x^iy^j$ 
    for some $0 \leq i \leq 1, 0 \leq j < n$.
  \end{proposition}
  \begin{proof}
    We will use induction on $m$, the number of elements we're composing. \\
    The cases for $0 \leq m \leq 2$ are obvious. \\
    Suppose this reduction holds for $m$. 
    Then, if $m$ is odd, we have
    \begin{align*}
      \left(x^{k_1}y^{k_2}\cdots x^{k_m}\right)y^{k_{m+1}}
      &= x^iy^jy^{k_{m+1}} \\
      &= x^iy^{j+k_{m+1}}
    \end{align*}
    The case where $m$ is even is more interesting. First, we will establish
    the following based off of the third relation:
    \begin{alignat*}{2}
      (xy)^2 = e &\implies xyxy &&= e \\
      &\implies x(yxy) &&= e \\
      &\implies x^{-1} &&=  yxy \\
      &\implies yx &&= x^{-1}y^{-1} \\
      &&&= xy^{n-1}
    \end{alignat*}
    Now, suppose $m$ is even. We then have, 
    with $\leq i \leq 1$, $0\leq j < n$, and $0\leq k \leq 1$:
    \begin{align*}
      \left(x^{k_1}y^{k_2}\cdots x^{k_{m-1}}y^{k_m}\right)x^{k_{m+1}}
      &= x^iy^jx^k \\
    \end{align*}
    Since every other case is trivial, we will assume
    $0 \neq j \neq n$ and $k = 1$. 
    Additionally, we will assume wlog that $j < n$.
    Then, we have
    \begin{align*}
      x^iy^jx^k &= x^iy^jx\\
      &= x^iy^{j-1}(yx)\\
      &= x^iy^{j-1}xy^{n-1}\\
      &= x^iy^{j-2}xy^{2n-2}\\
      &= x^iy^{j-2}xy^{n-2}\\
      &= x^iy^{j-3}xy^{n-3}\\
      &= \cdots \\
      &= x^iy^0xy^{n-j} \\
      &= x^{i+1}y^{n-j} \\
    \end{align*}
    as desired.
  \end{proof}
\end{solution}

\begin{problem}{2.10}
  Prove that $\mathbb{Z}/n\mathbb{Z}$ consists of precisely $n$ elements.
\end{problem}
\begin{proposition}
  $\mathbb{Z}/n\mathbb{Z}$ consists exactly of the elements $S= \{[0]_n, [1]_n, \dots [n-1]_n\}$.
\end{proposition}
\begin{proof}
  First, suppose $[a]_n, [b]_n \in S$ and (without loss of generality) $a < b$. 
  Then, since $b-a<n$, we have $n \nmid b-a$, and so $[a]_n$ and $[b]_n$ (and hence all elements of S) are distinct.\\
  Now, suppose $c \geq n$. 
  Then we have, for some positive integers $q \geq 1$ and $0 \leq r < n$, $c = qn + r$. 
  Hence $c-r = qn$, so $c \equiv r \mod n$. In other words, $[c]_n = [r]_n$ with $r<n$, completing the proof
  that $\mathbb{Z}/n\mathbb{Z} = S$.
\end{proof}

\begin{problem}{2.11}
  The square of every odd integer is congruent to 1 modulo 8.
\end{problem}
\begin{solution}
  Let $n \geq 0$ be an integer. We will prove that $(2n+1)^2 \equiv 1 \mod 8$. \\
  Note that $4x\equiv 0\mod 8$ if $x$ is even, and that
  \begin{align*}
    (2n+1)^2 &= 4n^2 + 4n + 1\\
    &= 4n(n+1) + 1
  \end{align*}
  If $n=2m+1$ is odd, then we have
  \begin{align*}
    n(n+1) &= (2m+1)(2m+2) \\
    &= 2(2m+1)(m+1)\\
    &\equiv 0\mod 2
  \end{align*}
  Similarly, if $n=2m$ is even, then we have
  \begin{align*}
    n(n+1) &= 2m(2m+1) \\
    &\equiv 0\mod 2
  \end{align*}
  Thus $n(n+1)$ is even, giving us $4n(n+1)\equiv 0\mod 8$, and hence 
  \begin{align*}
  (2n+1)^2 &= 4n(n+1) + 1 \\
  &\equiv 1\mod 8
  \end{align*}
\end{solution}

\begin{problem}{2.12}
  There are no nonzero integers $a,b,c$ such that $a^2+b^2=3c^2$.
\end{problem}
\begin{solution}
  I'll write this one down later. 
  Essentially, you work in $\mathbb{Z}/4\mathbb{Z}$ (as given in the text as a hint) split the problem into cases, and deduce an even-odd contradiction between $a^2+b^2$ and $3c^2$.
\end{solution}
\begin{problem}{2.13}
  There exist integers $a$ and $b$ such that $$am+bn=1$$ iff $\gcd(m,n)=1$.
\end{problem}
\begin{solution}
  First suppose $\gcd(m,n)=1$. 
  By Corollary 2.5, we know that $[m]_n$ generates $\mathbb{Z}/n\mathbb{Z}$, and hence 
  $$am\equiv 1\mod n$$ for some integer $a$.
  It then follows that $$am = bn + 1$$ for some integer $b$, and so $$am-bn=1$$ as desired.\\
  For the proof in the other direction, suppose there exist integers $a$ and $b$ such that 
  $$am+bn=1$$
  Then we have $$am = 1-bn$$. 
  Suppose, for the sake of contradiction, that $\gcd(m,n)=d>1$. We then have
  \begin{align*}
    \frac{am}{d} &= \frac{1}{d} - \frac{bn}{d} \\
    \frac{am}{d} + \frac{bn}{d} &= \frac{1}{d}
  \end{align*}
  The LHS is an integer since $d|m$ and $d|m$, but the (nonzero) RHS is not since $d>1$. Absurd!
\end{solution}

\begin{problem}{2.14}
  Show that multiplication on $\mathbb{Z}/n\mathbb{Z}$ is a well-defined operation.
\end{problem}
\begin{solution}
  First, we shall prove a useful intuition regarding modulus.
  \begin{proposition}
    If $a\equiv b\mod n$, then there exist integers $k_1,k_2\geq0$ and $r$ with $0\leq r<n$ such that
    \begin{align*}
      a&=k_1n+r\\
      b&=k_2n+r\\
    \end{align*}
  \end{proposition}
  \begin{proof}
    Recall that $\mathbb{Z}/n\mathbb{Z}$ consists entirely of the equivalence classes of the numbers in the set
    of nonnegative integers up to but not including $n$. Thus, if $a\equiv b\mod n$, there exists an
    $r$ with $0\leq r<n$ such that
    \begin{align*}
      n&|a-r\\
      n&|b-r
    \end{align*}
    Hence, we have, for some nonnegative integers $k_1, k_2$:
    \begin{align*}
      k_1n &= a-r\\
      k_2n &= b-r
    \end{align*}
    Therefore
    \begin{align*}
      a = k_1n + r\\
      b = k_2n + r
    \end{align*}
    as desired.
  \end{proof}
  With this, we can show that multiplication on $\mathbb{Z}/n\mathbb{Z}$ is well-defined.
  \begin{proposition}
    If $a\equiv a'\mod n$ and $b\equiv b'\mod n$, then $ab\equiv a'b'\mod n$.
  \end{proposition}
  \begin{proof}
    Suppose $a\equiv a'\mod n$ and $b\equiv b'\mod n$. 
    Then, by the lemma above, we have:
    \begin{align*}
      a = k_1n + r \\
      a' = k_2n + r \\
      b = \ell_1n+s \\
      b' = \ell_2n+s \\
    \end{align*}
    Next, consider the product $ab$:
    \begin{align*}
      ab &= (k_1n+r)(\ell_1n+s) \\
      &= k_1\ell_1n^2 + k_1sn + \ell_1rn + rs \\
      &= (k_1\ell_1n + k_1s + \ell_1r)n + rs \\
      &\equiv rs \mod n 
    \end{align*}
    Similarly, for $a'b'$, 
    \begin{align*}
      a'b' &= (k_2n+r)(\ell_2n+s) \\
      &= k_2\ell_2n^2 + k_2sn + \ell_2rn + rs \\
      &= (k_2\ell_2n + k_2s + \ell_2r)n + rs \\
      &\equiv rs \mod n 
    \end{align*}
    Hence $ab\equiv a'b'\mod n$ by transitivity.
  \end{proof}
\end{solution}
\end{document}
