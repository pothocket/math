%&pdflatex 
\documentclass[12pt]{article} 
\usepackage[margin=1in]{geometry} 
\usepackage{amsmath,amsthm,amssymb,amsfonts,tikz-cd} 

\newenvironment{problem}[2][Problem]{\begin{trivlist}
\item[\hskip \labelsep {\bfseries #1}\hskip \labelsep {\bfseries #2.}]}{\end{trivlist}}

\newenvironment{proposition}[1][Proposition]{\begin{trivlist}
\item[\hskip \labelsep {\bfseries #1.}]}{\end{trivlist}}

\newenvironment{definition}[1][Definition]{\begin{trivlist}
\item[\hskip \labelsep {\bfseries #1.}]}{\end{trivlist}}

\newcommand{\til}{\char`\~}
\newcommand{\bs}{\textbackslash} 

\newcommand{\catname}[1]{\normalfont\textbf{#1}}
\newcommand{\catsup}[2]{\normalfont\textbf{#1}^{#2}}
\newcommand{\catsub}[2]{\normalfont\textbf{#1}_{#2}}

\newcommand{\Hom}{\text{Hom}}
\newcommand{\Homc}[2]{\Hom_{\catname{#1}}(#2)}

\newcommand{\Obj}{\text{Obj}}
\newcommand{\Objc}[1]{\text{Obj}(\catname{#1})}
\newcommand{\Aut}{\text{Aut}}
\newcommand{\End}{\text{End}}

\newcommand{\id}{\text{id}}

\newcommand{\lcm}[1]{\text{lcm}(#1)}

\newenvironment{solution}
  {\renewcommand\qedsymbol{$\blacksquare$}\begin{proof}[Solution]}
{\end{proof}}

\newenvironment{sproof}{
  \renewcommand\qedsymbol{$\square$}
  \begin{proof}
  }{
  \end{proof}
}

\begin{document}

\title{Algebra: Chapter 0 Exercises\\ \large Chapter 2, Section 4}
\author{David Melendez}
\maketitle

\begin{problem}{4.9}
  Prove that if $m,n$ are positive integers such that $\gcd(m,n)=1$, then \\
  $C_{mn} \cong C_m\times C_m$.
\end{problem}
\begin{solution}
  We know that the order of $C_m\times C_n$ is $mn$, so we just have to prove that $C_m\times C_n$ has an
  element of order $mn$.
  \begin{proposition}
    $|([1]_m,[1]_n)| = mn$
  \end{proposition}
  \begin{sproof}
    We're looking for the smallest $k$ such that $k\equiv 0\mod m$ and $k\equiv 0\mod n$.
    By definition, we have $k = \lcm{m,n}=mn$.
  \end{sproof}
\end{solution}

\iffalse
\begin{problem}{4.10}
  Let $p\neq q$ be odd prime integers; show that $(\mathbb{Z}/pq\mathbb{Z})^*$ is not cyclic.
\end{problem}
\begin{proof}
  Let $N$ be the order of $G=(\mathbb{Z}/pq\mathbb{Z})^*$. 
  We know, from the properties of Euler's totient function (TODO: prove this myself?), that
  \begin{align*}
    N &= \phi(p)\phi(q) \\
    &= (p-1)(q-1)\\
    &= pq - p - q + 1 \\
    &= pq + 1 - (p+q)
  \end{align*}
  Suppose for the sake of contradiction that $G$ is cyclic, 
  and hence has a generating element $g$ of order $N$.
  We then have:
  \begin{align*}
    g^N &= g^{pq+1-(p+q)} \\
    &= g^{pq+1}g^{-(p+q)} \\
      &= g^0
  \end{align*}
  It then follows that $pq+1=p+q$. \\
  But there is a problem. Without loss of generality, let $2<q<p$. 
  We then have:
  \begin{align*}
    pq+1 &> pq \\
    &= p+(q-1)p \\
    &> p+q
  \end{align*}
  A contradiction. $G$ is not cyclic.
\end{proof}
\fi

\begin{problem}{4.11}
  Given that $x^d=1$ can have at most $d$ solutions in $(\mathbb{Z}/p\mathbb{Z})$ for prime p,
    prove that the multiplicative group $G=(\mathbb{Z}/p\mathbb{Z})^*$ is cyclic.
    (Hint: let $g\in G$ be an element of maximal order; show that $h^{|g|}=1$ for all $h\in G$)
\end{problem}
\begin{solution}
    Let $g\in G$ be an element of maximal order.
    By exercise 1.15, we know that $|h|$ divides $|g|$ for all $h\in G$, so $h^{|g|}=1$.
    Since $h^{|g|}=1$ for all $h\in G$, there are at least $|G|$ solutions to the equation
    $x^d=1$ in $\mathbb{Z}/p\mathbb{Z}$. It then follows that $|G|\leq|g|$ by the given theorem in the problem,
    so $|G| = |g|$ and therefore $G$ is cyclic.
\end{solution}
\end{document}
