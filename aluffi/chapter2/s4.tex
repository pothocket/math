%&pdflatex 
\documentclass[12pt]{article} 
\usepackage[margin=1in]{geometry} 
\usepackage{amsmath,amsthm,amssymb,amsfonts,tikz-cd} 

\newenvironment{problem}[2][Problem]{\begin{trivlist}
\item[\hskip \labelsep {\bfseries #1}\hskip \labelsep {\bfseries #2.}]}{\end{trivlist}}

\newenvironment{proposition}[1][Proposition]{\begin{trivlist}
\item[\hskip \labelsep {\bfseries #1.}]}{\end{trivlist}}

\newenvironment{definition}[1][Definition]{\begin{trivlist}
\item[\hskip \labelsep {\bfseries #1.}]}{\end{trivlist}}

\newcommand{\til}{\char`\~}
\newcommand{\bs}{\textbackslash} 

\newcommand{\catname}[1]{\normalfont\textbf{#1}}
\newcommand{\catsup}[2]{\normalfont\textbf{#1}^{#2}}
\newcommand{\catsub}[2]{\normalfont\textbf{#1}_{#2}}

\newcommand{\Hom}{\text{Hom}}
\newcommand{\Homc}[2]{\Hom_{\catname{#1}}(#2)}

\newcommand{\Obj}{\text{Obj}}
\newcommand{\Objc}[1]{\text{Obj}(\catname{#1})}
\newcommand{\Aut}{\text{Aut}}
\newcommand{\End}{\text{End}}

\newcommand{\id}{\text{id}}

\newcommand{\lcm}[1]{\text{lcm}(#1)}

\newenvironment{solution}
  {\renewcommand\qedsymbol{$\blacksquare$}\begin{proof}[Solution]}
{\end{proof}}

\newenvironment{sproof}{
  \renewcommand\qedsymbol{$\square$}
  \begin{proof}
  }{
  \end{proof}
}

\begin{document}

\title{Algebra: Chapter 0 Exercises\\ \large Chapter 2, Section 4}
\author{David Melendez}
\maketitle

\begin{problem}{4.9}
  Prove that if $m,n$ are positive integers such that $\gcd(m,n)=1$, then \\
  $C_{mn} \cong C_m\times C_m$.
\end{problem}
\begin{solution}
  We know that the order of $C_m\times C_n$ is $mn$, so we just have to prove that $C_m\times C_n$ has an
  element of order $mn$.
  \begin{proposition}
    $|([1]_m,[1]_n)| = mn$
  \end{proposition}
  \begin{sproof}
    We're looking for the smallest $k$ such that $k\equiv 0\mod m$ and $k\equiv 0\mod n$.
    By definition, we have $k = \lcm{m,n}=mn$.
  \end{sproof}
\end{solution}

\iffalse
\begin{problem}{4.10}
  Let $p\neq q$ be odd prime integers; show that $(\mathbb{Z}/pq\mathbb{Z})^*$ is not cyclic.
\end{problem}
\begin{proof}
  Let $N$ be the order of $G=(\mathbb{Z}/pq\mathbb{Z})^*$. 
  We know, from the properties of Euler's totient function (TODO: prove this myself?), that
  \begin{align*}
    N &= \phi(p)\phi(q) \\
    &= (p-1)(q-1)\\
    &= pq - p - q + 1 \\
    &= pq + 1 - (p+q)
  \end{align*}
  Suppose for the sake of contradiction that $G$ is cyclic, 
  and hence has a generating element $g$ of order $N$.
  We then have:
  \begin{align*}
    g^N &= g^{pq+1-(p+q)} \\
    &= g^{pq+1}g^{-(p+q)} \\
      &= g^0
  \end{align*}
  It then follows that $pq+1=p+q$. \\
  But there is a problem. Without loss of generality, let $2<q<p$. 
  We then have:
  \begin{align*}
    pq+1 &> pq \\
    &= p+(q-1)p \\
    &> p+q
  \end{align*}
  A contradiction. $G$ is not cyclic.
\end{proof}
\fi

\begin{problem}{4.11}
  Given that $x^d=1$ can have at most $d$ solutions in $(\mathbb{Z}/p\mathbb{Z})$ for prime p,
    prove that the multiplicative group $G=(\mathbb{Z}/p\mathbb{Z})^*$ is cyclic.
    (Hint: let $g\in G$ be an element of maximal order; show that $h^{|g|}=1$ for all $h\in G$)
\end{problem}
\begin{solution}
    Let $g\in G$ be an element of maximal order.
    By exercise 1.15, we know that $|h|$ divides $|g|$ for all $h\in G$, so $h^{|g|}=1$.
    Since $h^{|g|}=1$ for all $h\in G$, there are at least $|G|$ solutions to the equation
    $x^d=1$ in $\mathbb{Z}/p\mathbb{Z}$. It then follows that $|G|\leq|g|$ by the given theorem in the problem,
    so $|G| = |g|$ and therefore $G$ is cyclic.
\end{solution}

\begin{problem}{4.12}
  Compute the order of $[9]_{31}$ in the group $(\mathbb{Z}/31\mathbb{Z})^*$ 
  and determine if $x^3-9=0$ has any solutions in $\mathbb{Z}/31\mathbb{Z}$.
\end{problem}
\begin{solution}
  The order of $[9]_{31}$ in $(\mathbb{Z}/31\mathbb{Z})^*$ is 15.
  \begin{proposition}
    The equation $x^3-9=0$ has no solutions in $\mathbb{Z}/31\mathbb{Z}$.
  \end{proposition}
  \begin{sproof}
    Suppose $x\in\mathbb{Z}/31\mathbb{Z}$, and $$x^3-9\equiv0\mod 31.$$
    We then have $$x^3\equiv9\mod31,$$ and so
    $$x^{45}\equiv1\mod31.$$
    This tells us that $|x|$ divides $45$ and so the order of $x$ is either 3, 5, 9, 15, or 45.
    It cannot equal 45 because the order of $(\mathbb{Z}/31\mathbb{Z})^*$ is less than 45,
    and it cannot be 3, 9, or 15 because this would contradict the order of $[9]_{31}$ being 13.
    Hence, $|[x]_{31}$ must equal 5.
    However, this tells us that
    \begin{align*}
      9^5 &\equiv (x^3)^5 \mod 5 \\
      &\equiv (x^5)^3 \mod 5 \\
      &\equiv 1 \mod 5,
    \end{align*}
    which contradicts the order of $[9]_{31}$ in $(\mathbb{Z}/31\mathbb{Z})*$ being 45.
  \end{sproof}
\end{solution}

\begin{problem}{4.14}
  Prove that the order of the group of automorphisms of a cyclic group $C_n$ is the number of positive
  integers $r<n$ that are relatively prime to n.
\end{problem}
\begin{solution}
  First, we will prove that the homomorphisms on a cyclic group are uniquely determined by their
  values at a generator.
  \begin{proposition}
    Let $\varphi_1$ and $\varphi_2$ be homomorphisms on the cyclic group $C_n$, and let $[m]_n\in C_n$
    be a generator. Then $\varphi_1 = \varphi_2$ if and only if $\varphi_1([m]_n) = \varphi_2([m]_n)$.
  \end{proposition}
  \begin{sproof}
    One direction is obvious. 
    For the other direction, let $[m]_n\in C_n$ be a generator and let 
    $\varphi_1, \varphi_2\in \Aut(C_n)$ be such that
    $\varphi_1([m]_n) = \varphi_2([m]_n)$. 
    Since $\varphi_1$ and $\varphi_2$ are homomorphisms, we have
    \begin{align*}
      \varphi_1(k[m]_n) &= k\varphi_1([m]_n) \\
      &= k\varphi_2([m]_n) \\
      &= \varphi_2(k[m]_n)
    \end{align*}
    for $0\leq k < n$; that is, $\varphi_1=\varphi_2$.
  \end{sproof}
  We know that a class $[m]_n$ generates $C_n$ if and only if $\gcd(m,n)=1$, so all we have to do is prove that
  an endomorphism $\varphi$ is iso if and only if it sends a generator to a generator.
  \begin{proposition}
    Let $\varphi$ be an endomorphism on the cyclic group $C_n$ and let $[m]_n$ be a generator. 
    Then $\varphi$ is an automorphism if and only if $\varphi([m]_n)$ generates $C_n$.
  \end{proposition}
  \begin{sproof}
    First suppose $\varphi$ is an automorphism.
    Then, since $\varphi$ is surjective, we know that for every $x\in C_n$, there exists a $k$ such that
    \begin{align*}
      x &= \varphi(k[m]_n) \\
      &= k\varphi([m]_n).
    \end{align*}
    Hence $\varphi([m]_n)$ generates $C_n$, completing the proof in one direction.\\
    Next, suppose $\varphi([m]_n)$ generates $C_n$. 
    It is clear, then, that $\varphi$ is surjective. 
    To prove that $\varphi$ is injective, we will show that $\ker \varphi = {[1]_n}$.\\
    Suppose $\varphi(x) = [1]_n$. 
    Since $[m]_n$ generates $C_n$, we then have, for some $k$, \\
    \begin{equation*}
      \varphi(k[m]_n) = [1]_n.
    \end{equation*}
    It then follows that 
    \begin{equation*}
      k\varphi([m]_n)=[1]_n,
    \end{equation*}
    and hence $k=|C_n|$, since $\varphi([m]_n)$ is a generator.
    However, since $[m]_n$ is also a generator, it then follows that
    $k[m]_n=[1]_n$, and so $x=[1]_n$, completing the proof.
  \end{sproof}
  Having established all this, we know that an endomorphism $\varphi\in C_n$ is an automorphism
  if and only if $\varphi([1]_n) = [k]_n$ generates $C_n$, 
  and we know that $[k]_n$ generates $C_n$ if and only if $\gcd(k,n)=1$,
  so it then follows that there are precisely as many automorphisms in $C_n$ as there are
  integers less than and coprime to $n$.
\end{solution}
\end{document}
