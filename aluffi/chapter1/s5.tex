%&pdflatex 
\documentclass[12pt]{article} 
\usepackage[margin=1in]{geometry} 
\usepackage{amsmath,amsthm,amssymb,amsfonts,tikz-cd} 

\newenvironment{problem}[2][Problem]{\begin{trivlist}
\item[\hskip \labelsep {\bfseries #1}\hskip \labelsep {\bfseries #2.}]}{\end{trivlist}}

\newenvironment{proposition}[1][Proposition]{\begin{trivlist}
\item[\hskip \labelsep {\bfseries #1.}]}{\end{trivlist}}

\newenvironment{definition}[1][Definition]{\begin{trivlist}
\item[\hskip \labelsep {\bfseries #1.}]}{\end{trivlist}}

\newcommand{\til}{\char`\~}
\newcommand{\bs}{\textbackslash} 

\newcommand{\catname}[1]{\normalfont\textbf{#1}}
\newcommand{\catsup}[2]{\normalfont\textbf{#1}^{#2}}
\newcommand{\catsub}[2]{\normalfont\textbf{#1}_{#2}}


\newcommand{\Hom}{\text{Hom}}
\newcommand{\Homc}[2]{\Hom_{\catname{#1}}(#2)}

\newcommand{\Obj}{\text{Obj}}
\newcommand{\Objc}[1]{\text{Obj}(\catname{#1})}
\newcommand{\Aut}{\text{Aut}}
\newcommand{\End}{\text{End}}

\newcommand{\id}{\text{id}}

\newenvironment{solution}
  {\renewcommand\qedsymbol{$\blacksquare$}\begin{proof}[Solution]}
{\end{proof}}

\newenvironment{sproof}{
  \renewcommand\qedsymbol{$\square$}
  \begin{proof}
  }{
  \end{proof}
}

\begin{document}

\title{Algebra: Chapter 0 Exercises\\ \large Chapter 1, Section 5}
\author{David Melendez}
\maketitle

\begin{problem}{5.1}
  A final object in a category \catname{C} is initial in the opposite category $\catname{C}^{op}$.
\end{problem}
\begin{proof}
  Let $T \in \Objc{C}$ be initial and $Z \in \Objc{C}$ be any object. 
  Since $\Homc{C}{T, Z}$ is a singleton, 
  $\Hom_{\catname{C}^{op}}(Z, T) = \Homc{C}{T, Z}$ 
  is also a singleton, making $T$ final in $\catname{C}^{op}$
\end{proof}

\begin{problem}{5.2}
  The empty set $\emptyset$ is the \textit{unique} initial object in \catname{Set}.
\end{problem}
\begin{proof}
  Recall that the number of morphisms between any two sets $A$ and $B$ is given by $B^A$. 
  Thus, the problem of finding the size of an initial object in \catname{Set} boils down to solving $|Z|^{|I|}=1$ for $|I|$, where $Z$ is an arbitrary set.
  The only solution to this is $|I|=0$, which is only satisfied by the the null set, given that $\emptyset^\emptyset = 0$.
\end{proof}

\begin{problem}{5.3}
  Final objects are unique up to isomorphism.
\end{problem}
\begin{proof}
  Let \catname{C} be a category and $T_1, T_2\in \Objc{C}$ be final. Because $T_1$ and $T_2$ are final, we have, for all $Z\in\Homc{C}{Z, T}$:
  \begin{align*}
    \End(T_1) &= \{\id_{T_1}\}\\
    \End(T_2) &= \{\id_{T_2}\}\\
    \Hom(T_1, T_2) &= \{f\} \text{ for some } f\\
    \Hom(T_2, T_1) &= \{g\} \text{ for some } g\\
  \end{align*}
  We then have $fg$ = $\id_{T_2}$ and $gf$ = $\id_{T_2}$, giving us $T_1 \cong T_2$.
\end{proof}

\newpage

\begin{problem}{5.4}
  What are the initial and final objects in the category of 'pointed sets' (Example 3.8)? Are they unique?
\end{problem}
\begin{solution}
  For the sake of completeness, a description of this category $\catsup{Set}{*}$ will be included.
  \begin{definition}
    Define $\catsup{Set}{*}$ as follows:\\
    Objects in $\catsup{Set}{*}$ are morphisms $f : \{*\} \to A$ where $A$ is any set, denoted $(f, A)$. \\
    Morphisms $\sigma \in \Hom_{\catsup{Set}{*}}( (f_A, A), (f_B, B))$ are commutative diagrams: 
    \[\begin{tikzcd}
        A \ar{rr}{\sigma} && B \\
        & \{*\} \ar{ul}{f_A} \ar{ur}[swap]{f_B}
    \end{tikzcd}\]
  \end{definition}
  Now, we will answer the question.
  \begin{proposition}
    Initial and final objects in $\catsup{Set}*$ are objects $(i, T)$, where $T$ is a singleton and $i$ is the unique function that maps from $\{*\}$ to $T$.
  \end{proposition}
  \begin{sproof}
    We will prove that the objects described are both initial and final in $\catsup{Set}*$.
    \begin{enumerate}
      \item Initial: \\
        Consider the object described above. 
        Then, a morphism from this object to another object 
        $(f, B)\in \Obj(\catsup{Set}*)$
        is given by the following commutative diagram:
        \[\begin{tikzcd}
          T \ar{rr}{\sigma} && B \\
          & \{*\} \ar{ul}{i} \ar{ur}[swap]{f}
        \end{tikzcd}\]
        The only choice of $\sigma$ that makes this diagram commute is defined by $\sigma(i(*)) = f(*)$ (as $T$ is a singleton), so $T$ is initial.

      \item Final: 
        Consider the object described above. 
        Then, a morphism to this object from another object 
        $(f, B)\in \Obj(\catsup{Set}*)$
        is given by the following commutative diagram:
        \[\begin{tikzcd}
          A \ar{rr}{\tau} && T \\
          & \{*\} \ar{ul}{f} \ar{ur}[swap]{i}
        \end{tikzcd}\]
        The only choice of $\tau$ that makes this diagram commute is defined by $\tau(f(*)) = i(*)$ (as $T$ is a singleton), so $T$ is final.
    \end{enumerate}
  \end{sproof}
\end{solution}

\newpage

\begin{problem}{5.5}
  What are the final objects in the category considered in 5.3?
\end{problem}
\begin{solution}
  Final objects in this category (let's denote it \catname{C}) are singletons.
  \begin{sproof}
    Let $\{*\}$ be a singleton.
    Then morphisms $\tau$ from $(Z, \varphi) \in \Objc{C}$ to $\{*\}$ are commutative diagrams:
    \[\begin{tikzcd}
        Z \ar{rr}{\tau} && \{*\} \\
        & A \ar{ul}{\varphi} \ar{ur}[swap]{i}
    \end{tikzcd}\]
    Since $\{*\}$ is a singleton, the only $\tau$ that satisfies
    $\tau(\varphi(a)) = \iota(a)$ is defined by $\tau(z) = *$ for all $z\in Z$.
    Thus $(\{*\}, \tau)$ is final in \catname{C}.
  \end{sproof}
\end{solution}

\begin{problem}{5.6}
  Consider the category corresponding to endowing (as in Example 3.3) the set $\mathbb{Z}^+$ with the divisibility relation. 
  Thus there is exactly one morphism $d \to m$ in this category iff $d$ divides $m$ without remainder; there is no morphism between $d$ and $m$ otherwise.
  Show that this category has products and coproducts. What are their 'conventional' names?
\end{problem}
\begin{solution}
  Let \catname{Div} denote this category and $a, b \in \Objc{Div}$ be objects.
  That \catname{Div} has products means the following:\\
  For every $a, b \in \Objc{Div}$, 
  there exists an object $a\times b \in \Objc{Div}$ such that
  every $(x,f_a,f_b) \in \Obj(\catsup{Div}{a,b})$
  admits a unique morphism $\sigma \in \Homc{Div}{x, a\times b}$
  that makes the following diagram commute:
  \[\begin{tikzcd}
    &&&a\\
    &x \ar[bend left]{urr}{f_a} \ar{r}{\sigma} \ar[bend right]{drr}[swap]{f_b}
    &a\times b \ar{ur}{\pi_a} \ar{dr}[swap]{\pi_b}&\\
    &&&b
  \end{tikzcd}\]
  Meaning, for every pair of integers $(a,b)$, there exists a unique
  factor of a and b, $a\times b$, such that 
  every common factor of $a$ and $b$ divides $a\times b$.
  This is necessarily the greatest common factor of $a$ and $b$, since
  it must be a multiple of every common factor of $a$ and $b$, and 
  $\text{gcf}(a,b)$ is the product of the common factors of $a$ and $b$.
\end{solution}
\end{document}
