%&pdflatex 
\documentclass[12pt]{article} 
\usepackage[margin=1in]{geometry} 
\usepackage{amsmath,amsthm,amssymb,amsfonts,tikz-cd} 
\newenvironment{problem}[2][Problem]{\begin{trivlist}
\item[\hskip \labelsep {\bfseries #1}\hskip \labelsep {\bfseries #2.}]}{\end{trivlist}}

\newenvironment{proposition}[1][Proposition]{\begin{trivlist}
\item[\hskip \labelsep {\bfseries #1.}]}{\end{trivlist}}

\newcommand{\til}{\char`\~}
\newcommand{\catname}[1]{\normalfont\textbf{#1}}
\newcommand{\Hom}{\text{Hom}}
\newcommand{\Objj}{\text{Obj}}
\newcommand{\Obj}[1]{\text{Obj}(\catname{C})}
\newcommand{\Aut}{\text{Aut}}

\newenvironment{solution}
  {\renewcommand\qedsymbol{$\blacksquare$}\begin{proof}[Solution]}
{\end{proof}}

\newenvironment{sproof}{%
  \renewcommand\qedsymbol{$\square$}
  \begin{proof}
  }{
  \end{proof}
}

\begin{document}

\title{Algebra: Chapter 0 Exercises\\ \large Chapter 1, Section 4}
\author{David Melendez}
\maketitle

\begin{problem}{4.1}
    Composition is defined for \textit{two} morphisms. 
    If more than two morphisms are given, e.g.:

    \[\begin{tikzcd}
        A \ar[r,"f"] & B \ar[r,"g"] & C \ar[r,"h"] & D \ar[r,"i"] & E
    \end{tikzcd}\]

    then one may compose them in several ways, for example:
    \begin{equation*}
        (ih)(gf),\,\,\,\, (i(hg))f,\,\,\,\, i((hg)f),\,\,\,\, etc.
    \end{equation*}
    so that at every step one is only composing two morphisms. 
    Prove that the result of any such nested composition is independent of the placement of the parentheses.
\end{problem}
\begin{solution}
  Let $Z_m \in \text{Obj}(C)$ and $f_m \in \text{Hom}{(Z_{m+1}, Z_{m})}$ for every $m \in \mathbb{N}$. 
  Let $n$ be the number of morphisms we're composing.
  We will use induction on $n$.
  \\\\Base case: Suppose $n=3$. 
  Then, since $C$ is a category, we have $f_1(f_2f_3)$ = $(f_1f_2)f_3$.
  \\\\Induction: Suppose that all parenthesizations of $f_1,\, \ldots\, f_{j-1}$ under composition are equivalent for all $1 \leq j < n$.
  Then, for some $1 < k \leq n$, let $\alpha$ be some parenthesization of $f_1, \ldots, f_{k-1}$, and let $\beta$ be some parenthesization of $f_k, \ldots, f_{n}$.
  Any parenthesization of $f_1, \ldots, f_n$ will then be of the form $\alpha\beta$.
  By associativity and our inductive hypothesis, we have $\alpha=\left( (f_k \ldots f_{n-1})f_n \right)$, and so
  \begin{align*}
    \alpha\beta &= (f_1\ldots f_{k-1})\left( (f_k\ldots f_{n-1})f_n \right) \\
    &= \left( (f_1 \ldots f_{k-1})(f_k \ldots f_{n-1}) \right)f_n \\
    &= ((\ldots((f_1f_2)f_3)\ldots)f_n
  \end{align*}
  as desired.
\end{solution}
\newpage
\begin{problem}{4.2}
  In Example 3.3 we have seen how to construct a category from a set endowed with a relation, provided this latter is reflexive and transitive.
  For what types of relations is the corresponding category a groupoid?
\end{problem}
\begin{solution}
  Recall that a \textit{groupoid} is a category in which every morphism is an isomorphism. \\
  Let C be a category as defined in Example 3.3, and let $(S, \sim)$ be the category's designated set and relation.
  C is a groupoid if $\sim$ is symmetric.
  \begin{sproof}
    Let $(a,b)$ be a morphism from $a$ to $b$ in C. 
    By our definition of C, we have $a\sim b$.
    Since $\sim$ is symmetric, we then have $b\sim a$, and so $(b,a)$ is also a morphism in C (from $b$ to $a$).
    Composing these, we have $(a,b)(b,a)=(b,b)=\text{id}_b$.
    Similarly, we also have $(b,a)(a,b)=(a,a)=\text{id}_a$, making $(a,b)$ an isomorphism as desired.
  \end{sproof}
\end{solution}

\begin{problem}{4.3}
  Let $A,B$ be objects of a category C, and $f\in \text{Hom}_C(A,B)$ a morphism.
\end{problem}
\begin{solution}.
  \begin{enumerate}
    \item If $f$ has a right-inverse, then $f$ is an epimorphism.
      \begin{sproof}
        Let $f\in \text{Hom}_C(A,B)$ be a morphism, $g\in \text{Hom}_C(B,A)$ its right-inverse, and $\alpha_1, \alpha_2 \in \text{Hom}_C(A,Z)$ morphisms for some $Z\in \text{Obj}(C)$ with $\alpha_1f=\alpha_2f$.
        We then have
        \begin{align*}
          \alpha_1 &= \alpha_1(fg)\\
          &= (\alpha_1f)g\\
          &= (\alpha_2f)g\\
          &= \alpha_2(fg)\\
          &= \alpha_2
        \end{align*}
        making $f$ an epimorphism.
      \end{sproof}
    \item The converse of 1 does not hold; that is, there exists in some category C an epimorphism that does not have a right-inverse.
      \begin{sproof}
        The category obtained by endowing $\mathbb{Z}$ with the relation $\leq$ contains morphisms that satisfy this property.
        Let \catname{C} be this category; $a, b \in \Obj{C}$ such that $a\neq b$ (so $a < b$); $f \in \Hom(a, b)$; $z \in \Obj{C}$ such that $b \leq z$; and $\alpha_1, \alpha_2 \in \Hom(b, z)$. 
        That $\alpha_1f = \alpha_2f$ implies $\alpha_1=\alpha_2$ is trivially true since \Hom(b, z) has exactly one morphism, so $f$ is an epimorphism. \\
        However, since $a < b$, $b > a$, meaning \Hom(b,a) has no morphisms. Thus, $f$ has no right-inverse.

      \end{sproof}
  \end{enumerate}
\end{solution}

\newpage

\begin{problem}{4.4}
  Prove that the composition of two morphisms is a monomorphism. 
Deduce that one can define a subcategory $\catname{C}_{\text{mono}}$ of a category \catname{C} by taking the same objects as in \catname{C}, and defining $\Hom_{\catname{C}_\text{mono}}(A,B)$ to be the subset of $\Hom_{\catname{C}}(A,B)$ consisting of monomorphisms, for all objects $A,B$. 
  Do the same for epimorphisms. 
  Can you define a subcategory $\catname{C}_{\text{nonmono}}$ of \catname{C} by restricting to morphisms that are \textit{not} monomorphisms? 
\end{problem}
\begin{solution}
    Let \catname{C} be a category; 
      $A, B, C \in \Obj{C}$ be objects in C; and
      $f\in\Hom(A,B)$,
      $g\in\Hom(B,C)$ be morphisms in C.
      If $f$ and $g$ are monic, then $gf$ is also monic.
      \begin{sproof}
        Since $f$ and $g$ are monic, we have, for all $Z_1, Z_2 \in \Obj{C}$,
        $\alpha_1, \beta_1 \in \Hom(B,Z_1)$, and
        $\alpha_2, \beta_2 \in \Hom(C, Z_2)$:
        \begin{align*}
          \alpha_1f = \beta_1f \implies \alpha_1 = \beta_1\\
          \alpha_2g = \beta_2g \implies \alpha_2 = \beta_2
        \end{align*}
        We then have:
        \begin{alignat*}{3}\relax
          \alpha_2(gf) = \beta_2(gf) &\implies (\alpha_2g)f &&= (\beta_2g)f\\
          &\implies \alpha_2g &&= \beta_2g\\
          &\implies \alpha_2 &&= \beta_2
        \end{alignat*}
        making $gf$ monic as desired.
      \end{sproof}
      
      With this, we can define a category $\catname{C}_\text{mono}$ by 
      \begin{equation*}
        \Objj(\catname{C}_\text{mono}) = \Obj{C}
      \end{equation*} 
      \[\text{and} \]
      \begin{equation*}
        \Hom_{\catname{C}_\text{mono}}(A,B) = \{f \in \Hom_{\catname{C}}(A,B)\ \mid f \text{ is monic}\}
      \end{equation*} 
      for all $A, B \in \Objj(\catname{C}_\text{mono})$.
      Composition of morphisms is defined as normal (since we've proved monomorphisms are closed under composition, and the indentities are those in \catname{C} since identities are trivially monic. \\
        The proof for epimorphisms is analogous.\\
        Non-monomorphisms, however, do not form a category:
        \begin{proposition}
          Let \catname{C} be a category;
          $A,B,C,Z$ be objects in \catname{C}, and
          $f\in \Hom_{\catname{C}}(A,B)$ and
          $g\in \Hom_{\catname{C}}(B,C)$
          be nonmonomorphisms. Then there exist morphisms
          $\alpha_1, \alpha_2 \in \Hom_{\catname{C}}(C,Z)$ such that
          $\alpha_1(gf) = \alpha_2(gf)$ but $\alpha_1\neq \alpha_2$.
        \end{proposition}
        \begin{sproof}
          Since $g$ is not monic, there exist $\alpha_1, \alpha_2 \in \Hom_{\catname{C}}(C,Z)$ such that $a=\alpha_1g = \alpha_2g$ but $\alpha_1\neq \alpha_2$. Let $\alpha_1$ and $\alpha_2$ be these morphisms. Then, we have
          \begin{align*}
            \alpha_1(gf) &= (\alpha_1g)f\\
            &= (\alpha_2g)f \\
            &= \alpha_2(gf)
          \end{align*}
          but $\alpha_1\neq \alpha_2$. Thus, $gf$ is not monic.
        \end{sproof}
\end{solution}

\end{document}
